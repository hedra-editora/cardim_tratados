\hyphenation{goe-la Troue-ssart}

\newcommand{\especie}[1]{\textsc{#1}\index{#1}}

\chapter[Do clima e terra do Brasil e de algumas cousas notáveis que
se acham na terra como no mar]{Do clima e terra do Brasil\subtitulo{e de algumas cousas notáveis que
se acham na terra como no mar\protect\footnote[*]{ \NoCaseChange{ Estes textos do Padre Fernão Cardim foram publicados pela primeira
vez, no ano de 1625, em Londres, incluídos na famosa coleção de
viagens \textit{Purchas his Pilgrimes}, sob o título \textit{A Treatise
of Brasil written by a Portugal which had long live there.} Cf. Samuel
Purchas, \textit{Purchas, his Pilgrimes}, ``The Seaventh Booke'' -- 
``Voyages to and about the Southerne America, with many Marine
Observations and Discourses of those Seas and Lands by Englishmen and
others'', London, 1625. Os manuscritos encontram"-se, em português, na
Biblioteca Pública e Arquivo Distrital de Évora, mencionados no já
referenciado Catálogo de Cunha Rivara, fazendo parte de uma miscelânea
de documentos do século \textsc{xvi}, com o título genérico de \textit{Cousas do
Brasil}, códice \textsc{cxvi}/1--33, fls. 13--34. Vide ``Introdução'' desta obra, p. \pageref{cunharivara}.}}}}

\hedramarkboth{Do Clima e Terra do Brasil}{Fernão Cardim}

\paragraph{Do clima e terra do Brasil}

O clima do Brasil geralmente é temperado de bons, delicados, e
salutíferos ares, donde os homens vivem muito até noventa, cento e
mais anos, e a terra é cheia de velhos; geralmente não tem frios, nem
calmas, ainda que no Rio de Janeiro até São Vicente há frios, e calmas,
mas não muito grandes; os céus são muito puros e claros,
principalmente de noite, a lua é mui prejudicial à saúde, e corrompe
muito as cousas; as manhãs são salutíferas, têm pouco de crepúsculos,
assim matutinos, como vespertinos, porque, em sendo manhã, logo se sai
o sol, e em se pondo logo anoitece. O inverno começa em março, e acaba
em agosto, o verão começa em setembro e acaba em fevereiro; as noites e
dias são quase todo o ano iguais. 

 A terra é algum tanto melancólica, regada de muitas águas, assim de
rios caudais, como do céu, e chove muito nela, principalmente no
inverno; é cheia de grandes arvoredos que todo o ano são verdes; é
terra montuosa, principalmente nas fraldas do mar, e de Pernambuco até
à Capitania do Espírito Santo se acha pouca pedra, mas daí até S.
Vicente são serras altíssimas, mui fragosas, de grandes penedias e
rochedos. Os mantimentos e águas são geralmente sadios, e de fácil
digestão. Para vestir há poucas comodidades por não se dar na terra
mais que algodão, e do mais é terra farta, principalmente de gados e
açúcares. 

\vspace{\baselineskip}

\paragraph{Dos animais}

\especie{Veado} Na língua brasílica se chama \textit{Sugoaçu};\footnote{ O veado 
é a denominação comum a mamíferos artiodáctilos
da família dos Cervídeos, que podem apresentar chifres ramificados ou
simples. Existem várias espécies no território brasileiro e o próprio
Cardim assim o testemunha. São de referir o \textit{suaçuapara} ou
``veado"-campeiro'' (\textit{Ozotocerus bezoarticus}), o \textit{suaçuca
atinga} ou ``veado"-catingueiro'' (\textit{Mazama simplicicornis}), o
\textit{suacuetê} ou ``veado"-mateiro'' (\textit{Mazama americana}) ou
ainda o cervo"-do"-pantanal (\textit{Blastocerus dichotomus).} É
denominado em tupi de \textit{Sugoaçu} ou \textit{Suaçu} composto de
\textit{çoo} = ``animal'' + \textit{guaçu} = ``grande'', ou seja, ``o animal
grande'', ``a caça mais avultada''. O termo tupi ocorre a primeira vez num
texto português com Fernão Cardim.} há uns muito grandes,
como formosos cavalos; têm grande armação, e alguns têm dez e doze
pontas; estes são raros, e acham"-se no Rio de S. Francisco e na
Capitania de S. Vicente; estes se chamam \textit{Suaçuapara},\footnote{ No caso de 
\textit{Suaçuapara} a significação do tupi é
de ``veado de cornos'', já que \textit{apará} como adjetivo designa ``o
que verga'', ``vergado'', ``curvo'', ``contorto''.} são
estimados do Carios,\footnote{ \textit{Carios} é a grafia
utilizada pelos autores espanhóis para o nome do grupo tribal
Tupi"-Guarani, os Carijós, que dominavam o litoral brasileiro de
Cananéia para Sul, hoje denominados de Guarani.} e das pontas e nervos
fazem os bicos das flechas, e umas bolas de arremesso que usam para
derrubar animais ou homens.

Há outros mais pequenos; também têm cornos, mas de uma ponta só. Além
destes há três ou quatro espécies, uns que andam somente nos matos,
outros somente nos campos em bandos. Das peles fazem muito caso, e da carne. 

\especie{Tapyretê}\footnote{ \textit{Tapyretê} é a anta
(\textit{Tapirus americanus}, Briss), um mamífero
perissodáctilo, da família dos Tapirídeos, que ocorre desde a Colômbia
até à Argentina. É estritamente herbívoro, tem o nariz alongado em
curta tromba móvel e apresenta quatro dedos nas patas dianteiras e três
nas traseiras. Vive perto de rios e de lagoas. O nome tupi é
susceptível de várias explicações, entre as quais a aplicação do sufixo
\textit{etê} = ``verdadeiro'',``legítimo'', que serviu para diferenciar o
ungulado do bovino, que os índios só conheceram depois do contato
europeu, e ao qual chamaram de \textit{tapyra.} Este termo ocorre a
primeira vez num texto português em 1584, na \textit{Informação dos
Casamentos dos índios do Brasil}, do Pe. José de Anchieta.} 
Estas são as antas, de cuja pele se fazem as adargas; parecem"-se com
vacas e muito mais com mulas, o rabo é de um dedo, não têm cornos, têm
uma tromba de comprimento de um palmo que encolhe e estende. Nadam e
mergulham muito, mas em mergulhando logo tomam fundo, e andando por ele
saem em outra parte. Há grande cópia delas nesta terra.

\especie{Porco montês} Há grande cópia de porcos monteses, e é o
ordinário mantimento dos índios desta terra, têm o embigo nas costas e
por ele sai um cheiro, como de raposinho, e por este cheiro os seguem
os cães e são tomados facilmente. Há outros chamados
\textit{Tayaçutirica},\footnote{ \textit{Tayassú}, porco
montês ou porco"-do"-mato, é o ungulado artiodáctilo da família dos
Suídeos, gênero Tayaçu. No Brasil existem duas espécies, como descreve
Fernão Cardim, \textit{Tauaçu albirostris}, Cuv., que é a maior, cuja
denominação tupi é ``dente grande'', isto é, ``o porco do mato'', a ``queixada'' 
e a outra espécie é \textit{Tayaçu tayaçu}, Cuv., que também
se denomina de \textit{caititú} ou \textit{cateto.} Estas devem ser as
descritas por Cardim, ainda que a significação atribuída não esteja
muito de acordo com a etimologia do tupi, \textit{tayaçu} = ``porco'' +
\textit{tirica} = ``medroso'', ``que foge'', ``tímido'' \textit{tayaçu} = 
``porco'' + \textit{pigta} = ``vermelho''. Os termos tupi \textit{taiaçu}
e \textit{taiaçuetê} ocorrem a primeira vez num texto português em
1587, na \textit{Notícia do Brasil}, de Gabriel Soares de Sousa; por
sua vez os termos \textit{taiaçupita} e \textit{taiaçutirica} ocorrem
pela primeira vez neste texto de Cardim.} sc.\footnote{ (sc.) = \textit{scilicet} (``isto é'', em latim), abreviatura utilizada pelo autor 
ao longo dos textos, que mantivemos.} porco que bate, e trinca os
dentes, estes são maiores que os comuns, e mais raros, e com seus
dentes atassalham quantos animais acham.

 Outros se chamam \textit{Tayaçupigta}, sc.~porco que aguarda, ou faz
finca"-pé. Estes acometem os cães, e os homens, e tomando"-os os comem, e
são tão bravos que é necessário subirem"-se os homens nas árvores para
lhes escapar, e alguns esperam ao pé das árvores alguns dias até que o
homem se desça, e porque lhes sabem esta manha, sobem"-se logo com os
arcos e flechas às árvores e de lá os matam. 

Também há outras espécies de porcos, todos se comem, e são de boa substância.

\especie{Acuti} ou \textit{Cutia}\footnote{ \textit{Acuti} ou
\textit{cutia} é o nome atribuído a três espécies de mamíferos
roedores da família dos Dasiproctídeos, de 50 a 60\,cm, com pelo áspero e
cauda rudimentar, como o \textit{Dasyprocta aguti}, L. que ocorre do
norte do Brasil até ao Rio de Janeiro e Minas Gerais, o \textit{D.
azarae}, no sul do Rio de Janeiro, Minas Gerais e Paraguai e o
\textit{D. leporina}, que existe na América do Sul. Segundo o
termo tupi é um animal roedor, que comia depressa, o que confirma a
descrição de Cardim. Assim \textit{a} = ``gente'' + \textit{cur"-ti} = 
``modo de comer ou tragar com as patas dianteiras''. O termo tupi ocorre
pela primeira vez num texto português em 1576, com Pêro de Magalhães de
Gândavo, na \textit{História da Província Santa Cruz.} Mas coube a
André Thevet, nas \textit{Singularités de la France Antarctique}, em
1557, a primeira descrição deste animal, que chamou
\textit{agoutin.}} Estas Acutis se parecem com os coelhos de
Espanha, principalmente nos dentes: a cor é loura, e tira a amarela;
são animais domésticos, e tanto que andam por casa, e vão fora, e
tornam a ela; quando comem tudo tomam com as mãos e assim o levam à
boca, e comem muito depressa, e o que lhes sobeja escondem para quando
têm fome. Destas há muitas espécies, todas se comem. 

\especie{Paca}\footnote{ \textit{Paca} é um mamífero
roedor da família dos Cuniculídeos (\textit{Cuniculus paca, L.}) que
tem patas curtas, cauda rudimentar e vive em regiões próximas de água.
A sua carne é muito apreciada ainda hoje como alimento. O termo tupi
vem do verbo \textit{pag} = ``acordar'', ``despertar'', expresso no
gerúndio"-supino \textit{paca} = ``a esperta'', a ``vívida''. Este nome
ocorre pela primeira vez num texto português em 1576, no
\textit{Tratado da Província do Brasil}, de Pêro de Magalhães de Gândavo.}
Estas Pacas são como leitões, e há grande abundância delas:
a carne é gostosa, mas carregada; não parem mais que um só filho. Há
outras muito brancas, são raras, e acham"-se no Rio de São Francisco. 

\especie{Iagoaretê}\footnote{ \textit{Iagoaretê, Jaguareté} ou
\textit{onça pintada } é a espécie típica do gênero \textit{Felis}, da
família dos Felídeos, das quais existem no Brasil nove espécies. A
descrita por Cardim é a onça pintada, \textit{Felis onça}, L., que é de
todas a maior. O nome tupi \textit{jaguaretê} é composto de
\textit{jaguar} = ``onça'', ``cão'' + \textit{etê} = ``verdadeiro''. Este
termo ocorre pela primeira vez num texto português com Fernão Cardim.} 
Há muitas onças, umas pretas, outras pardas, outras pintadas: é animal
muito cruel, e feroz; acometem os homens sobremaneira, e nem em
árvores, principalmente se são grossas, lhes escapam; quando andam
cevadas de carne não há quem espere principalmente de noite; matam logo
muitas reses juntas, desbaratam uma casa de galinhas, uma manada de
porcos, e basta darem uma unhada em um homem, ou qualquer animal para o
abrirem ao meio; porém são os índios tão ferozes que arremete com uma,
e tem mão nela e depois a matam em terreiro como fazem aos contrários,
tomando nome, e fazendo"-lhes todas as cerimônias que fazem aos mesmos
contrários. Das cabeças delas usam por trombetas, e as mulheres
Portuguesas usam das peles para alcatifas, \textit{maxime}\footnote{ ``Principalmente'', em latim. [\versal{N.}~do \versal{E.}]} das pintadas, e na
capitania de São Vicente.

\especie{Sarigué}\footnote{ \textit{Sarigué, sariguê, sarué,
mucurá} e \textit{gambá} são os nomes que na sinonímia popular
significam as espécies de marsupiais da família dos Didelfídeos, que
usualmente se designa por gambá, o \textit{Didelphis aurita}, 
L., ou ``raposas de Espanha'', como menciona Cardim. A
palavra tupi vem de \textit{çoó"-r"-iguê} = ``animal de saco ou bolsa'' 
(em referência ao saco em que cria os filhotes) e ocorre pela primeira
vez num texto português em 1560, numa carta do Pe. José de
Anchieta. Mas já em 1535, Oviedo na \textit{Historia natural y general
de las Indias} tinha descrito este animal, que desde aí passou a
figurar com o nome indígena em todos os tratados das regiões
americanas.} Este animal se parece com as raposas de Espanha,
mas são mais pequenos, do tamanho de gatos; cheiram muito pior a
raposinhos que as mesmas de Espanha, e são pardos como elas. Têm uma
bolsa das mãos até às pernas com seis ou sete mamas, e ali trazem os
filhos escondidos até que sabem buscar de comer, e parem de ordinário
seis, sete. Estes animais destroem as galinhas porque não andam de dia,
senão de noite, e trepam pelas árvores e casas, e não lhes escapam
pássaros, nem galinhas. 

\especie{Tamanduá}\footnote{ \textit{Tamanduá} é a designação
usual para os mamíferos desdentados, da família dos Mirmecofagídeos,
que se alimentam, principalmente, de formigas e cupins. Este seu hábito
e o aspecto exótico causaram grande admiração entre os primeiros
cronistas do Brasil, não deixando de o descrever minuciosamente. O
termo tupi ocorre pela primeira vez em 1560, numa carta do Pe.
José de Anchieta. Este tem sido explicado de várias formas,
considerando"-se que \textit{ta =} contração de \textit{tacy} = 
``formiga'' + \textit{monduar} = ``caçador'', ou seja, ``caçador de
formigas''.} Este animal é de natural admiração: é do tamanho
de um grande cão, mais redondo que comprido; e o rabo será de dois
comprimentos do corpo, e cheio de tantas sedas, que pela calma, e
chuva, frio, e ventos, se agasalha todo debaixo dele sem lhe aparecer
nada; a cabeça é pequena, o focinho delgado, nem tem maior boca que de
uma almotolia, redonda, e não rasgada, a língua será de grandes três
palmos de comprimento e com ela lambe as formigas de que somente se
sustenta: é diligente em buscar os formigueiros, e com as unhas, que
são do comprimento dos dedos da mão de um homem o desmancha, e deitando
a língua fora pegam"-se nela as formigas, e assim a sorve porque não tem
boca para mais que quanto lhe cabe a língua cheia delas; é de grande
ferocidade, e acomete muito a gente e animais. As onças lhe dão medo,
nem prestam para mais que para desancar os formigueiros, e são eles
tantos, que nunca estes animais os desbaratarão de todo.

\especie{Tatu}\footnote{ \textit{Tatu} é o nome genérico dos
desdentados da família dos Dasipodídeos, dos quais cerca de 24 
espécies existem no Brasil. Vivem em galerias subterrâneas,
alimentam"-se de insetos, larvas e vermes. Apresentam uma carapaça
dorsal, córnea, protetora, apoiada sobre placas ósseas, na pele, em
geral dividida por sulcos transversais de pele mais mole para que o
animal se possa dobrar. Tal como o anterior, a sua aparência exótica
despertou a atenção e admiração dos primeiros escritores que
descreveram a fauna brasílica. O termo é tupi, \textit{ta"-tu} que
significa ``casca encorpada ou densa'', ``encouraçada'', ocorre pela
primeira vez num texto português em 1560, numa carta do Pe.
José de Anchieta.} Este animal é do tamanho de um leitão, de
cor como branca, o focinho tem muito comprido, o corpo cheio de unhas
como lâminas com que fica armado, e descem"-lhe uns pedaços como têm as
Badas. Estas lâminas são tão duras que nenhuma flecha as pode passar se
lhe não dá pelas ilhargas; furam de tal maneira, que já aconteceu vinte
e sete homens com enxadas não poderem cavar tanto, como uma cavava com
o focinho. Porém, se lhe deitam água na cova logo são tomados; é animal
para ver, e chamam"-lhe cavalo armado: a carne parece de galinha, ou
leitão, muito gostosa, das peles fazem bolsas, e são muito galantes, e
de dura; fazem"-se domésticos e criam"-se em casa.
 Destes há muitas espécies e há grande abundância.

\especie{Canduaçu}\footnote{ \textit{Canduaçu, Coandu} ou
\textit{Candumini} são os nomes para designar as várias espécies de
mamíferos roedores, da subordem dos histricomorfos, da família dos
Eretizontídeos, cuja espécie maior é a \textit{Coendu villosus}, 
Licht, vulgarmente denominados de \textit{ouriço"-cacheiro.} São
caracterizados por terem o corpo guarnecido de espinhos e a capacidade
de se enrolar formando uma bola. O nome tupi \textit{cuandu} pode
derivar"-se de \textit{guã} = ``pelo'' + \textit{tu =} alteração de
\textit{mbo"-tu} = ``bater'', ou \textit{ty} = ``elevado'', ``erguido''. Este
termo tupi ocorre pela primeira vez num texto português com Cardim. É,
ainda, designado por \textit{cuim} em outros autores, como em 1587, na
\textit{Notícia do Brasil}, de Gabriel Soares de Sousa.} Este
animal é o porco"-espinho de África: tem também espinhos brancos e
pretos tão grandes que são de palmo e meio; e mais; e também os
despedem como os de África.

 Há outros destes que se chamam Candumiri,\footnote{ \textit{Candumiri} 
é o mesmo animal só que como Cardim menciona é de pequeno porte, o
que coincide com o nome tupi já que o sufixo \textit{mirim} é o
formativo de diminutivos.} por serem mais pequenos, e também
têm espinhos da mesma maneira.

 Há outros mais pequenos do tamanho de gatos, e também têm espinhos
amarelos e nas pontas pretos. Todos estes espinhos têm esta qualidade
que entrando na carne, por pouco que seja, por si mesmo passam a carne
de parte a parte, e por esta causa servem estes espinhos de
instrumentos aos índios para furar as orelhas, porque, metendo um pouco
por elas, em uma noite as fura de banda a banda.

 Há outros mais pequenos, como ouriços, também têm espinhos, mas não nos
despedem; todos estes animais são de boa carne e gosto. 

\especie{Eirara}\footnote{ \textit{Eirara, irara} ou
\textit{papa"-mel} é um carnívoro da família dos Mustelídeos
(\textit{Tayra Barbara}, L.). A cor é usualmente parda, com uma mancha
amarelada na garganta. O nome tupi deriva de \textit{ira} ou
\textit{eira} = ``mel'' + \textit{ra} = ``tomar'', ``colher'', assim será ``aquele 
que colhe mel'', o ``papa"-mel'', que é de fato o apelido que 
lhe cabe já que tem por hábito lascar os troncos das árvores onde se
encontram os ninhos de Meliponídeos ou o mel"-de"-pau, de que se
alimenta. O termo tupi ocorre pela primeira vez num texto português com
Cardim.} Este animal se parece com gato"-de"-Algália:\footnote{ O 
\textit{gato"-de"-Algália} é a designação por que é conhecido um animal
quadrúpede semelhante à marta, também designado de
\textit{almiscareiro.}} ainda que alguns dizem que o não é, são de
muitas cores, sc.~pardos pretos, e brancos: não comem mais que mel, e
neste ofício são tão terríveis que por mais pequeno que seja o buraco
das abelhas o fazem tamanho que possam entrar, e achando mel não no
comem até não chamar os outros, e entrando o maior dentro não faz senão
tirar, e dar aos outros, cousa de grande admiração e exemplo de
caridade para os homens, e ser isto assim afirmam os índios naturais.

\especie{Aquigquig}\footnote{ \textit{Aquigquig}: não há
referência a este animal na sinonímia vulgar, mas tratando"-se de um
bugio, como descreve Cardim, é um macaco ululador, originário da
América tropical. Pode"-se relacionar com o \textit{buriqui} ou
\textit{muriqui}, símio da família dos Cebídeos (\textit{Eriodes 
arachnoides}, Cuv.) que é o maior macaco existente no território
brasileiro. Aparece em outros textos da época designado por
\textit{guigo}, que ainda hoje é a denominação local baiana para certa
espécie de saguis grandes. O termo tupi \textit{aquiqui} ocorre pela
primeira vez num texto português com Fernão Cardim.} Estes
bugios\footnote{ Bugio é a designação genérica dos macacos
ululadores originários da América tropical, da família dos Cebídeos, 
do gênero \textit{Alouatta}, tais como o bugio"-preto (\textit{Alouatta
caraya}), guariba"-preto e o bugio"-ruivo, ou barbado 
(\textit{Alouatta fusca}) que têm cauda preênsil, são arborícolas e
alimentam"-se de folhas e frutos.} são muito grandes como um bom cão,
pretos, e muito feios, assim os machos, como as fêmeas, têm grande
barba somente no queixo debaixo, nasce às vezes um macho tão ruivo que
tira a vermelho, o qual dizem que é seu Rei. Este tem o rosto branco, e
a barba de orelha a orelha, como feita à tesoura; têm uma cousa muito
para notar, e é, que se põem em uma árvore, e fazem tamanho ruído que
se ouve muito longe, no qual atura muito sem descansar, e para isto tem
particular instrumento esta casta: o instrumento é certa cousa côncava
como feita de pergaminho muito rijo, e tão lisa que serve para brunir,
do tamanho de um ovo de pata, e começa do princípio da goela até junto
da campainha, entre ambos os queixos, e é este instrumento tão ligeiro
que em lhe tocando se move como a tecla de um cravo. E, quando este
bugio assim está pregando espuma muito, e um dos pequenos que há"-de
ficar em seu lugar lhe limpa muitas vezes a espuma da barba.

 Há outros de muitas castas, e em grande multidão sc.~pretos, pardos,
amarelos; dizem os naturais que alguns destes quando lhes atiram uma
flecha a tomam na mão e tornam com ela a atirar à pessoa; e quando os
ferem buscam certa folha e a mastigam, e metem na ferida para sararem:
e porque andam sempre nas árvores, e são muito ligeiros, quando o salto
é grande que os pequenos não podem passar, um deles se atravessa como
ponte, e por cima dela passam os outros, o rabo lhe serve tanto como
mão, e se algum é ferido com o rabo se cinge, e ao ramo onde está, e
assim fica morrendo dependurado sem cair. Têm outras muitas habilidades
que se veem cada dia, como é tomar um pau, e dar pancadas em alguém que
lhes faz mal; outro achando um cestinho de ovos e dependurou pela corda
ao pescoço, e subindo a um telhado fazia de lá muitos momos ao senhor
que o ia buscar, e quebrando"-os os sorveu todos diante dele,
atirando"-lhe as cascas. 

\especie{Coati}\footnote{ \textit{Coati} ou \textit{cuati} é
um mamífero carnívoro da família dos Procionídeos, que existe no sul do
território brasileiro, o \textit{Nasua narica}, L. e no norte, o
\textit{Nasua nasua}, Wied. O nome tupi pode ser traduzido por
\textit{agua} = ``ponta'' + \textit{tî} = ``nariz'', ou seja, ``nariz de
ponta'', ``nariz pontiagudo'', ``focinho'', o que condiz com a descrição
cardiniana, que é a primeira vez num texto português.} Este animal
é pardo, parece"-se com os texugos de Portugal, tem o focinho muito
comprido, e as unhas; trepam pelas árvores como bugios, não lhes
escapam cobra, nem ovo, nem pássaro, nem quanto podem apanhar; fazem"-se
domésticos em casa, mas não há quem os sofra, porque tudo comem, brincam
com gatinhos, e cachorrinhos, e são maliciosos, aprazíveis, e têm
muitas habilidades.

 Há outras duas, ou três castas maiores, como grandes cães, e têm dentes
como porcos javalis de Portugal; estes comem animais e gente, e achando
presa, acercam uns por uma parte, outros por outra, até a despedaçarem.

\especie{Gatos"-bravos}\footnote{ \textit{Gatos bravos} ou
\textit{gatos"-do"-mato} é a designação coletiva para os Felídeos
menores do gênero \textit{Felis}, (\textit{Felis tigrine, Felis wiedii e felis
geoffroyi}) de pequeno porte, com pelagem lisa ou manchada.} 
Destes há muitas castas, uns pretos, outros brancos açafroados,\footnote{ O autor utiliza a expressão ``assafroados'', 
possivelmente para especificar a cor do animal branca, mas com tons de
açafrão, pelo que será de escrever ``açafroados''.} e são muito
galantes para qualquer forro; são estes gatos muito terríveis e
ligeiros: vivem de caça e pássaros, e também acometem a gente; alguns
são tamanhos como cães. 

\especie{Iaguaruçu}\footnote{ \textit{Iaguaruçu, joquara"-guaçu} ou 
\textit{guará} (como se diz por abreviação no Brasil) é da
família dos Canídeos, da qual é o maior dos representantes (\textit{Canis jubatus}, Desm.). Também é designado por
``cachorro"-do"-mato'' (\textit{Dusicyon thous}) e podem ser encontrados
em quase todas as matas da América do Sul. Existem muitas espécies no
território brasileiro, como o \textit{Canis cancrivorus}, Desm.,
\textit{Canis microtis}, Mivart, \textit{Canis azarae}, Wied,
\textit{Canis urostictus}, Mivart, \textit{Canis parvidens}, Mivart e o
\textit{Canis venaticus}, Lund, segundo o \textit{Catalogus Mammalium}, de Trouessart, Paris, 1898. O termo tupi vem de \textit{jaguar} = 
``onça'' + \textit{uçu}, por \textit{açu} = ``grande'', logo ``onça grande'' 
e \textit{guará} ocorre pela primeira vez, para designar
este mamífero carnívoro em 1618, no \textit{Diálogo das Grandezas do
Brasil.} Note"-se a alusão de Cardim de que o cão não existia no Brasil,
para onde foi levado pelos colonos portugueses e por quem rapidamente
os índios se afeiçoaram, devido, em grande parte, ao seu contributo na
caça.} Estes são os cães do Brasil, são de um pardo
almiscarado de branco, são muito ligeiros, e quando choram parecem cães;
têm o rabo muito felpudo, comem frutas e caça, e mordem terrivelmente.

\especie{Tapiti}\footnote{ \textit{Tapiti}, mamífero lagomorfo
roedor da família dos Leporídeos (\textit{Lepus brasiliensis}, Briss.), 
também chamado de coelho, ou lebre, dadas as semelhanças
entre esses animais. Aparece designado, noutros autores contemporâneos
de Cardim, por \textit{Tapotim} ou \textit{Tapeti.} Pode ainda ser
denominado de ``coelho"-do"-mato'' (\textit{Sylvilagus brasiliensis}) e
ocorre, com várias subespécies, em campos e beira de matas, em todo o
Brasil. O termo \textit{tapiti} ou \textit{tapeti}, para designar um
animal semelhante ao coelho, ocorre pela primeira vez num
texto português com Cardim.} Este animal se parece com os
coelhos de Portugal, estes ladram cá nesta terra como cães,
\textit{maxime} de noite, e muito amiúde. Os índios têm estes ladridos
por agouro; criam três e quatro filhos; são raros porque têm muitos
adversários, como aves de rapina, e outros animais que os comem.

\especie{Iaguacini}\footnote{ \textit{Iaguacini} ou
\textit{guaxinim} é um carnívoro da família dos Procionídeos 
(\textit{Procyon cancrivorus}, Cuv.) que também é designado por
``mão"-pelada'', das quais existem sete espécies no território
brasileiro. Vivem em florestas, perto de água e lavam os alimentos
antes de comê"-los. Devido a este costume também são denominados de ``urso"-lavadeiro''. 
O nome tupi é explicado por \textit{gua"-chi"-ni} = ``aquele que rosna'', 
o ``roncador'', alusão ao hábito desse animal de
roncar quando se lhe tocava na cauda. O termo tupi \textit{jaguacininga}
ocorre pela primeira vez num texto português com Cardim.} Este animal 
é tamanho como raposa de Portugal, tem a mesma cor de
raposa, sustenta"-se somente de caranguejos, e dos canaviais de açúcar,
e destroem muitos deles; são muito dorminhocos, e dormindo os matam,
não fazem mal. 

\especie{Biarataca}\footnote{ \textit{Biarataca, jaritataca,
jaraticaca} ou \textit{maritataca} é um mamífero marsupial carnívoro
da família dos Mustelídeos (\textit{Conepatus suffocans}, Azara),
também chamado de ``gambá'', ``cangambá'' ou ``zorrilho''. O nome deve"-se ao
fato de este animal produzir uma secreção anal que expele para
defender"-se, de tal sorte nauseabunda, que afugenta os perseguidores.
Segundo os naturalistas a substância que dá à secreção o repelente
cheiro é o sulfidrato de etila, mais conhecido pelo nome de mercaptan,
e quando são muito repetidas formam vapores esverdeados.} Este animal é 
do tamanho de um gato, parece"-se com Furão,\footnote{ Cardim menciona o 
Furão, da família dos Mustelídeos, porque a
semelhança é grande entre estes dois animais, só que este não emite o
cheiro forte e nauseabundo dos \textit{Biarataca.}} pelo lombo tem uma
mancha branca, e outra parda, que lhe ficam em cruz muito bem feita;
sustentam"-se de pássaros, e seus ovos, e outras cousas, \textit{maxime}
de âmbar, e gosta tanto dele que toda a noite anda pelas praias a
buscá"-lo, e onde o há ele é o primeiro; é muito temido, não
porque tenha dentes nem outra arma com que se defenda, mas dá certa
ventosidade tão forte, e de tão ruim, que os paus, pedras, e quando
diante de si acha, penetra, e é tanto que alguns índios morreram já de
tal fedor; já cão que a ele se achega, não escapa, e dura este cheiro
quinze, vinte e mais dias, e é tal que se dá esta ventosidade junto de
alguma aldeia logo se despovoa para não serem sentidos, cavam no chão,
e dentro dão a ventosidade, e a cobrem com a terra; e quando os acham
para não serem tomados, sua defensa é disparar aquela ventosidade.
 Há outras espécies destes animais que não têm tão mau cheiro; criam"-se
em casa, e ficam domésticos, e os índios os estimam.

\especie{Preguiça}\footnote{ \textit{Preguiça} é a designação
comum às espécies de desdentados da família dos Bradipodídeos que vivem
nas florestas tropicais. No Brasil existem sobretudo as espécies
preguiça"-de"-dois"-dedos ou unau (\textit{Choloepus didactylus}), a
preguiça"-de"-bentinho ou preguiça"-aí (\textit{Bradypus tridactylus}) e a
preguiça"-de"-coleira (\textit{Bradypus torquatus}).} A
preguiça que chamam do Brasil, é o animal para ver, parece"-se com cães
felpudos, os perdigueiros; são muito feios, e o rosto parece de mulher
mal toucada; têm as mãos e pés compridos, e grandes unhas, e cruéis,
andam com o peito pelo chão, e os filhos abraçados na barriga, por mais
que lhe deem, andam tão devagar que hão mister muito tempo para subir a
uma árvore, e por isso são tomados facilmente; sustentam"-se de certas
folhas de figueiras,\footnote{ A árvore de cujas folhas a preguiça
se alimenta é a imbaúba (\textit{Cecropia}, sp.).} e por isso não
podem ir a Portugal, porque como lhes faltam, morrem logo.

\especie{Ratos}\footnote{ Cardim refere"-se aos ratos de espécie
indígena da família dos Cricetídeos, vulgarmente chamados de ``ratos"-do"-mato''. 
As principais espécies pertencem aos gêneros
\textit{Oryzomys} (rato"-dos"-bambus, rato"-de"-fava, calunga,
rato"-de"-algodão), \textit{Holochilus} (rato"-de"-capim, rato"-capivara,
rato"-de"-cana) ou ainda \textit{Nectomys} (rato"-de"-água), entre
outros.} Nestas partes há grande número de ratos, e
haverá deles algumas dez, ou doze castas, uns pretos, outros ruivos,
outros pardos, todos se comem, e são gostosos, \textit{maxime} alguns
grandes são como coelhos; em alguns tempos são tantos que dando em uma
roça, a destroem. 

\paragraph{Das cobras que andam na terra e não têm peçonha}

\especie{Gibóia}\footnote{ \textit{Gibóia}, \textit{jiboya} ou 
\textit{jebóia} é uma das cobras da família dos Boídeos 
(\textit{Constrictor constrictor}, L.). O termo tupi \textit{yibói} é
explicado como ``cobra"-d'água'' ou ``cobra"-de"-pau'', atendendo a que a
jiboia é sempre terrestre, o segundo termo parece mais adequado. A
jiboia existe em todo o território brasileiro, sendo um dos maiores
ofídios. Não é venenosa, tem hábitos arborícolas, vive em campos e
florestas, alimentando"-se de roedores e aves. Este ofídio impressionou
bastante os primeiros cronistas do Brasil, que a ele se referem
pormenorizadamente. O termo tupi ocorre pela primeira vez num texto
português com Cardim.} Esta cobra é das grandes que por cá há,
e algumas se acham de vinte pés de comprido; são galantes, mas mais o
são em engolir um veado inteiro; não têm peçonha, nem os dentes são
grandes conforme ao corpo; para tomar a caça que se sustenta usa desta
manha: estende"-se pelos caminhos, e em perpassando a caça lança"-se
sobre ela, e de tal maneira se enrodilha, e aperta que lhe quebra
quantos ossos tem, e depois a lambe, e seu lamber tem tal virtude que a
mói toda, e então a engole, e traga. 

 Há outras que chamam \textit{Guigraupiagoara},\footnote{ \textit{Guigraupiagoara, 
papa"-ovo} ou \textit{papa"-pinto} é uma das
cobras da família dos Colubrídeos (\textit{Herpetodryas carinatus}, L.). 
A descrição de Cardim está adequada, já que a decomposição do
termo tupi é \textit{guirá =} ``pássaro'' + \textit{upiá} = ``ovo'' +
\textit{guara} = particípio do verbo \textit{ú} = ``o que come'',
``comedor'', logo ``o que come ovos e pássaros''. O termo tupi
\textit{guiraupiaguara} ocorre pela primeira vez num texto português
com Cardim.} sc.~comedora dos ovos dos pássaros, é muita preta,
comprida, e tem os peitos amarelos, andam por cima das árvores, como
nadando por água, e não há pessoa que tanto corra por terra, como elas
pelas árvores. Esta destrói os pássaros e seus ovos.

 Há outra muito grossa, e comprida, chamada \textit{Caninana}:\footnote{ \textit{Caninana} 
é uma cobra da família dos Colubrídeos (\textit{Spilotes pullatus}, L.) 
que aparece em alguns autores antigos com o nome de \textit{caninam.} O termo tupi utilizado neste texto
designa ``cobra não venenosa'' e ocorre pela primeira vez com Cardim.} é
toda verde, e de notável formosura. Esta também come ovos, pássaros, e mata os pintainhos.

 Há outra chamada \textit{Boitiapoá},\footnote{ \textit{Boitiapoá,
cobra"-de"-cipó} é uma cobra também da família dos Colubrídeos 
(\textit{herpetodryas Fucus}, L.). Existem várias cobras"-de"-cipó no
território brasileiro, com o corpo afilado e hábitos arborícolas. Os
índios tinham por tradição apoitar os assentos das mulheres estéreis
com estas cobras, como refere Cardim no seu texto. O termo tupi
\textit{boitiapuã} seria ``cobra de focinho redondo'', de \textit{boi} 
= ``cobra'' + \textit{tiapuã} = ``focinho comprido'', como Cardim traduz no
seu texto, que é onde ocorre pela primeira vez.} sc.~cobra que tem
focinho comprido, é muito delgada e comprida, e sustenta"-se somente de
rãs, têm os índios com esta um agouro que quando a mulher não tem
filhos tomam esta cobra, dando"-lhe com ela nas cadeiras e dizem que logo há"-de parir.

 Há outra chamada \textit{Gaitiepia},\footnote{ Não conseguimos
identificar esta espécie de cobra, que é apenas apresentada por Cardim
neste seu texto, que alega existir apenas no Rari. Consideramos ainda
que esta região pode ser Pari ou Ari, já que a designação cardiniana
parece não existir no território brasileiro. Se for \textit{Pari} é um
rio de Mato Grosso, mas se for \textit{Peri} é uma povoação do
Maranhão; mas pode ainda ser \textit{Ari} ou \textit{Airi}, 
que é a designação para um local com palmeiras denominadas de
\textit{airi}.} acha"-se somente no Rari; é de notável grandura,
cheira tanto a raposinhos que por onde quer que vai que não há quem a sofra.

 Há outra, a qual se chama \textit{Boyuna},\footnote{ \textit{Boyuna,
mussurana} ou \textit{cobra"-preta} é também da mesma família dos
Colubrídeos (\textit{Oxyrhopus cloetia}, Daud.). O termo tupi significa
exatamente ``cobra negra, preta'', de \textit{boi} = ``cobra'' +
\textit{una} = ``preta, negra''. Os índios criaram mesmo o mito da
\textit{Boiúna}, que existe sobretudo nas populações ribeirinhas da
Amazônia, onde foi identificada com a ``mãe"-de"-água'', entidade maléfica
que persegue e ataca os homens e os animais. É uma cobra gigantesca que
habita o rio e engole pessoas. Aparece de noite, com grandes olhos de
fogo para perseguir e virar embarcações. Pode ainda surgir como uma
embarcação fantasma de grandes ou pequenas dimensões, muito iluminada,
que segue outros barcos e muda bruscamente de direção quando se
aproxima, chegando a entrar pelo mato e a engolir as pessoas que
encontra. As origens deste mito são atribuídas, quer a uma jiboia que
se tenha desenvolvido mais e abandonando as florestas se tenha fixado
junto ao rio, quer a um ente nascido das relações entre uma mulher e um
fantasma mau. Era também denominada pelos indígenas de
\textit{paranamaia}, ``mãe"-do"-rio'' numa associação de água e serpente
muito usual nos povos ameríndios. O termo tupi ocorre pela primeira vez
num texto português com Cardim.} sc.~cobra preta, é muito comprida, e
delgada, também cheira muito a raposinhos.

 Há outra que se chama \textit{Bom},\footnote{ Não encontramos qualquer
referência a esta espécie de cobra, ainda que a descrição de Cardim não
nos pareça fora de usual, já que, por vezes, os índios atribuíam nomes
aos animais consoante os sons que emitiam, como parece ser este o caso.}
sc.~porque quando anda vai dizendo bom, bom, também é grande, e não faz mal.

 Há outra, a qual se chama \textit{Boicupecanga},\footnote{ \textit{
Boicupecanga}: também é uma cobra difícil de identificar pela
descrição de Cardim, e seguindo a formação da palavra em tupi o seu
étimo é em parte satisfatório, já que \textit{boi} = ``cobra'' +
\textit{cupé} = ``dorso'', ``costas'' + \textit{acanga =} que pode
significar ``ramo'', ``galho'', mas que não aparece nos dicionários 
tupi"-guarani com a acepção de espinho. Este termo ocorre pela primeira
vez num texto português com Cardim.} sc.~cobra que tem espinhos pelas
costas, é muito grande, e grossa, as espinhas são muito peçonhentas, e
todos se guardam muito delas.

\paragraph{Das cobras que têm peçonha}

\especie{Jararaca}\footnote{ \textit{Jararaca} é uma das cobras
venenosas da família dos Viperídeos (\textit{Lachesis lanceolatus}, 
Lacep.). É a designação comum para as cobras venenosas do gênero
\textit{Bothrops}, que atingem em média um metro de comprimento.
Existem por quase todo o território brasileiro com várias espécies como
Cardim menciona no seu texto. O termo tupi pode derivar de \textit{yara
+ roág} = ``que envenena quem agarra''. Este ocorre, pela primeira vez 
num texto português, em 1560, numa carta de José de Anchieta.
Por extensão ficou ainda na terminologia brasileira para designar
``pessoa de gênio irascível'' ou ``indivíduo intratável''.} 
Jararaca é nome que compreende quatro gêneros de cobras muito
peçonhentas: a primeira e maior, é \textit{Jararacuçu},\footnote{ \textit{Jararacuçu}: 
da mesma família da anterior (\textit{Lachesis
jararacuçu}, Lacerda). A designação em tupi é exatamente ``jararaca
grande'' de \textit{jararaca + uçu =} grande. Pode vir a atingir cerca
de 2,20\,m de comprimento, e ocorre nas zonas baixas e alagadas do
território brasileiro, desde o litoral sul e do leste até à região
centro"-oeste. Da família dos Crotalídeos (\textit{Bothrops
jararacussu).} O termo tupi ocorre pela primeira vez num texto português
com Cardim.} sc.~jararaca grande, e são dez palmos; têm grandes presas
na boca, escondidas ao longo do queixo, e quando mordem estendem"-no
como dedo da mão, têm a peçonha nas gengivas, têm os dentes curvos, e
nas costas deles um rego por onde lhes corre a peçonha. Outros dizem
que a têm dentro do dente que é furado por dentro. Têm tão veemente
peçonha, que em 24 horas, e menos, mata uma pessoa; a peçonha é muito
amarela como água de açafrão, parem muitos filhos, e algumas se acham
treze na barriga.

 Há outra que se chama \textit{Jararagoaipigtanga},\footnote{ \textit{Jaragoaipigtanga} 
é a mesma \textit{Lachesis lenceolatus}, vulgarmente conhecida por ``jararaca"-do"-rabo"-branco'', enquanto é nova.
O nome tupi explica"-se por \textit{jararaca =} ``a cobra'' + \textit{iguai} = 
``cauda'', ``rabo'' + \textit{pitinga} = ``branco'' e apenas é
mencionado neste texto cardiniano.} sc.~que tem a ponta do rabo mais
branco que pardo; estas são tão peçonhentas como as víboras de Espanha,
e têm a mesma cor e feição.

 Há outra \textit{Jararacopeba},\footnote{ \textit{Jararacopeba}: pela
descrição de Cardim deve ser a \textit{Lachesis atrox}, Linn. O termo
tupi vem de \textit{jararaca} = ``cobra'' + \textit{peba} = 
``chato'', ``achatado'' e ocorre pela primeira vez num
texto português com Cardim.} é peçonhentíssima, tem uma côdea pelo
lombo vermelho, e os peitos e o mais corpo é todo pardo.

 Há outras Jararacas mais pequenas, que a maior será de dois palmos; são
de cor de terra, têm umas veias pela cabeça como as víboras, e também
cacarejam como elas.

\especie{Surucucu}\footnote{ \textit{Surucucu}: cobra da mesma
família das anteriores dos Crotalídeos (\textit{Lachesis mutus}, L.).
É a mais temível serpente venenosa brasileira e a maior das víboras,
chegando a atingir 3,60\,m de comprimento. Existem várias espécies, como
a \textit{Surucucu"-de"-fogo, Surucucu"-pico"-de"-jaca, Surucutinga e
Sururucutinga.} O termo indígena vem de \textit{suú"-u"-u} = ``o que dá
dentadas'', ``o que dá muitos botes'', ou seja, ``cobra venenosa que ataca
com repetidos botes''. Este nome tupi ocorre pela primeira vez num texto
português em 1576, no \textit{Tratado da Província do Brasil}, de Pêro de
Magalhães de Gândavo. Tal como aconteceu com o termo ``jararaca'', este
nome tupi \textit{surucucu} ficou por extensão na terminologia
brasileira para designar uma pessoa de mau gênio, irascível.} 
Esta cobra é espantosa, e medonha; acham"-se de quinze
palmos; quando os índios naturais as matam, logo lhes enterram a cabeça
por ter muita peçonha; para tomar caça, e gente, mede"-se com uma
árvore, e em vendo a presa se deixa cair ela e assim a mata. 

\especie{Boicininga}\footnote{ \textit{Boicininga}: usualmente
denominada de \textit{cascavel}, como o faz o próprio Cardim no texto,
é uma cobra da mesma família das anteriores (\textit{Crotalus
terrificus}, Laur.) que se alimenta sobretudo de roedores. Ocorre nas
regiões secas do território brasileiro e é reconhecida pelo guizo, ou
chocalho, que apresenta na extremidade da cauda, e que faz vibrar como
sinal de advertência para os seus predadores. O termo tupi é formado
de \textit{boi} = ``cobra'' + \textit{cininga} = ``tintinante'',
``ressonante'', ``chocalhante'' e ocorre pela primeira vez num texto
português em 1560, numa carta do Pe. José de
Anchieta.} Esta cobra se chama cascavel; é de grande peçonha,
porém faz tanto ruído com um cascavel que tem na cauda, que a poucos
toma: ainda que é tão ligeira que lhe chamam a cobra que voa: seu
comprimento é de doze e treze palmos.

 Há outra chamada \textit{Boiciningbeba};\footnote{ \textit{Boiciningbeba} 
é uma das espécies de cascavel, ainda que seja
difícil encontrar uma explicação para o termo tupi. Possivelmente será
``cobra cascavel achatada'' já que \textit{beba} ou \textit{peba} =
``chato'' ou ``achatado''. Este ocorre pela primeira vez num texto
português com Cardim.} esta também tem cascavel, mas mais pequeno, é
preta, e tem muita peçonha.

\especie{Igbigracuâ}\footnote{ \textit{Igbigracuâ}: de difícil
identificação já que não encontramos referência a um ofídio com este
nome tupi. Outro autor contemporâneo Gabriel Soares de Sousa descreve
esta cobra com o nome de \textit{ubiracoá}.} É tão veemente
a peçonha desta cobra que em mordendo a uma pessoa, logo lhe faz deitar
o sangue por todos os meatos que tem, sc.~olhos, narizes, boca,
orelhas, e por quantas feridas tem em seu corpo, e corre"-lhe por muito
espaço de tempo, e se lhe não acodem todo se vai em sangue, e morre.

\especie{Igbigboboca}\footnote{ \textit{Igbigboboca, ibiboboca} ou
\textit{cobra"-coral}, da família dos Colubrídeos (\textit{Elaps
marcgravi}, Wied.). Normalmente designada apenas por ``cobra"-coral'' é
identificada como sendo da família dos Elapídeos (\textit{Micrurus
corallinus}), com cerca de treze espécies distribuídas pelo Brasil, são
muito venenosas. A designação tupi caiu em desuso passando a ser
utilizado o termo mais vulgar de ``cobra"-coral'' ou \textit{bacorá.} O
nome tupi \textit{ibiboboca} ocorre pela primeira vez num texto
português em 1560, numa carta do Pe. José de
Anchieta.} Esta cobra é muito formosa, a cabeça tem vermelha,
branca e preta, e assim todo o corpo manchado destas três cores. Esta é
mais peçonhenta de todas, anda devagar, e vive em gretas da terra, e
por outro nome se chama a cobra dos corais. Não se pode explicar a
grande veemência que têm estas cobras peçonhentas sobreditas, nem as
grandes dores que causam, nem as muitas pessoas que cada dia morrem
delas, e são tantas em número, que não somente os campos, e matos, mas
até as casas andam cheias delas, acham"-se nas camas, dentro das botas,
quando as querem calçar. Indo os Irmãos para o repouso as acham nele,
enrodilhadas nos pés dos bancos, e se lhe não acodem, quando mordem,
sarjando"-lhe a ferida, sangrando"-se, bebendo unicórnio, ou carimá,\footnote{ A 
\textit{carimá} ou \textit{carimã} é a farinha de mandioca
seca e fina, que tem funções de contraveneno e medicinais. O nome tupi
ocorre pela primeira vez num texto português, em 1554, numa carta
de Luís da Grã e às suas qualidades se refere Gabriel Soares de Sousa,
em 1587, na sua \textit{Notícia do Brasil}, ``[\ldots{}] Muito é para notar
que de uma mesma cousa saia peçonha e a mesma raiz seca é
contrapeçonha, a qual se chama carimá\ldots{}''.} ou água do pau"-de"-cobra,
ou qualquer outro remédio, eficaz, em 24 horas, e menos, morre uma
pessoa com grandes gritos, e dores, e são espantosas, que como uma
pessoa é mordida logo pede a confissão, e faz conta que morre, e assim
dispõe de suas cousas.

 Há outras cobras principalmente as Jararacas que cheiram muito a
almíscar, e onde quer que estão dão sinal de si pelo bom e suave cheiro.

 Há muitos \textit{Alacrás}\footnote{ Os \textit{Alacrás},
\textit{lacraus} ou \textit{lacraias} aparecem indevidamente incluídos por Cardim no grupo
das ``cobras que têm peçonha''. Este fato deve"-se, possivelmente, pelo
mal"-estar que causam as suas picadas. São Escorpionídeos do gênero
\textit{Tytius.}} que se acham nas camas cada dia, e entre os livros
nos cubículos; de ordinário não matam, mas dentro de 24 horas não há
viver com dores.

 Parece que este clima influi peçonha, assim pelas infinitas cobras que
há, como pelos muitos Alacrás, aranhas, e outros animais imundos, e as
lagartixas são tantas que cobrem as paredes das casas, e agulheiros
delas.

\paragraph{Das aves que há na terra e dela se sustentam}

Assim como este clima influi peçonha, assim parece influir
formosuras nos pássaros, e assim como toda a terra é cheia de bosques,
e arvoredos, assim o é de formosíssimos pássaros de todo gênero de
cores.

\especie{Papagaios}\footnote{ Os papagaios são diversas espécies de
aves psitaciformes, pertencentes à família dos Psitacídeos. Existem
várias espécies no território brasileiro, como o
\textit{papagaio"-campeiro, papagaio"-de"-cor"-roxa,
papagaio"-de"-peito"-roxo, papagaio"-galego, papagaio"-moleiro}, entre
muitos outros.} Os papagaios nesta terra são infinitos, mais
que gralhas, zorzais, estorninhos, nem pardais de Espanha, e assim
fazem gralhada como os sobreditos pássaros; destroem as milharadas;
sempre andam em bandos, e são tantos que há ilhas onde são lá mais que
papagaios; comem"-se e é boa carne, são de ordinário muito formosos e de
muito várias cores, e várias espécies, e quase todos falam, se os
ensinam.

\especie{Arara}\footnote{ \textit{Arara} é o nome comum aos
Psitacídeos maiores que os papagaios, dos gêneros
\textit{Anodorhynchus, Ara e Cyanopsitta.} Vivem em bandos e
alimentam"-se de frutos e sementes. No que concerne ao termo indígena se
for de origem tupi pode ser \textit{ará} por \textit{guirá} = ``pássaro'', 
exprimindo o significativo \textit{ará"-ra} = ``pássaro
grande'', mas se for na língua aimará, \textit{arara} = ``palrador'',
``falador'', o que também se adequa ao pássaro em questão. O termo tupi
\textit{arara} ocorre pela primeira vez num texto português em 1576,
com Pêro de Magalhães de Gândavo na sua \textit{História da Província 
Santa Cruz.} } Estes papagaios são os que por outro nome se
chamam \textit{Macaos}:\footnote{ O nome \textit{Macao} designa o
\textit{Ara macao}, L., também denominado de \textit{arara"-canga,
arara"-piranga e arara"-vermelha.}} é pássaro grande, e são raros, e pela
fralda do mar não se acham; é uma formosa ave em cores, os peitos tem
vermelhos como grã; do meio para o rabo alguns são amarelos, outros
verdes, outros azuis, e por todo o corpo têm algumas penas espargidas,
verdes, amarelas, azuis, e de ordinário cada pena tem três, quatro
cores, e o rabo é muito comprido. Estes não põem mais de dois ovos,
criam nas tocas das árvores, e em rochas de pedras. Os índios os
estimam muito, e de suas penas fazem suas galanterias, e empenaduras
para suas espadas; é pássaro bem estreado, faz"-se muito doméstico, e
manso, e falam muito bem, se os ensinam.

\especie{Anapúru}\footnote{ \textit{Anapúru} é uma espécie de
papagaio da mesma família dos Psitacídeos, mas não conseguimos
encontrar uma melhor identificação, ainda que em outros autores apareça
referido como sendo valioso, chegando a valer, entre os índios, dois a
três escravos. O termo tupi ocorre pela primeira vez num texto
português em 1576, na \textit{História da Província Santa Cruz}, de
Pêro de Magalhães de Gândavo.} Este papagaio é formosíssimo e, nele
se acham quase todas as cores em grande perfeição, sc.~vermelho,
verde, amarelo, preto, azul, pardo, cor de rosmaninho, e de todas estas
cores têm o corpo salpicado, e espargido. Estes também falam, e têm uma
vantagem que é criar em casa, e tirar seus filhos, pelo que são de grande estima. 

\especie{Ajurucurao}\footnote{ \textit{Ajurucurao, ajuru"-carau} é
também da mesma família dos anteriores (\textit{Amazona amazonica}, L.). 
O termo tupi deriva de \textit{ajuru =} que é o nome genérico
tupi dos papagaios + \textit{curaú =} ``falador'', ``maldizente'', ``que
solta a língua'' ou ``resmungador''. Este ocorre pela primeira vez num
texto português com Cardim.} Estes papagaios são formosíssimos: 
são todos verdes, têm um barrete, e coleira amarela
muito formosa, e em cima do bico umas poucas de penas de azul muito
claro, que lhe dão muito lustre, e graça; têm os encontros das asas
vermelhos, e as penas do rabo de vermelho, e amarelo salpicadas de azul.

\especie{Tuin}\footnote{ \textit{Tuin, tuim, tuî} é o nome genérico
dos Psitacídeos pequenos (\textit{Forpus xanthopterygius).} Nos
dicionários de tupi"-português aparece traduzido como Periquito. O termo
tupi \textit{tuim} ocorre pela primeira vez num texto português em
1511, no \textit{Livro da Nau Bretoa.}} O tuin é uma espécie
de papagaio pequeno do tamanho de um pardal; são verdes espargidos de
outras várias cores, são muito estimados, assim pela sua formosura,
como também porque falam muito, e bem, e são muito domésticos, e tão
mansinhos que andam correndo por toda uma pessoa, saltando"-lhe nas
mãos, nos peitos, nos ombros, e cabeça, e com o bico lhe esgravatam os
dentes, e estão tirando o comer da boca à pessoa que os cria, e fazem
muitos momos, e sempre falam, ou cantam a seu modo.

\especie{Guigrajuba}\footnote{ \textit{Guigrajuba, guirajuba,
guarúba} ou \textit{guarajuba} é um papagaio da mesma
família dos Psitacídeos (\textit{Conurus guarouba}, Gm.). O termo em
tupi vem de \textit{guirá} = ``pássaro'' + \textit{juba} = ``amarelo'', o
que confirma a significação apresentada por Cardim no texto. Forma
depois as várias variantes por aglutinação. Este nome tupi ocorre a
primeira vez num texto português com Cardim.} Chama"-se este
pássaro \textit{Guigrajuba}, sc.~pássaro amarelo; não falam, nem
brincam, antes são muito melancolizados, e tristes, mas muito
estimados, por se trazerem de duzentas, e trezentas léguas, e não se
acham, senão em casas de grandes principais, e têm"-nos em tanta estima
que dão resgate, e valia de duas pessoas por um deles, e tanto o
estimam como os Japões as trempes, e panelas, e quaisquer outros
senhores alguma cousa de grande preço, como falcão, girifalte etc.

\especie{Iapu}\footnote{ \textit{Iapu, japu} é o nome comum a
diversas aves da família dos Icterídeos (\textit{Ostinops decumanus}, 
Pall.). O termo tupi explica"-se por \textit{ya =} ``o que'', ``aquele
que'' + \textit{pu} = ``soar'', ``fazer rumor'' ou seja ``o que soa'' ou
``rumoreja'', ``o que faz barulho''. Este ocorre pela primeira vez num
texto português com Cardim.} Este pássaro é do tamanho de uma
pega, o corpo tem de um preto fino, e o rabo todo amarelo gracioso; na
cabeça tem três penachozinhos, que não parecem senão cornitos quando os
levanta; os olhos têm azuis, o bico muito amarelo, é pássaro formoso, e
tem um cheiro muito forte quando se agasta; são muito solicitados em
busca de comer, não lhes escapa aranha, barata, grilo, e são grande
limpeza de uma casa, e andam por elas como pegas, não lhes fica cousa
que não corram; é perigo grande terem"-no na mão, porque arremetem aos
olhos e tiram"-nos.

\especie{Guainumbig}\footnote{ \textit{Guainumbig, Guainumbi,
Gainambi} é a designação atribuída às aves da família dos
Trochilídeos, usualmente denominados de ``beija"-flor''. O termo aparece
pela primeira vez em português numa carta, do Pe. José de
Anchieta, em 1560. As espécies referidas por Cardim estão corretas
etimologicamente, já que o nome \textit{guaracigá} ou \textit{guaraciá}
significa ``fruto do sol'', por \textit{coaracy} = ``sol'' e
\textit{á} = ``fruto''; \textit{guaracigóba} ou \textit{guaracióba}
indica ``cobertura do sol'', de \textit{óba} = ``folha'', mas
implica o sentido de ``cobrir'', ``o que cobre'', ``cobertura'' e
\textit{guaracigaba}, ou \textit{guaraciaba} significa por sua vez
``cabelo do sol'', de \textit{aba} = ``cabelo''. Esta variedade de
expressões, morfológicas e semânticas, porque foram designadas algumas
das espécies desta ave, deve"-se, sem dúvida, à admiração que causaram
aos indígenas e aos primeiros colonizadores europeus as cores
brilhantes, resplandecentes e belas dos beija"-flores. Algumas destas
designações foram utilizadas pela primeira vez num texto português com
Cardim.} Destes passarinhos há várias espécies, sc.,
\textit{Guaracigá}, sc.~fruta do sol, por outro nome
\textit{Guaracigoba}, sc.~cobertura do sol, ou \textit{Guaracigaba}, 
sc.~cabelo do sol; nas Antilhas lhe chamam ``o pássaro ressuscitado'', e
dizem que seis meses dorme e seis meses vive; é o mais fino pássaro que
se pode imaginar, tem um barrete sobre sua cabeça, a qual se não pode
dar cor própria, porque de qualquer parte que a tomam mostra vermelho,
verde, preto, e mais cores todas muito finas, e resplandecentes, e o
papo é tão formoso que de qualquer parte que o tomam, mostra todas as
cores principalmente um amarelo mais fino que ouro.

 O corpo é pardo, tem o bico muito comprido, e a língua de dois
comprimentos do bico; são muito ligeiros no voar, e quando voam fazem
estrondo como abelhas, e mais parecem abelhas na ligeireza que
pássaros, porque sempre comem de voo sem pousar na árvore; assim como
abelhas andam chupando o mel das flores; têm dois princípios de sua
geração; uns se geram de ovos como outros pássaros, outros de
borboletas, e é cousa para ver, uma borboleta começar"-se a converter
neste passarinho, porque juntamente é borboleta e pássaro, e assim se
vai convertendo até ficar neste formosíssimo passarinho; cousa
maravilhosa, e ignota aos filósofos pois um vivente sem corrupção se
converte noutro.

\especie{Guigranheengetá}\footnote{ \textit{Guigranheegetá,
guirá"-nheengetá}, da família dos Tiranídeos (\textit{Taenioptera
mengeta}, L.). O termo tupi vem de \textit{guirá} = ``pássaro'' +
\textit{nheeng} = ``falar'' + \textit{etá} = ``muito'', ou seja, ``pássaro
que fala ou canta muito''. O nome vem desaparecendo dando origem a
\textit{gronhatá} ou \textit{grunhatá.} A nível vulgar é conhecido por
``pombinha"-das"-almas'' e ``maria"-branca''.} Este pássaro é
do tamanho de um pintassilgo, tem as costas, e asas azuis, e o peito, e
barriga de um amarelo finíssimo. Na testa tem um diadema amarelo que o
faz muito formoso; é pássaro excelente para gaiola, por falar de muitas
maneiras, arremedando muitos pássaros, e fazendo muitos trocados e
mudando a fala em mil maneiras, e atura muito no canto, e são de
estima, e destes de gaiola há muitos e formosos, e de várias cores. 

\especie{Tangará}\footnote{ \textit{Tangará} é o nome comum a
diversas espécies de aves da família dos Piprídeos, sobretudo a
\textit{Chirosophia caudata}, Sw. vulgarmente chamada ``dançador'', o
que confirma com a descrição deste texto cardiniano, que foi onde o
termo ocorreu pela primeira vez.} Este é do tamanho de um
pardal: todo preto, a cabeça tem de um amarelo laranjado muito fino;
não canta, mas tem uma cousa maravilhosa que tem acidentes como de gota
coral, e por esta razão o não comem os índios por não terem a doença;
tem um gênero de baile gracioso, sc.~um deles se faz morto, e os
outros o cercam ao redor, saltando, e fazendo um cantar de gritos
estranhos que se ouve muito longe, e como acabam esta festa, grita, e
então todos se vão, e acabam sua festa, e nela estão tão embebidos
quanto a fazem que ainda que sejam vistos, e os espreitem não fogem;
destes há muitas espécies, e todos têm acidentes. 

\especie{Quereiuá}\footnote{ \textit{Quereiuá} \textit{quiruá} ou
\textit{querejuá}, da família dos Cotingídeos (\textit{Cotinga
cincta}, Kuhl.). O nome tupi é difícil de explicar tendo ocorrido pela
primeira vez neste texto de Cardim.} Este
pássaro é dos mais estimados da terra, não pelo canto, mas pela
formosura da pena; são de azul claro em parte, e escuro, e todo o peito
roxo finíssimo, e as asas quase pretas, são tão estimadas, que os
índios os esfolam, e dão duas e três pessoas por uma delas, e com as
penas fazem esmaltes, diademas, e outras galanterias.

\especie{Tucana}\footnote{ \textit{Tucana}: vulgarmente denominado de
\textit{tucano}, para designar diversas aves da família dos
Ranfastídeos, como o \textit{Ramphastos vitellinus}, tucano"-de"-bico"-preto, 
\textit{R. dicolorus}, tucano"-de"-bico"-verde,
\textit{R. tucanus}, o tucano"-de"-bico"-vermelho e o \textit{R. toco}, 
que é o tucano"-boi ou tucanuçu. O termo tupi parece ser formado por
\textit{ti} = ``bico'' + \textit{cang} = ``osso'', ou seja, ``bico em osso''
e ocorre pela primeira vez num texto português com Cardim. Mas, parece
ter sido André Thevet quem descreveu pela primeira vez esta ave, à qual
dedica um capítulo, dando"-lhe o nome indígena \textit{toucan:} ``[\ldots{}] 
Na costa o principal produto comerciado é a plumagem dum pássaro que
eles chamam na sua língua \textit{toucan}\ldots{}''. Cf. Frank Lestringant,
\textit{Le Brésil d'André Thevet, Les Singularités de la France
Antarctique (1557}), Paris, Ed. Chandeigne, 1997, cap. \textsc{xlvii}, 
pp. 185--188.} Este pássaro é do tamanho de uma pega; é todo
preto, tirando o peito, o qual é todo amarelo com um círculo vermelho;
o bico é de um grande palmo, muito grosso e amarelo, e por dentro muito
vermelho, tão brunido e lustroso, que parece envernizado; fazem"-se
domésticos, e criam"-se em casa, são bons para comer, e a pena se estima
muito por ser fina.

\especie{Guigraponga}\footnote{ \textit{Guigraponga} ou
\textit{araponga}, da família dos Cotingídeos (\textit{Chasmorhynchus
nudicollis}, Vieill.). É um pássaro frugívoro das florestas da América
do Sul, e o seu canto é de tom metálico, semelhante ao bater de um
martelo sobre a bigorna, o que faz com que vulgarmente seja denominado
de ``ferreiro'' ou ``ferrador''. Ocorre pela primeira vez neste texto de
Cardim, com a primeira designação. O termo tupi vem de \textit{guirá} = 
``pássaro'' + \textit{ponga} = ``sonante'', ``que soa''.} Este
pássaro é branco, e sendo não muito grande, dão tais brados que não
parece senão um sino, e ouve"-se meia légua, e seu cantar é ao modo de repique de sino.

\especie{Macucaguá}\footnote{ \textit{Macucaguá}, \textit{macaguá,
macucaua ou macacáua}, ave da família dos Falconídeos
(\textit{Herpetotheres cachinnans}, L.). Ocorre pela primeira em textos
portugueses em 1576, com Pêro de Magalhães de Gândavo, na \textit{História da
Província do Brasil.} O termo tupi é formado por \textit{má} por
\textit{ybá} = ``fruto'' + \textit{cugiguar}, por \textit{currinhar} = 
``que come'', ``comedor'', ou seja, ``comedor de frutos''.} Esta
ave é maior que nenhuma galinha de Portugal; parece"-se com faisão, e
assim lhe chamam os portugueses, tem três titelas uma sobre a outra, e
muita carne, e gostosa, põe duas vezes no ano, e de cada vez treze ou
quinze ovos; andam sempre pelo chão, mas quando vem gente se sobem nas
árvores, e à noite quando se empoleiram como fazem as galinhas. Quando
se põem nas árvores, não põem os pés nos paus, mas as canelas das
pernas, e mais da parte dianteira. Destas há muitas espécies, e
multidão, e facilmente se flecham.

 Entre elas há uma das mais pequenas, tem muitas habilidades: adivinha
quando canta a chuva, dá tão grandes brados que se não pode crer pássaro
tão pequeno, e a razão é porque tem a goela muito grande, começa na
cabeça, e sai pelo peito ao longo da carne, e chega ao sesso, e faz
volta, e torna"-se a meter no papo, e então procede como aos outros
pássaros, e fica como trombeta com suas voltas. Correm após qualquer
pessoa, às picadas brincando como cachorrinhos, se lhe deitam ovos de
galinha choca"-os, e cria os pintainhos, e se vê as galinhas com
pintainhos tanto as persegue até que lhos toma e os cria. 

\especie{Mutu}\footnote{ \textit{Mutu} ou \textit{mutum} é o nome
genérico para as aves galiformes da família dos Cracídeos, que ocorrem
por quase toda a América do Sul. Existem várias espécies como
\textit{Carx globulosa}, que é o mutum"-fava ou mutum"-açu, \textit{C.
blumenbachii}, o mutum"-do"-sudeste, o \textit{C. fasciolata pinima}, que
é o mutum"-de"-penacho, entre outras espécies. Ocorre pela primeira vez
num texto português com Cardim.} Esta galinha é muito
caseira, tem uma crista de galo espargida de branco e preto, os ovos
são grandes como de pata, muito alvos, tão rijos que batendo um no
outro, tinem como ferro, e deles fazem os seus maracas, sc.~cascavéis;
todo cão que lhe come os ossos, e aos homens nenhum prejuízo lhes faz. 

\especie{Uru}\footnote{ \textit{Uru} é o nome comum a duas espécies
de aves da família dos Fasianídeos: o \textit{Odotophorus guyanensis}, 
Gm. e o \textit{O. capueira}, Spix. O primeiro existente na região
amazônica e o segundo no litoral. Cardim deve ter descrito o segundo, e
foi exatamente neste texto que ocorreu pela primeira vez em
português.} Nesta terra há muitas espécies de perdizes que
ainda que se não pareçam em todo com as de Espanha, todavia são muito
semelhantes na cor, e no gosto, e na abundância. 

 Há nesta terra muitas espécies de rolas, tordos, melros, e pombas de
muitas castas, e todas estas aves se parecem muito com as de Portugal;
e as pombas e rolas são em tanta multidão que em certos campos muito
dentro do sertão são tantas que quando se levantam impedem a claridade
do sol, e fazem estrondo, como de um trovão; põem tantos ovos, e tão
alvos, que de longe se veem os campos alvejar com os ovos como se fosse
neve, e com servirem de mantimento aos índios não se podem desençar,
antes dali em certos tempos parece que correm todas as partes desta província. 

\especie{Nhandugoaçu}\footnote{ \textit{Nhandugoaçu}: tal como Cardim
refere trata"-se da \textit{Ema} da família dos Reídeos (\textit{Rhea
americana}, L.), erradamente chamada de avestruz e que é a maior das aves
brasileiras, chegando o macho, o \textit{congo}, a atingir 34 quilos de
peso. Ocorre pela primeira vez em português neste texto de Cardim. O
termo tupi explica"-se por \textit{nhan} = ``corre'' + \textit{tu (} ou
\textit{ub} = perna) = ``estrepitante'' + \textit{guaçu} = ``grande''.} 
Nesta terra há muitas emas, mas não andam senão pelo sertão dentro. 

\especie{Anhigma}\footnote{ \textit{Anhigma, anhuma} ou \textit{inhuma}
é uma ave da família dos Anhimídeos (\textit{Palamedea cornuta}, L.), 
de grande porte, patas longas que ocorre sobretudo nas regiões
alagadiças da Amazônia, Nordeste, vale do rio São Francisco,
Centro"-Oeste, São Paulo e Paraná. Ocorre pela primeira vez em 1560,
numa carta do Pe. José de Anchieta.} Este pássaro
é de rapina, grande, e dá brados que se ouvem meia légua, ou mais; é
todo preto, os olhos tem formosos, e o bico maior que de galo, sobre
este \mbox{bico tem} um cornito de comprimento de um palmo; dizem os naturais
que este corno é grande medicina para os que se lhe tolhem a fala como
já aconteceu que, pondo ao pescoço de um menino que não falava, falou logo.

 Há outras muitas aves de rapina, sc.~águias, falcões, 
açores,\footnote{ O autor refere"-se ao \textit{açor} que é uma ave de rapina,
diurna, da família dos Falconídeos, semelhante ao gavião, mas de maior
envergadura.} esmerilhões,\footnote{ \textit{Esmerilhão} é uma ave
de rapina, diurna, da família dos Falconídeos.} 
francelhos,\footnote{ Trata"-se do nome vulgar por que também são designados, especialmente, o
mioto e o peneireiro, ambas aves de rapina.} e outras muitas, mas são
todas de ordinário tão bravas que não servem para caçar, nem acodem à mão.

\paragraph{Das árvores de fruto}

\especie{Acaju}\footnote{ \textit{Acaju} ou \textit{caju} é
o fruto e árvore da família das Anacardiáceas (\textit{Anacardium
occidentale}, L.). Hoje o nome \textit{caju} reserva"-se para a
\textit{Cedrela guyanensis}, J. da família das Meliáceas, que vegeta
na Amazônia. É originário da América do litoral atlântico tropical,
incluindo as Antilhas. O termo tupi vem de \textit{acã} = ``caroço'' +
\textit{yu}, por \textit{y"-ub =} ``que dá'', ``que tem'', logo ``que tem
caroço''. Os portugueses encontraram"-no no Brasil onde se integrava no
conjunto das árvores mais apreciadas e utilizadas pelos ameríndios, de
onde obtinham tudo, da raiz aos frutos, alimentos, madeira, lenha,
gomas, repelentes, conservantes para embarcações e redes, além de
remédios e uma bebida fermentada. São expressivas, amplas e minuciosas
as descrições desta árvore e do seu fruto em textos dos séculos \textsc{xvi} e
\textsc{xvii}, ocorrendo pela primeira vez em Pêro de Magalhães de Gândavo, em 1576,
no \textit{Tratado da Província do Brasil.}} Estas árvores
são muito grandes, e formosas, perdem a folha em seus tempos, e a flor
se dá nos cachos que fazem umas pontas como dedos, e nas ditas pontas
nasce uma flor vermelha de bom cheiro, e após ela nasce uma castanha, e
da castanha nasce um pomo do tamanho de um repinaldo, ou maçã camoeza;
é fruta muito formosa, e são alguns amarelos, e outros vermelhos, e
tudo é sumo: são bons para a calma, refrescam muito, e o sumo põe nódoa
em pano branco que se não tira senão quando se acaba. A castanha é tão
boa, e melhor que as de Portugal; comem"-se assadas, e cruas deitadas em
água como amêndoas piladas, e delas fazem maçapães, e bocados doces
como amêndoas. A madeira desta árvore serve pouco ainda para o fogo,
deita de si goma boa para pintar, e escrever em muita abundância. Com a
casca tingem o fiado, e as cuias\footnote{ Cardim refere"-se às
\textit{cuias} que eram uma espécie de vasilha, de forma
semielipsoidal ou semiesférica, feita com a casca da cuieira, planta
da família das Bignoniáceas, ou seja, uma cabaça. Este termo tupi
ocorre pela primeira vez em textos portugueses com Cardim.} que lhe
servem de panelas. Esta pirada e cozida com algum cobre até se gastar a
terça de água, é único remédio para chagas velhas e saram depressa.
Destas árvores há tantas como os castanheiros em Portugal, e dão"-se por
esses matos, e se colhem muitos molhos de castanhas, e a fruta em seus
tempos a todos farta. Destes acajus fazem os índios vinho.\footnote{ O
Pe. Fernão Cardim menciona o \textit{cauim} que era uma bebida
fermentada com o fruto do caju e que é descrita noutro texto deste
autor de forma pormenorizada.} 

\especie{Mangaba}\footnote{ \textit{Mangaba} é o fruto e árvore, a
mangabeira da família das Apocináceas (\textit{Hancornia speciosa}, 
Gomez). O fruto tem a aparência de maçã, sendo comestível \textit{in
natura} e apropriado para compotas, doces, gelados e refrigerantes.
Ocorre pela primeira vez em português num texto de 1554, numa
carta do jesuíta Brás Lourenço. O termo tupi \textit{manguaba}
designa ``coisa de comer''.} Destas árvores há grande cópia, 
\textit{maxime} na Bahia, porque nas outras partes são raras; na feição se
parece com macieira de anafega, e na folha com a de freixo; são árvores
graciosas, e sempre têm folhas verdes. Dão duas vezes fruto no ano: a
primeira de botão, porque não deitam então flor, mas o mesmo botão é a
fruta; acabada esta camada que dura dois ou três meses, dá outra,
tornando primeiro flor, a qual é toda como de jasmim, e de tão bom
cheiro, mas mais esperto; a fruta é do tamanho de abricós, amarela, e
salpicada de algumas pintas pretas, dentro tem algumas pevides, mas
tudo se come, ou sorve como sorve as de Portugal; são de muito bom
gosto, sadias, e tão leves que por mais que comam, parecem que não
comem fruta; não amadurecem na árvore, mas caem no chão, e daí as
apanham já maduras, ou colhendo"-as verdes as põem em madureiro; delas
fazem os índios vinhos; a árvore é a mesma fruta em verde, toda está
cheia de leite branco, que pega muito nas mãos, e amarga.

\especie{Macuoé}\footnote{ \textit{Macuoé, mucuoé, mucugê}, 
\textit{macugé} ou \textit{mucuruje}, da mesma família das Apocináceas
(\textit{Couma rigida}, Mull. Arg.) O termo tupi é de étimo
desconhecido.} Esta fruta se dá em umas árvores
altas; parece"-se com peras"-de"-mato de Portugal, o pé tem muito
comprido, colhem"-se verdes, e põem"-se a madurar, e maduros são muito
gostosos, e de fácil digestão; quando se hão"-de colher sempre se corta
toda a árvore e por serem muito altas, e se não fora esta destruição
houvera mais abundância, e por isso são raras; o tronco tem grande
cópia de leite branco, e coalha"-se; pode servir de lacre se quiserem
usar dele.

\especie{Araçá}\footnote{ \textit{Araçá} é o nome atribuído às
Mirtáceas do gênero \textit{Psidium cattleyanum}, fruto do
araçazeiro, das quais há várias espécies e que os escritores
quinhentistas e seiscentistas mostram conhecer. Ocorre pela primeira
vez num texto português em 1561, numa carta do Pe. Manuel da
Nóbrega.} Destas árvores há grande cópia, de muitas
castas; o fruto são uns perinhos, amarelos, vermelhos, outros verdes:
são gostosos, desenfastiados, apetitosos, por terem alguma ponta de
agro. Dão fruto quase todo o ano.

\especie{Ombu}\footnote{ \textit{Ombu, umbu, imbu} ou \textit{ambu} é
o fruto do umbuzeiro da família das Anacardiáceas (\textit{Spondias
purpurea}, L.). Os frutos são pequenas drupas oblongas,
amarelo"-esverdeadas, de polpa verde"-clara, doce e aromática,
comestíveis \textit{in natura}, e utilizados em doces e compotas. É
difícil de atribuir uma significação ao termo tupi.} Este
ombu é árvore grande, não muito alta, mas muito espalhada; dá certa
fruta como ameixas alvares, amarela, e redonda, e por esta razão lhe
chamam os portugueses ameixas; faz perder os dentes e os índios que as
comem os perdem facilmente; as raízes desta árvore se comem, e são
gostosas e mais saborosas que a balancia,\footnote{ Considera"-se que
Cardim se refere à \textit{melancia}, atendendo ao contexto em que está
inserido este nome, assim como pela descrição cardiniana.} porque
são mais doces, e a doçura parece de açúcar. São frios, sadios, e
dão"-se aos doentes de febres; e aos que vão para o sertão serve de água
quando não têm outra. 

\especie{Jaçapucaya}\footnote{ \textit{Jaçapucaya} ou \textit{sapucaia} 
é o nome comum às diversas espécies de Lecitidáceas, do gênero
\textit{Lecythis}, cujos frutos, lenhosos, em geral cilíndricos, quando
abertos, apresentam a forma de uma cuia. Ocorre pela primeira vez num
texto português de 1574 num \textit{Livro de Contas}, e dois anos
depois é referido por Gândavo na \textit{História da Província do
Brasil.} O nome tupi forma"-se de \textit{ya} = ``fruto de árvore'' +
\textit{eçá pucá í} = ``que tem saltamento do olho'', segundo Baptista
Caetano.} Esta árvore é das grandes e formosas desta terra;
cria uma fruta como panela, do tamanho de uma grande bola de grossura
de dois dedos, com uma cobertura por cima, e dentro está cheia de umas
castanhas como mirabólanos, e assim parece que são os mesmos da
Índia.\footnote{ Cardim compara as castanhas da \textit{sapucaia} aos
mirabólanos do Índico, que procedem da \textit{Terminalia chebula}, 
Retz, da família das Combretáceas, que existem no Oriente, nas regiões
de Bengala, Cambaia, Guzerate, Malaca e Bornéu e que são utilizados na
alimentação, farmacopeia e tinturaria. Garcia da Orta descreve"-os nos
\textit{Colóquios dos simples e das drogas da Índia}, ed. fac"-simile,
de 1861, com notas do Conde de Ficalho, Lisboa, Imprensa Nacional/Casa da Moeda, 1987, colóquio 37º.} Quando estão já de vez se abre
aquela sapadoura, e cai a fruta; se comem muita dela verde, pela uma
pessoa quantos cabelos tem em seu corpo; assadas é boa fruta. Das
panelas usam para grais\footnote{ Cardim refere"-se aos almofarizes.} e
são de dura; a madeira da árvore é muito rija, não apodrece, e é de
estima para os eixos dos engenhos.

\especie{Araticu}\footnote{ \textit{Araticu} ou \textit{ariticum} é o
nome comum a diversas Anonáceas dos gêneros \textit{Anona} e 
\textit{Rollinia.} Ocorre pela primeira vez num texto português neste
escrito de Cardim. O termo tupi aparece também como \textit{aratycu}
significando uma fruta conhecida como ``cabeça de negro'', de
consistência mole, grumosa e de sabor adocicado. Existem várias
espécies no território brasileiro, como o Araticum"-alvadio
(\textit{Rollinia exalbida}), o Araticum"-apê (\textit{Anona pisonis}), 
o Araticum"-cagão (\textit{Anona cacans}), Araticum"-da"-areia
(\textit{A. senegalensis}), Araticum"-do"-brejo (\textit{A. glabra}), 
entre outras.} Araticu é uma árvore do tamanho de
laranjeira, e maior; a folha parece de cidreira, ou limoeiro; é árvore
fresca e graciosa, dá uma fruta da feição e tamanho de pinhas, e cheira
bem, tem arrazoado, e é fruta desenfastiada.

 Destas árvores há muitas castas, e uma delas chamada 
araticu"-paná;\footnote{ \textit{Araticum"-paná} é uma das plantas da família das
Anonáceas e do seu fruto. Atendendo à descrição de Cardim deve
tratar"-se da espécie designada de Araticum"-cortiça (\textit{Anona
crassiflora}) que é uma árvore que chega a atingir 10\,m de altura e de
flores amarelas, cuja casca serve de cortiça, mas que não é totalmente
impermeável. Este termo tupi ocorre pela primeira vez num texto
português com Cardim.} se comem muito da fruta fica em fina peçonha, e
faz muito mal. Das raízes destas árvores fazem boias para redes, e são
tão leves como cortiças.

\especie{Pequeá}\footnote{ \textit{Pequeá, pequiá ou piquiá}: da
família das Cariocariáceas, das quais existem várias espécies como a 
\textit{Caryocar barbinerve}, a \textit{C. coriaceum.}, a \textit{C.
butisorum} e a \textit{C. crenatum.} Todas fornecem madeira resistente
e as sementes óleo, como o ``sebo de pequiá'' ou ``manteiga de pequiá'', ou
o látex conhecido como chicle ou chiclete. Ocorre pela primeira vez num
texto português com Cardim. O termo tupi pode derivar"-se de \textit{pé} = 
``casca'' + \textit{quiá} = ``suja'', ``manchada''.} Destas
árvores há duas castas; uma delas dá uma fruta do tamanho de uma boa
laranja, e assim tem a casca grossa como laranja; dentro desta casca
não há mais que mel tão claro, e doce como açúcar em quantidade de um
ovo, e misturado com ele tem as pevides.

 Há outra árvore Pequeá: é madeira das mais pesadas desta terra; em
Portugal se chama cetim;\footnote{ Pau"-cetim. [\versal{N.}~do \versal{E.}]} tem ondas muito galantes, dura muito, e não
apodrece.

\especie{Jaboticaba}\footnote{ \textit{Jaboticaba} ou
\textit{jabuticaba}: árvore, a jabuticabeira pertence à família das
Mirtáceas (\textit{Myrciaria cauliflora}, Berg.). O termo tupi é
\textit{yauti"-guaba} = ``comida de cágado''. Ocorre pela primeira vez num
texto português em Cardim.} Nesta árvore se dá uma fruta do
tamanho de um limão"-de"-ceitil; a casca e gosto parece de uva"-ferral,
desde a raiz da árvore por todo o tronco até o derradeiro raminho; é
fruta rara, e acha"-se somente pelo sertão adentro da capitania de São
Vicente. Desta fruta fazem os índios vinho e o cozem como vinho de uvas.

Neste Brasil há muitos coqueiros,\footnote{ O coqueiro (\textit{Cocos
nucifera}, L.) não é espontâneo do Brasil, tendo sido trazido da Índia
pelos portugueses, levado para África e daí para o Brasil, onde se veio
a dar melhor, alcançando grandes dimensões. Foi uma árvore que atraiu
muito os navegadores, pela sua abundância e pela sua grande utilidade.
A descrição de Duarte Barbosa, que é uma das primeiras a ser feita, em
língua latina, é muito precisa, ``[\ldots{}] esta terra, ou por
melhor dizer toda a do Malabar, é coberta ao longo do mar de palmeiras,
tão altas como altos ciprestes, tem os pés mui limpos e lisos; e em
cima uma copa de ramos, entre os quais nasce uma fruta grande que
chamam cocos; é fruta de que eles muito aproveitam, e de cada ano
carregam muitas naus''. Cf. Duarte Barbosa, \textit{Livro do
que viu e ouviu no Oriente Duarte Barbosa}, in \textit{Colecção de
Notícias para a História e Geografia das Nações Ultramarinas, que vivem
nos domínios portugueses publicada pela Academia Real das Ciências}, 
Lisboa, Tipografia da Academia, 1867, p. 343.} que dão cocos
excelentes como os da Índia; estes de ordinário se plantam, e não se
dão pelos matos, senão nas hortas, e quintais; e há mais de vinte
espécies de palmeira e quase todas dão fruto, mas não tão bom como os
cocos; com algumas destas palmeiras cobrem as casas.

 Além destas árvores de fruto há muitas outras que dão vários frutos, de
que se aproveitaram e sustentaram muitas nações de índios, juntamente
com o mel,\footnote{ As sociedades ameríndias semissedentárias
adotaram um padrão de subsistência fundamentalmente assente no
cultivo de raízes, sobretudo da mandioca, na pesca, na caça e na
recoleção. Estas atividades forneciam"-lhe os alimentos essenciais à
sua subsistência a par da obtenção do mel, que consumiam das abelhas
selvagens e domesticadas. Os índios designavam"-nas por \textit{heru} ou
\textit{tapiúja}, ocorrendo estes nomes pela primeira vez num texto em
português, em 1587, na \textit{Notícia do Brasil}, de Gabriel Soares de
Sousa.} de que há muita abundância, e com as caças, porque não têm outros mantimentos.

\especie{Pinheiro}\footnote{ Cardim menciona o \textit{Pinheiro} ou
\textit{pinheiro"-do"-paraná}, que é da família das Coníferas
(\textit{Araucaria angustifolia}, A. Rich. Lamb.) também denominada de
\textit{araucária}.} No sertão da Capitania de São Vicente 
até ao Paraguai há muitos e grandes pinhais propriamente como os de
Portugal, e dão pinhas como pinhões; as pinhas não são tão compridas,
mas mais redondas, e maiores; os pinhões são maiores, e não são tão
quentes, mas de bom temperamento e sadios.

\paragraph{Das árvores que servem para mezinhas}

\especie{Cabureigba}\footnote{ \textit{Cabureigba, caburahida,
cabureiba} ou \textit{cabreúva} é uma árvore da família das
Leguminosas, subfamília das Papilionáceas (\textit{Myrocarpus
fastigiatos}, Fr. All.). Ocorre pela primeira vez num texto português
com Pêro de Magalhães de Gândavo, na \textit{História da Província Santa
Cruz}, em 1576. O nome tupi vem de \textit{caburé} = ``a
coruja'' + \textit{ybá} = ``árvore'', ``pau''. A resina mencionada no texto,
extraída do pericarpo, é conhecida por
\textit{caburé"-icica.}} Esta árvore é muito estimada, e
grande, por causa do bálsamo que tem; para se tirar este bálsamo se
pica a casca da árvore, e lhe põem um pequeno de algodão nos golpes, e
de certos em certos dias vão recolher o óleo que ali se estila;
chamam"-lhe os portugueses bálsamo por se parecer muito com o verdadeiro
das vinhas de Engaddi;\footnote{ Cardim compara esta mezinha extraída
de \textit{cabureigba} com o bálsamo das vinhas de Engaddi, que era uma
das vinhas mais conhecidas entre os Hebreus, como as de Hebron, as
quais gozaram de justa e enorme fama, pela sua frondosidade e
rendimento. Eram ainda famosos os vinhedos das colinas de Samaria e do
Carmelo, além dos do vale do Jordão. Sendo Fernão Cardim um
eclesiástico era conhecedor das Escrituras, daí a referência às vinhas
de Engaddi, mencionadas no Antigo Testamento, que eram como um bálsamo.
Cf. \textit{Cântico dos Cânticos}, cap. 1, vers. 14, ``[\ldots{}] \textit{Como um
racimo de flores de hena nas vinhas de En"-Gedi, é para mim o meu
amado''.}} serve muito para feridas frescas, e tira todo o sinal,
cheira muito bem, e dele, e das cascas do pão se fazem rosários e
outras cousas de cheiro; os matos onde os há cheiram bem, e os animais
se vão roçar nesta árvore, parece que para sararem de algumas
enfermidades. A madeira é das melhores deste Brasil, por ser muito
forte, pesada, eliada e de tal grossura que delas se fazem as 
gangorras,\footnote{ As gangorras eram peças de engenhos de água, feitas de
madeira muito resistente e destinadas a espremer o bagaço da cana
moída. Também eram assim designados os engenhos de madeira, manuais,
constantes apenas de dois rolos entre os dois esteios verticais e os
engenhos de pau usados por pequenos plantadores de cana"-de"-açúcar para
o fabrico de rapaduras, sobretudo na zona canavieira pernambucana.}
eixos, e fusos para os engenhos. Estas são raras, acham"-se
principalmente na Capitania do Espírito Santo.

\especie{Cupaigba}\footnote{ \textit{Cupaigba} ou \textit{copahiba}: 
também da família das Leguminosas, subfamília das Caesalpináceas
(\textit{Copahiba langsdorfii}, Desf.). Ocorre pela primeira vez num
texto português com Gândavo, \textit{História.}, em 1576, com uma
descrição muito semelhante à de Cardim. O termo tupi é de étimo
incerto.} É uma figueira comumente muito alta, direita e
grossa; tem dentro dela muito óleo; para se tirar a cortam pelo meio,
onde tem o vento, e aí tem este óleo em tanta abundância, que algumas
dão um quarto, e mais de óleo; é muito claro, de cor de azeite; para
feridas é muito estimado, e tira todo sinal. Também serve para as
candeias e arde bem; os animais, sentindo sua virtude, se vêm esfregar
nelas; há grande abundância, a madeira não vale nada.

\especie{Ambaigba}\footnote{ \textit{Ambaigba, ambahiba, embaúba} ou
 \textit{imbaúba} é uma árvore da família das Moráceas
(\textit{Cecropia adenops}, Mart.). Ocorre pela primeira vez num texto
português com Cardim. É uma árvore típica das florestas úmidas do
Brasil tropical e equatorial. É muito procurada pela preguiça que se
alimenta das suas folhas e brotos. Os índios utilizam os ramos para
obter fogo e aproveitam o caule na produção de instrumentos musicais. O
termo tupi vem de \textit{ambá} = ``oco'' + \textit{yba} = ``árvore''.} 
Estas figueiras não são muito grandes, nem se
acham nos matos verdadeiros, mas nas copueras,\footnote{ \textit{Capoeiras}, mata secundária. [\versal{N.}~do \versal{E.}]} onde este roça; a casca 
desta figueira, raspando"-lhe da parte de dentro, e espremendo aquelas
raspas na ferida, pondo"-lhas em cima, e atando"-as com a mesma casca,
em breve sara. Delas há muita abundância, e são muito estimadas por sua
grande virtude; as folhas são ásperas, e servem para alisar qualquer
pau; a madeira não serve para nada.

\especie{Ambaigtinga}\footnote{ \textit{Ambaigtinga} ou
\textit{imbaúba"-branca}: também da família das Moráceas
(\textit{Cecropia palmata}, Willd.).} Esta figueira é a
que chamam do inferno: acham"-se em taperas,\footnote{ \textit{Tapera}
era a designação, em tupi, para a aldeia indígena abandonada ou para a
habitação em ruínas. Este ocorre pela primeira vez num texto em
português, em 1562, numa \textit{Carta de Sesmaria}, ``[\ldots{}] 
\textit{partindo pela banda do campo ao longo dos midos (sic) de terra
e pela tapera, que foi do Grilo}\ldots{}'', in Serafim Leite,
\textit{Cartas dos Primeiros Jesuítas do Brasil (1538--1563}), \textsc{iii},
Roma, 1958, p. 508.} dão certo azeite que serve para a candeia: têm
grande virtude, como escreve Monardes,\footnote{ O Pe. Fernão Cardim
refere"-se a Nicolás Monardes (1493--1588), médico e naturalista espanhol, de Sevilha, 
que apesar de nunca ter atravessado o Oceano dedicou"-se ao
estudo das produções naturais da América, que conseguia obter por meio
dos viajantes. Desse modo conseguiu formar um pequeno museu de
História Natural, em Sevilha, que foi dos mais antigos, existindo já em
1554. A principal de suas obras intitula"-se de \textit{Primera y
segunda y tercera partes de la Historia medicinal de las cosas que se
traen de nuestras Indias Occidentales, que sirven en Medicina etc.}, 
publicada em Sevilha em 1574, e onde se acham reunidos diversos
tratados anteriormente dados à estampa. A primeira parte foi publicada
em 1565 e depois em 1569; a segunda em 1571. A referência de Cardim
encontra"-se in fl. 6v. da primeira parte do livro de Monardes, quando
trata do azeite da figueira do inferno.} e as folhas são muito
estimadas para quem arrevessa, e não pode ter o que come, untando o
estômago com óleo, tira as opilações, e cólica; para se tirar este
óleo, põem"-na ao sol alguns dias, e depois a pisam, e cozem, e logo lhe
vem aquele azeite acima que se colhe para os sobreditos efeitos. 

\especie{Igbacamuci}\footnote{ \textit{Igbacamuci}: não encontramos
referência científica a esta árvore. O termo tupi vem de
\textit{ybá} = ``fruta'' + \textit{cambucy} ou \textit{camucy} = ``pote'',
ou seja, ``pote de fruta'', como descreve Cardim neste texto.} 
Destas árvores há muitas em São Vicente: dão umas frutas,
como bons marmelos da feição de uma panela, ou pote; tem algumas
sementes dentro muito pequenas, são único remédio para as câmaras de sangue.\footnote{ Diarreia. [\versal{N.}~do \versal{E.}]}

\especie{Igcigca}\footnote{ \textit{Igcigca, icica} ou
\textit{almecegueira}, da família das Burseráceas (\textit{Protium
brasiliense}, Eng.). Ocorre pela primeira vez num texto português com
Cardim. O termo tupi vem de \textit{y"-cyca} = ``água pegajosa'', ``goma'',
``resina''.} Esta árvore dá a almécega;\footnote{ Cardim
refere"-se à \textit{almécega} que é uma goma resinosa aromática,
translúcida e adstringente extraída da aroeira, almecegueira ou
lentisco, usada em produtos farmacêuticos e vernizes, o que coincide
com o termo tupi mencionado por Cardim.} onde está cheira muito por um
bom espaço, dão"-se alguns golpes na árvore, e logo incontinente estila
um óleo branco que se coalha; serve para emplastos em doenças de
frialdade, e para se defumarem; também serve em lugar de incenso.

 Há outra árvore desta casta chamada \textit{Igtaigcigca},\footnote{ \textit{Igtaigcica} 
ou \textit{itaycyca} é a almecegueira
(\textit{Protium icicariba}) que é um arbusto da família das
Anacardiáceas, também denominada de \textit{almécega, aroeira} ou
\textit{lentisco}, que produz uma goma chamada almécega. O nome tupi
significa resina ou goma de pedra, enxofre, o que confirma a descrição
cardiniana, onde ocorre pela primeira vez.} sc.~almécega dura como
pedra, assim mais parece anime\footnote{ Cardim distingue o
\textit{anime} da \textit{almécega}, ainda que ambas sejam espécies de
resina cor de enxofre e muito aromática, também denominada ainda de
\textit{goma"-copal.}} do que almécega, e é tão dura e resplandecente,
que parece vidro, e serve de dar vidro à louça, e para isto é muito
estimada entre os índios, e serve também para doenças de frialdade.

 Há um rio entre Porto Seguro, e os Ilhéus\footnote{ O rio que Cardim
diz lançar"-se ao mar entre Ilhéus e Porto Seguro, e que vem do sertão
alto, deve ser o Jequitinhonha} que vem mais de 300 léguas
pelo sertão: traz muita cópia de resina que é o mesmo anime, a que os
índios chamam \textit{Igatigcica} e os portugueses incenso branco, e
tem os mesmos efeitos que o incenso.

\especie{Curupicaigba}\footnote{ \textit{Curupicaigba} ou
\textit{curupicahiba} é uma planta da família das Euforbiáceas,
usualmente designada por ``leiteira'' ou ``pau"-de"-leite'', o que condiz
com a descrição de Cardim. Ocorre pela primeira vez num texto português
com Cardim.} Esta árvore parece na folha com os pessegueiros
de Portugal; as folhas estilam um leite como o das figueiras de
Espanha, o qual é único remédio para feridas frescas e velhas, e para
boubas,\footnote{ Ao longo dos seus textos este autor menciona muitas
vezes as \textit{boubas}, que eram a doença mais comum entre os
indígenas, denominada em tupi \textit{piã}, que é uma treponematose não
venérea que provoca lesões cutâneas e ósseas. Do contato com os
ameríndios esta doença, assim como a sífilis, acabou por ser introduzida na
Europa. Cf. Jorge Couto, \textit{A Construção do Brasil. Ameríndios,
Portugueses e Africanos, do início do povoamento a finais de
Quinhentos}, Lisboa, Ed. Cosmos, 1995, pp. 326--330.} e das feridas
tira todo sinal; se lhe picam a casca deita grande quantidade de visco
com que se tomam os passarinhos.

\especie{Caaroba}\footnote{ \textit{Caaroba} ou \textit{caroba} é
uma árvore da família das Bignoniáceas, do gênero Jacarandá
(\textit{Jacaranda caroba}, Vell.). Algumas espécies fornecem madeira
de boa qualidade para a marcenaria e carpintaria. Esta espécie descrita
por Cardim parece ser a \textit{Jacaranda brasiliana}, que é um pequeno
arbusto de casca acinzentada, folhas pecioladas e flores azuis,
campanuladas, grandes, que fornece através da casca um produto
utilizado pela medicina popular contra afecções do sistema urinário. O
termo tupi vem de \textit{caá} = ``folha de planta'' + \textit{roba} = ``amargosa'', 
``acre''. Ocorre pela primeira vez num texto em português com
Cardim.} Destas árvores há uma grande abundância, as folhas
delas mastigadas, e postas nas boubas as fazem secar, e sarar de
maneira que não tornam mais, e parece que o pau tem o mesmo efeito que
o da China, e Antilhas para o mesmo mal. Da flor se faz conserva para
os doentes de boubas.

\especie{Caarobmoçorandigba}\footnote{ \textit{Caarobmoçorandigba,
maçarandiba} ou \textit{maçaranduba}: da família das Sapotáceas
(\textit{Mimusops elata}, Fr. All.) do gênero \textit{Manikara}. São
árvores de grande porte, de tronco reto, que fornecem madeira escura,
compacta e resistente, própria para obras externas, carpintaria e
marcenaria. Ocorre pela primeira vez num texto português em Cardim. O
termo tupi é de étimo incerto.} Este pau parece
que é o da China: toma"-se da mesma madeira que o de lá, e sara os
corrimentos, boubas, e mais doenças de frialdade; é pardo, e tem o
âmago duro como pau"-da"-China.\footnote{ Cardim menciona o \textit{Pau
da China} que possivelmente era o \textit{Pau"-de"-cobra} ou
\textit{Pau"-de"-maluco}, ambas originárias do Oriente, de origem vegetal
e que tinham utilidade na farmacopeia.} 

\especie{Iabigrandi}\footnote{ \textit{Iabigrandi} ou \textit{jaburandi}
é um arbusto da família das Piperáceas e Rutáceas 
(\textit{Pilocarpus pinatifolius}, Linn.). São plantas que fornecem
folhas consideradas medicinais e utilizadas como sudoríferas e
excitantes das glândulas salivares. O termo tupi é de difícil
explicação e ocorre pela primeira vez num texto português com Cardim.}
 Esta árvore há pouco que foi achada, e é, como dizem
alguns indiáticos, o Betele\footnote{ Cardim menciona o \textit{betele}
que é um arbusto existente na Índia, o \textit{Piper betle}, Linn.
\textit{Betle, betre, bétele} ou \textit{bétel}, é, conforme o conde de
Ficalho em nota aos \textit{Colóquios}, de Garcia da Orta (vol. \textsc{ii},
p. 402), a adaptação portuguesa do tamil \textit{vettilei} e do malaio
\textit{vetilla}, que se diz significar apenas ``folha''. A folha desta
planta é habitualmente mascada juntamente com a areca, tanto na Índia
como no Sueste Asiático. A esta planta se refere Duarte Barbosa, que a
descreve e explica mesmo a sua utilização: ``[\ldots{}] não dão estas
árvores nenhum fruto, somente uma folha muito hemática que em todas as
Índias costumam muito comer assim homens como mulheres, assim de dia
como de noite: pelas praças e caminhos de dia; e até de noite na cama,
de maneira que nunca deixam de comer, a qual folha é misturada com uma
pome pequena que chamam areca; e quando a hão"-de comer, primeiro a
untam com cal molhada que é feita de casca de mexilhões e amêijoas''. 
Cf. Duarte Barbosa, \textit{op. cit.}, pp. 202--203. Mais tarde, Garcia
da Orta mostra como os portugueses acabaram por servir"-se desta planta,
mesmo nos seus pormenores menos aceitáveis, sempre do ponto de vista
médico, como menciona igualmente Cardim. Cf. Garcia da Orta,
\textit{op. cit.}, pp.~343--352.} nomeado da Índia; os rios e ribeiros
estão cheios destas árvores: as folhas comidas são o único remédio para
as doenças de fígado, e muitos neste Brasil sararam já de mui graves
enfermidades do fígado, comendo delas.

 Há outra árvore também chamada Betele, mais pequena, e de folha
redonda; as raízes dela são excelente remédio para a dor de dentes,
metendo"-a na cova deles, queima como gengibre.
 
Dizem também que há neste Brasil a árvore da \mbox{canafístula};\footnote{ \textit{Canafístula} 
(\textit{Cassia ferruginea}, Schrad.), da família das
Leguminosas"-Cesalpináceas. É uma árvore de cerca de 10\,m de altura, de
casca cinzenta, rugosa, de folhas compostas e flores amarelas,
dispostas em longos racimos pêndulos e aromáticas. Fornece madeira
clara, porosa, própria para carpintaria e caixotaria. As vagens tinham
aplicações medicinais e a casca, sendo rica em tanino, é utilizada nos
curtumes e na tinturaria. Existe uma variedade originária da região de
Cambaia, na Índia, a \textit{Cassia fistula}, que os Venezianos já
distribuíam na Europa. Coube aos portugueses a difusão da variedade
brasílica (\textit{Cassia ferruginea}). Há referência à sua introdução
no Reino em um auto notarial redigido em Lisboa a 20 de maio de 1503,
onde o impressor Valentim Fernandes sublinhava que as mercadorias que
os navios lusos traziam da Terra Santa Cruz eram as seguintes: 
``[\ldots{}] pau"-brasil, \textit{cassia linea} e outras \textit{cassia
fístulas}, bem como papagaios de diversas espécies''. Cf. in António
Alberto de Andrade, ``O auto notarial de Valentim Fernandes (1503) e o
seu significado como fonte histórica'', in \textit{Arquivos do Centro
Cultural Português} (Paris), \textsc{v}, (1972), p. 544, cit. in Jorge Couto,
\textit{op. cit.}, pp. 282--283. Existem várias espécies ao longo do
território brasileiro, como a Canafístula"-do"-brejo (\textit{Cassia
nana}) de pequeno porte e a Canafístula"-verdadeira (\textit{C.
fistula}) que atinge grandes dimensões. Os índios denominavam"-na de
\textit{geneúna, geneuva, janauba} ou \textit{úanauma.}} é ignota aos
índios; os Espanhóis usam dela e dizem que é tão boa como a da Índia.

\paragraph{Dos óleos de que usam os índios para se untarem}

\especie{Andá}\footnote{ \textit{Andá} ou \textit{andá"-açu}, 
árvore da família das Euforbiáceas (\textit{Johannesia princeps}, Vell.). 
Fornece madeira branca, macia, usada para a fabricação de
palitos de fósforos, caixotaria e papel. A casca e as sementes são
consideradas medicinais. Ocorre pela primeira vez num texto português
com Cardim. O termo tupi vem de \textit{a"-ãtã} = ``fruto rijo'', ``a noz'',
``amêndoa dura''.} Estas árvores são formosas, e grandes, e
a madeira para tudo serve; da fruta se tira um azeite com que os índios
se untam, e as mulheres, os cabelos, e também serve para feridas, e as
seca logo. E também fazem muitas galanterias pelo corpo, braços e
pernas com este óleo, pintando"-se.

\especie{Moxerecuigba}\footnote{ \textit{Moxerecuigba} ou
\textit{moxiricuíba}: árvore ou arbusto difícil de identificar que
ocorre a primeira vez num texto português com Cardim.} Esta árvore se
acha no sertão nos campos; é pequena, dá uma fruta do tamanho de
laranja, e dentro dela tem umas pevides, e de tudo junto fazem um
azeite para se untarem; a casca serve para barbasco dos peixes, e todo
animal que bebe da água donde se deita, morre. 

\especie{Aiuruatubira}\footnote{ \textit{Airuatubira} ou
\textit{ajuruatubira}: tal como a anterior, é uma árvore difícil de
identificar e que ocorre a primeira vez num texto português com
Cardim.} Esta árvore que é pequena dá uma fruta vermelha, e
dela se tira um óleo vermelho com que se untam os índios. 

\especie{Aiabutipigta}\footnote{ \textit{Aiabutipigta}, 
\textit{jabotapita} ou \textit{jabutapitá}: planta da família das
Ocnáceas, é segundo Martius a \textit{Gomphia parviflora}, D.C. O
termo tupi ocorre pela primeira vez num texto português com
Cardim.} Esta árvore será do comprimento de cinco, seis
palmos; é como amêndoas, e preta, e assim é o azeite que estimam muito,
e se untam com ele em suas enfermidades.

\especie{Ianipaba}\footnote{ \textit{Ianipaba}, \textit{genipapo} ou
\textit{jenipapo}, fruto e árvore da família das Rubiáceas (\textit{Genipa americana}, L.). 
É uma árvore de caule reto, ramificada
na copa, folhas opostas e flores branco"-amareladas. O fruto é uma baga
globosa de casca mole, aromática, com polpa adocicada sendo muito
utilizado no fabrico de licores e compotas. A casca é rica em tanino e
serve para curtume, fornece madeira clara, flexível e resistente, com
utilização diversificada. O fruto fornece suco de cor azul escura, que
se torna negro, por oxidação e que era muito utilizado pelos indígenas
na pintura corporal. O nome tupi explica"-se por \textit{nhandipab} ou
\textit{jandipab} = ``fruta de esfregar'' ou ``que serve para pintar'', o
que coincide com a descrição de Cardim. Este termo ocorre pela primeira
vez num texto português em 1574, num \textit{Inventário}, pub. in
\textit{Documentos para a História do Açúcar}, ``Engenho, Sergipe do
Conde'', Espólio de Mem de Sá (1569--1579), Instituto do Açúcar e do
Álcool, Rio de Janeiro, vol. \textsc{iii}, 1963, p. 335: ``[\ldots{}] outra caixa de
pau de jenipapo'' e com maior desenvolvimento, em 1576, no
\textit{Tratado da Província do Brasil}, de Pêro de Magalhães de Gândavo.} 
Esta árvore é muito formosa, de um verde alegre,
todos os meses muda a folha que se parece com folha de nogueira; as
árvores são grandes, e a madeira muito boa, e doce de lavrar; a fruta é
como grandes laranjas, e se parece com marmelos, ou peras pardas; o
sabor é de marmelo: é boa mezinha para câmaras de toda ordem. Desta
fruta se faz tinta preta, quando se tira é branca, e em untando"-se com
ela não tinge logo, mas daí a algumas horas fica uma pessoa tão preta
como azeviche; é dos índios muito estimada, e com esta fazem em seu
corpo imperiais gibões, e dão certos riscos pelo rosto, orelhas,
narizes, barba, pernas e braços, e o mesmo fazem as mulheres, e ficam
muito galantes, e este é o seu vestido assim de semana, como de festa,
ajuntando"-lhe algumas penas com que se ornam, e outras jóias de osso;
dura esta tinta no corpo assim preta nove dias, e depois não fica nada,
faz o couro muito duro, e para tingir há"-de"-se colher a fruta verde,
porque madura não tinge.

\especie{Iequigtiygoaçu}\footnote{ \textit{Iequigtygoaçu}: segundo o
texto cardiniano, deve ser o \textit{saboeiro} da família das
Sapindáceas (\textit{Sapindus divaricatus}, Will e Camb.). A casca
polposa do fruto, esfregada na água, produz espuma e é empregada como
sabão para lavar roupa; as sementes servem para botões, ou segundo o
texto de Cardim para contas. O termo tupi é difícil de explicar, no
entanto \textit{quity} = ``esfregar'' ou ``limpar'' pode aplicar"-se às
árvores que se chamam vulgarmente de saponárias.} Esta
árvore dá umas frutas como madronhos, e dentro uma conta tão rija como
um pau que é a semente; são das melhores contas que se podem haver
porque são muito iguais, e muito pretas, e tem um resplandor como de
azeviche; a casca que cobre estas contas, amarga mais que piorno, serve
de sabão e assim ensaboam como o melhor de Portugal. 



\paragraph{Da árvore que tem água}

Esta árvore se dá nos campos e sertão da Bahia em lugares
onde não há água; é muito grande e larga, nos ramos tem uns buracos de
comprimento de um braço que estão cheios de água que não tresborda nem
no Inverno, nem no Verão, nem se sabe donde vem esta água, e quer dela
bebam muitos, quer poucos, sempre está no mesmo ser, e assim serve
não somente de fonte mas ainda de um grande rio caudal, e acontece
chegarem 100 almas ao pé dela, e todos ficam agasalhados, bebem, e
levam tudo o que querem, e nunca falta água; é muito gostosa, e clara,
e grande remédio para os que vão ao sertão quando não acham
outra.\footnote{ A árvore a que o Pe. Fernão Cardim se refere deve ser
uma leguminosa da família das Papelionáceas, a \textit{Geoffraea
spinosa}, L., vulgarmente conhecida por \textit{umary}, que vegeta no
nordeste brasileiro e que, dos espinhos que a cobrem toda, verte
líquido em tal quantidade, que, às vezes, no inverno, chega a molhar o
solo, o que para o sertanejo é presságio de chuva. O próprio termo tupi
é contração de \textit{ymbo"-ri"-y} = ``árvore que faz com que verta
água''. No entanto, o fenômeno descrito pelo Pe. Fernão Cardim, da
``árvore fonte'', está um pouco exagerado. Ocorre pela primeira vez num
texto português com a designação do termo tupi em 1618, no
\textit{Diálogo das Grandezas do Brasil.}} 


\paragraph{Das árvores que servem para madeira}

Neste Brasil há arvoredos em que se acham árvores de notável
grossura, e comprimento, de que se fazem mui grandes canoas, de largura
de 7 e 8 palmos de vão, e de comprimento de cinquenta e mais palmos,
que carregam como uma grande barca, e levam 20 e 30 remeiros; também se
fazem mui grandes gangorras para os engenhos.

 Há muitos paus como incorruptíveis que metidos na terra não apodrecem,
e outros metidos na água cada vez são mais verdes, e rijos. Há pau
santo,\footnote{ \textit{Pau"-santo}, árvore da família das
Leguminosas, Cesalpináceas, (\textit{Zollernia paraensis}, Hub.). De
grande porte, de casca clara, folhas lanceoladas e flores dispostas em
panículas. Fornece madeira vermelho"-escura, quase preta, com grandes
manchas amarelo"-esverdeadas, pesada, resistente, utilizada em
marcenaria.} de umas águas brancas de que se fazem leitos muito ricos,
e formosos. Pau do Brasil,\footnote{ \textit{Pau"-brasil}, árvore da
mesma família e subfamília da anterior (\textit{Caesalpinia echinata}, 
Lamk.). É de grande porte que pode atingir 30 metros de altura, de
tronco e ramos armados de espinhos, folhas bipinadas e flores amarelas
e aromáticas. Fornece madeira pesada, dura, alaranjada, quando fresca,
passando a vermelho"-violácea. Esta cor é devido à presença de um
corante solúvel em água conhecido como \textit{brasilina}, que depois
de extraído oxida dando origem à \textit{brasileína}, que era utilizado
para tingir tecidos e fabricar tinta de escrever. O nome tupi é
\textit{Ibirapitinga}, por \textit{ybyrá} = ``árvore'', ``pau'', ``madeira''
+ \textit{pitanga} = ``vermelho''.} de que se faz tinta vermelha, e
outras madeiras de várias cores, de que se fazem tintas muito
estimadas, e todas as obras de torno e marcenaria. Há paus de cheiro,
como Jacarandá,\footnote{ \textit{Jacarandá} é a designação comum a
diversas espécies da família das Leguminosas e das Papilionáceas. São
árvores de porte regular ou arbustos grandes que fornecem madeira com
diversas utilidades, geralmente resistentes em obras expostas e muito
procuradas no comércio das madeiras. Entre as diversas espécies existem
no território brasileiro a Jacarandá"-cabiúna (\textit{Dalbergia
lateriflora}), a Jacarandá"- da"-Bahia (\textit{D. nigra}), a
Jacarandá"-de"-espinho (\textit{Machaerium leucopterum}), a
Jacarandá"-do"-campo (\textit{Swartzia fugax}), a Jacarandá"-mimoso
(\textit{Jacaranda mimosaefoliae}), entre outras. Ocorre pela primeira
vez num texto português com Cardim.} e outros de muito preço e estima.
Acham"-se sândalos brancos\footnote{ \textit{Sândalo branco} é
originário do sul da Índia e apesar de Cardim a ele se referir não
parece que tenha sido introduzido no Brasil.} em quantidade. Pau de
aquila\footnote{ \textit{Pau d'aquila} ou \textit{pau de aguila} é
da família da Aquilarináceas (\textit{Aquilaria agallocha}) originário
da Indochina e, tal como a anterior, não parece que tenha sido
introduzida no Brasil.} em grande abundância que se fazem navios dele,
cedros,\footnote{ \textit{Cedro}: da família das Meliáceas 
(\textit{Cedrela glaziovii}.), existem diversas espécies no território
brasileiro, mais precisamente cerca de 130. A madeira é resistente e
útil para obras hidráulicas, construção civil e marcenaria.} pau de
angelim,\footnote{ \textit{Pau d'angelim} ou \textit{angelim} é a
designação para certas árvores Leguminosas, Caesalpináceas, como a 
\textit{Machaerium heterptenium}, Fr. All e a \textit{Hymenolobium} 
comumente denominada de Angelim"-do"-Pará, entre muitas outras como
Angelim"-coco ou Angelim"-doce, Angelim"-de"-espinho, Angelim"-do"-campo,
Angelim"-rosa e Angelim"-pinima. São árvores de grande porte e crescem
nas regiões tropicais da Ásia e da América. É árvore de madeira muito
valiosa e largamente utilizada no Malabar. A palavra corresponde ao
indo"-britânico \textit{angelywood}; também é conhecida na Índia como
\textit{Jaqueira"-brava .} } e árvore de noz moscada;\footnote{ \textit{Noz moscada} 
é a semente ou caroço do fruto da \textit{Myristica fragrans}, Houtt., pequena árvore da família das
Miristicáceas. É uma árvore ou arbusto de folhas alternas, persistentes
e flores dioicas. Os frutos são bagas carnosas, aromáticas, cujo arilo
fornece a noz"-moscada e o macis, ambos utilizados em culinária, como
especiarias. Na época da expansão portuguesa na Ásia apenas se dava nas
ilhas de Banda, vindo somente em finais do século \textsc{xviii} a ser
aclimatada nas Antilhas pelos Franceses e Ingleses e pelos Portugueses
no Brasil. Existe uma árvore, a \textit{noz"-moscada"-do"-Brasil} (\textit{Cryptocarya moschata}), 
de grande porte, de casca lisa e esbranquiçada,
e cujos frutos possuem polpa carnosa e um caroço sulcado como o da noz
moscada asiática. Fornece madeira escura, pesada e aromática, utilizada
em obras internas e carpintaria.} e ainda que estas madeiras não sejam
tão finas, e de tão grande cheiro como as da Índia, todavia falta"-lhes
pouco, e são de grande preço, e estima.

\paragraph{Das ervas que são fruto e se comem}

\especie{Mandioca}\footnote{ \textit{Mandioca} é da família das
Euforbiáceas a \textit{Manihot esculenta Grantz}, antigamente
classificada de \textit{Manihot utilissima}, Pohl. É um arbusto de
raízes tuberosas, folhas palmiformes de cor verde"-azulada e flores de
cálice amarelo, dispostas em panículas, com uma altura que varia entre
1,50 e 2,40\,m. Os tubérculos são ricos em amido, comestíveis depois de
cozidos ou utilizados na fabricação do polvilho e da farinha de
mandioca, alimento básico em muitas regiões do Brasil. Com esta planta
preparam"-se inúmeras iguarias, como descreve no seu texto o Pe. Fernão
Cardim. A atestar a sua importância como alimento indispensável aos
indígenas e até aos primeiros colonizadores que a ela se adaptaram,
existe um grande número de documentos que a ela se referem. Nenhum 
outro vocábulo de origem tupi está mesmo tão amplamente documentado na
língua portuguesa como a mandioca. Todos os autores são unânimes em
atribuir"-lhe as mais notáveis qualidades alimentícias e alguns 
colocam"-na imediatamente abaixo do trigo, considerando"-a superior ao
milho e aos outros cereais. Ocorre pela primeira vez num texto
português em 1549, numa carta do Pe. Manuel da Nóbrega: ``[\ldots{}] 
O mantimento comum da terra é uma raiz de pau, que chamam
mandioca, do qual fazem uma farinha, de que comemos todos. E dá também
milho, o qual misturado com a farinha faz um pão, que escusa o de
trigo''.} O mantimento ordinário desta terra que serve de
pão se chama mandioca, e são umas raízes como de cenouras, ainda que
mais grossas e compridas. Estas deitam umas varas, ou ramos, e crescem
até à altura de quinze palmos. Estes ramos são muito tenros, e têm um
miolo branco por dentro, e de palmo em palmo têm certos nós. E desta
grandura se quebram, e plantam na terra em uma pequena cova, e lhes
ajuntam terra ao pé, e ficam metidos tanto quanto basta para se terem,
e daí a seis, ou nove meses têm já raízes tão grossas que servem de mantimento.

 Contém esta mandioca debaixo de si muitas espécies,\footnote{ Além de
dezenas de variedades de mandioca amarga, como a
\textit{mandiocamirim}, \textit{manaibuçu, manaibaru, manaitinga}, 
entre outras, os ameríndios plantavam mandioca doce, macaxeira ou
\textit{aipim} (antigamente classificada como \textit{M. dulcis Baill).} 
Cf. Darrell A. Posey, ``Etnobiologia: Teoria e Prática'', in
\textit{Suma Etnológica Brasileira}, 1., \textit{Etnobiologia}, 
pp. 21--22, cit., in Jorge Couto, \textit{op. cit.}, p. 68.} e todas se comem
e conservam"-se dentro na terra, e até oito anos, e não é necessário
celeiro, porque não fazem senão tirá"-las, e fazer o mantimento fresco
de cada dia, e quanto mais estão na terra, tanto mais grossas se fazem,
e rendem mais.

 Tem algumas cousas de nota, sc.~que tirado o homem, todo animal se
perde por ela crua, e a todos engorda, e cria grandemente, porém se
acaba de espremer, beberem aquela água só por si, não têm mais vida que
enquanto lhe não chega ao estômago. Destas raízes espremidas, e raladas
se faz farinha que se come; também se deita de molho até apodrecer, e
depois de limpa, espremida, se faz também farinha, e uns certos 
beijus\footnote{ Os \textit{beijus} são bolos de farinha de mandioca ou
tapioca que os índios fabricavam e cujos processos de fabrico foram
transmitidos aos colonizadores, sendo com pequenas modificações, ainda
hoje empregados em várias regiões do Brasil, particularmente no
nordeste. Os portugueses acrescentaram açúcar e outros condimentos à
massa e os escravos negros enriqueceram"-nos molhando no leite de coco.
A descrição desta iguaria ocorre pela primeira vez num texto português
em 1576, no \textit{Tratado da Província do Brasil}, de Pêro de Magalhães de 
Gândavo.} como filhós, muito alvos, e mimosos. Esta mesma raiz depois
de curtida na água feita com as mãos em pilouros se põe em caniços ao
fumo, onde se enxuga e seca de maneira que se guarda sem corrupção
quanto querem e raspada do fumo, pisada em uns pilões grandes, fica uma
farinha tão alva, e mais que de trigo, da qual misturada em certa
têmpera com a crua se faz uma farinha biscoitada que chama de guerra,
que serve aos índios e portugueses pelo mar, e quando vão à guerra como biscoito. 

 Outra farinha se faz biscoitada da mesma água da mandioca verde se a 
deixam coalhar e enxugar ao sol, ou fogo; esta é sobre todas alvíssima,
e tão gostosa, e mimosa que não faz para quem quer. Desta mandioca
curada ao fumo se fazem muitas maneiras de caldos que chamam
mingaus,\footnote{ O Pe. Fernão Cardim refere"-se ao \textit{mingau} que
é um alimento de consistência pastosa, uma espécie de papa preparada
com farinha de mandioca ou de trigo, ou ainda fubá, aveia, ou outra
farinha, diluída e cozida em água ou em leite e a que se adicionam
açúcar, ovos, canela, entre outros condimentos. O nome tupi ocorre pela
primeira vez em português neste texto de Cardim.} tão sadios, e
delicados que se dão aos doentes de febres em lugar de amido, e
tisanas, e da mesma se fazem muitas maneiras de bolos, 
coscorões,\footnote{ Outro alimento fabricado com farinha de mandioca, neste caso
o \textit{coscorão} que é uma filhó feita de farinha e ovos.} tartes,
empenadilhas, queijadinhas de açúcar, etc, e misturada com farinha de
milho, ou de arroz, se faz pão com fermento, e lêvedo que parece de
trigo. Esta mesma mandioca curada ao fumo é grande remédio contra a
peçonha, principalmente de cobras. 

 Desta mandioca há uma que chamam \textit{aipim}\footnote{ \textit{Aipim} 
é uma das variedades da mandioca que também é muito
descrita nos primeiros textos. O Pe. Simão de Vasconcelos, em 1663, na
sua obra \textit{Coisas do Brasil}, apresenta treze variedades
distintas de aipim, nomeando"-as de acordo com a terminologia indígena.
O nome tupi ocorre pela primeira vez em castelhano, em 1554, numa
carta do Pe. Luís da Grã e em português, em 1576, no
\textit{Tratado da Província do Brasil}, de Pêro de Magalhães de Gândavo.} 
 que contém também debaixo de si muitas espécies. Esta não mata crua, e
cozida, ou assada, que é de bom gosto, e dela se faz farinha, e beijus,
etc. Os índios fazem vinho dela, e é tão fresco e medicinal para o
fígado que a ele se atribui não haver entre eles doentes do fígado.
Certo gênero de Tapuias come a mandioca peçonhenta crua sem lhe fazer
mal por serem criados nisso.

 Os ramos desta erva, ou árvores são a mesma semente, porque os paus
dela se plantam, as folhas, em necessidade, cozidas servem de mantimento. 

\especie{Naná}\footnote{ \textit{Naná} ou \textit{ananás}, 
da família das Bromeliáceas (\textit{Ananassa sativa}, Lindl.). É uma
planta originária da América tropical, cultivada em muitas regiões
quentes. O ananás despertou significativo interesse entre os primeiros
autores que escreveram sobre o Brasil, considerando todos que se
tratava de uma planta com inúmeras qualidades, alimentares e
medicinais. O interesse era de tal forma que há referências do envio de
ananases em conserva para o Reino desde o início do período de
colonização. A isso se refere o Pe. Manuel da Nóbrega, em carta datada
de 1561, mencionando o seu valor para tratamento dos doentes de ``dor
de pedra'': ``[\ldots{}] O mestre leva estas conservas pera os enfermos,
scilicet, os ananazes, pera dor de pedra, os quais posto que não tenham
tanta virtude como verdes, todavia fazem proveito''. Cf. Pe. Manuel da
Nóbrega, in \textit{Cartas}, Serafim Leite, 1955, p. 377. O termo tupi
ocorre pela primeira vez num texto português em 1557, na
\textit{Relação do Descobrimento da Florida}, de D. Fernando Souto. Não
há dados precisos sobre a origem tupi desta planta, mas se for esse o
caso, vem de \textit{na"-nã} = ``cheira"-cheira'', o que coincide com as
características dessa planta.} Esta erva é muito comum,
parece"-se com erva babosa, e assim tem as folhas, mas não tão grossas e
todas em redondo estão cheias de uns bicos muito cruéis; no meio desta
erva nasce uma fruta como pinha, toda cheia de flores de várias cores
muito formosas, e ao pé desta quatro, ou cinco olhos que se plantam; a
fruta é muito cheirosa, gostosa, e uma das boas do mundo, muito cheia
de sumo e gostoso, e tem sabor de melão ainda que melhor, e mais
cheiroso; é boa para doentes de pedra, e para febres muito prejudicial.
Desta fruta fazem vinho os índios muito forte, e de bom gosto. A casca
gasta muito o ferro ao aparar, e o sumo tira as nódoas da roupa. Há
tanta abundância desta fruta que se cevam os porcos com ela, e não se
faz tanto caso pela muita abundância: e também se fazem em conserva, e
cruas desenjoam muito no mar, e pelas manhãs com vinho são \mbox{medicinais.} \enlargethispage{\baselineskip}

\especie{Pacoba}\footnote{ \textit{Pacoba} ou \textit{pacova} é o
nome do fruto das Musáceas ou bananeiras indígenas, que se agrupa em
cachos de tamanhos variados; a casca de cor amarela (mais comum), verde
ou avermelhada, recobre a polpa amilácea, nutritiva e saborosa,
comestível crua ou cozida, e utilizada ainda na produção de doces.
Existem várias espécies de bananeiras no território brasileiro, como a
banana"-da"-terra (\textit{Musa paradisiaca}), banana"-do"-mato
(\textit{Monstera deliciosa}), banana"-figo (\textit{Musa sapientum}), 
banana"-nanica (\textit{Musa cavendishii}), entre outras. Os botânicos
acreditam que a bananeira de São Tomé foi transplantada pelos
portugueses no início do século \textsc{xvi}, apesar de Jean de Léry e Gabriel
Soares de Sousa afirmarem que as bananas, em tupi \textit{pacovas} ou
\textit{pacobas}, eram naturais da terra. Afirma este último: ``Pacoba é
uma fruta natural desta terra, a qual se dá em uma árvore muito mole e
fácil de cortar, cujas folhas são de doze a vinte palmos de comprido e
de três a quatro de largo\ldots{}'', in Gabriel Soares de Sousa,
\textit{Notícia do Brasil}, Lisboa, Pub. Alfa, 1989, p. 123. No
entanto, quer este autor, quer Léry chegaram ao Brasil apenas na
segunda metade do Quinhentos, sendo possível que a sua difusão tenha
ficado a dever"-se aos membros das guarnições das feitorias de Cabo Frio
e Igaraçu ou aos primeiros colonos. Cf. Carl O. Sauer, ``As Plantas
Cultivadas na América do Sul Tropical'', in \textit{Suma Etnológica
Brasileira, 1. Etnobiologia}, coord. de Berta G. Ribeiro, 2ª ed.,
Petrópolis, 1987, pp. 77--78 e Jorge Couto, \textit{op. cit.}, p. 329. O
nome tupi ocorre pela primeira vez na língua portuguesa em 1576, com
Pêro de Magalhães de Gândavo, no \textit{Tratado da Província Santa Cruz.}
O termo tupi vem de \textit{pac"-oba} = ``folha de enrolar'', o que
coincide com a forma como a folha da bananeira era utilizada pelos
índios.} Esta é a figueira que dizem de Adão,\footnote{ O
nome \textit{Figueira"-de"-Adão} é um termo botânico para designar a
\textit{bananeira.} Mas a designação da bananeira nas descrições dos
cronistas portugueses, e não só, encontrou uma certa incerteza
classificatória. Clúsio, na sua viagem a Portugal, dá notícia de que
viu ``algumas plantas pouco frutíferas'', chamando"-lhe ``figueira bana,
isto é, que produz um figo banana'', in \textit{Aromatum ed simplicium
Aliquot medicamentorium apud indos nascentium historia}, Antuérpia,
1567, na Ex"-oficina de Christophori Plantinii, trad. em latim de Garcia
da Orta, acompanhada pelas notas de Clúsio, ed. facsimilada, trad.
portuguesa org. por Jaime Alves e Pe. Manuel Alves, Lisboa, Junta de
Investigações do Ultramar, 1964; cit. in Alfredo Margarido, \textit{As
surpresas da Flora no tempo dos Descobrimentos}, Lisboa, Ed. Elo, 1994,
nota 28, p. 126.} nem é árvore, nem erva, porque por uma
parte se faz muito grossa, e cresce até vinte palmos em alto; o talo é
muito mole, e poroso, as folhas que deita são formosíssimas e algumas
de comprimento de uma braça, e mais, todas rachadas como veludo de
Bragança, tão finas que se escreve nelas, tão verdes, e frias, e
frescas que deitando"-se um doente de febres sobre elas fica a febre
temperada com sua frialdade, são muito frescas para enramar as casas e
igrejas. Esta erva deita em cada pé muitos filhos, cada um deles dá um
cacho cheio de uns como figos, que terá às vezes duzentos, e como está
de vez se corta o pé em que está o cacho, e outros vão crescendo, e
assim vão multiplicando \textit{in infinitum}; a fruta se põe a madurar e fica
muito amarela, gostosa, e sadia, \textit{maxime} para os enfermos de febres, e
peitos que deitaram sangue; e assadas são gostosas e sadias. É fruta
ordinária de que as hortas estão cheias, e são tantas que é uma
fartura, e dão"-se todo o ano.

\especie{Maracujá}\footnote{ \textit{Maracujá} ou \textit{murucujá} 
é o nome genérico das Passifloráceas indígenas, de que há diversas
espécies, cerca de quatrocentas. São plantas usualmente trepadeiras,
cujas folhas e raízes contêm a passiflorina, de efeito semelhante à
morfina, utilizada em medicina como calmante. Existem dezenas de
espécies, sendo as mais significativas o maracujá"-de"-cobra,
maracujá"-azul (\textit{P. caerulea}), o maracujá"-azedo, maracujá"-mirim
(\textit{P. edulis)}, maracujá"-mamão (\textit{P. maliformis}), entre
outros. Ocorre pela primeira vez num texto português precisamente com o
Pe. Fernão Cardim. O nome tupi vem de \textit{mborucuyá} = ``fruto que
faz vaso'', ``que dá vasilha'', o que condiz com a descrição cardiniana e
de outros contemporâneos.} Estas ervas são muito formosas,
\textit{maxime} nas folhas; trepam pelas paredes, e árvores como a hera; as
folhas espremidas com verdete é o único remédio para chagas velhas, e
boubas. Dá uma fruta redonda como laranjas, outras à feição de ovo, uns
amarelos, outros pretos, e de outras várias castas. Dentro tem uma
substância de pevides e sumo com certa teia que as cobre, e tudo junto
se come, e é de bom gosto, tem ponta de azedo, e é fruta de que se faz caso.

 Nesta terra há outros gêneros de frutas, como camarinhas\footnote{ As
\textit{camarinhas} são pequenos frutos drupáceos.} pretas, e
vermelhas, batatas,\footnote{ A batateira (\textit{Solanum tuberosum}, L.) 
é originária da América do Sul das terras altas andinas,
possivelmente do Chile à Colômbia. O seu tubérculo já era muito
utilizado pelas populações ameríndias antes da chegada dos europeus.
Mas no território brasileiro era apenas utilizada a \textit{jetica}
ou batata"-doce (\textit{Ipomoea batatas}, L.), daí a referência de
Cardim ao fabrico de coisas doces com essa planta, a par do pão. Este
tubérculo pode desenvolver duas espécies: uma de polpa solta,
farinhenta, enxuta, de coloração branco"-amarelada e outra de polpa
mole, aquosa, mais açucarada, o que coincide com a descrição
cardiniana. Talvez, devido ao seu sabor adocicado, tenha existido um
maior interesse por esta planta do que pela simples batata, cuja
banalização na Europa só se faz a partir do século \textsc{xviii}. Cf. Alfredo
Margarido, \textit{op. cit.}, pp. 103--104. O nome indígena da
batata"-doce, \textit{jetica}, ocorre pela primeira vez num texto
português em 1631, com Frei Cristóvão de Lisboa, na \textit{História
dos Animais e Árvores do Maranhão}, fl. 176v.: ``[\ldots{}] gitica quer dizer
batata e há três castas amarelas e brancas e são muito boas cozidas ou
assadas\ldots{}''.} outras raízes que chamam mangará,\footnote{ \textit{Mangara}: 
nome comum a diversas espécies de Aroideáceas,
tubérculos comestíveis. Ocorre pela primeira vez num texto português,
precisamente com o Pe. Fernão Cardim neste \textit{Tratado} e na
\textit{Narrativa Epistolar}. O termo tupi vem de \textit{ybá"-carã} = 
``fruto redondo''} outra que chamam cará,\footnote{ \textit{Cará}: nome
habitual a diversas espécies de Dioscoreáceas indígenas, que também
produzem tubérculos comestíveis, cozidos, assados ou em forma de
farinha. Esta planta (\textit{Dioscorea trifida}) é a variedade
americana do inhame. Tal como a planta anterior ocorre pela primeira
vez num texto português com Cardim.} que se parece com nabos, e túberas
da terra. Das batatas fazem pão e várias cousas doces; têm estes índios
outros muitos legumes, sc.~favas, mais sadias e melhores que as de
Portugal, e em grande abundância, muitos gêneros de 
abóboras,\footnote{ \textit{Abóbora} é o nome comum a diversas espécies de
Cucurbitáceas de polpa comestível, que variam consoante a cor, forma e
dimensão. Uma destas espécies, possivelmente a Abóbora"-menina
(\textit{Cucurbira maxima}) era denominada de \textit{Jerimum}
ou \textit{gerumu}, em tupi, cujo nome ocorre pela primeira vez num
texto português em 1587, na \textit{Notícia do Brasil}, de Gabriel
Soares de Sousa. A planta e a palavra instalaram"-se em Portugal, já que
no Minho continua a chamar"-se \textit{jerimu} a uma cucurbitácea cuja
origem americana é evidente e que desempenha um papel importante não só
na alimentação corrente, mas no próprio ritual culinário do Natal,
quando se preparam bolinhos de jerimu, que são fritos em azeite e
polvilhados de canela, mostrando o intercâmbio de sabores entre os
portugueses e os povos com que contactaram.} e algumas tão
grandes que fazem cabaças para carretar água que levaram dois almudes,
ou mais; feijões\footnote{ O \textit{feijão} é a semente de gêneros e
espécies diferentes dos Feijoeiros (\textit{Phaseolus} sp.) da família
das Leguminosas. São originárias de diversas regiões da Índia, Japão ou
da América. Neste último caso conta"-se o \textit{Phaseolus vulgaris}, 
L. que vem modificar profundamente os hábitos alimentares das
populações rurais pelo uso de uma nova leguminosa seca concorrente da
fava, ervilha e lentilha, que já eram muito utilizadas. Existem no
território brasileiro várias espécies desta planta, como a isso se
refere Cardim no seu texto, como o Feijão"-de"-porco (\textit{Canavalia
ensiformis}), o Feijão"-fradinho (\textit{Viga sinensis mona chalis}), 
Feijão"-de"-boi (\textit{Capparis flexuosa}), entre muitas outras.} de
muitas castas, são gostosos, e como os de Portugal. Milho\footnote{ O
\textit{milho}, denominado de ``americano'' (\textit{Zea mays}, L.), 
já existia no Brasil e era muito aproveitado pelas populações
pertencentes aos grupos tribais Tupi"-Guarani, da região subtropical.
Apesar de ser o mais pobre dos cereais em proteínas, o milho tornou"-se
o produto essencial da sua dieta alimentar. Os portugueses difundiram
esta cultura na África e no Oriente, tendo a sua penetração sido mais
profunda que a da mandioca e da batata doce, que exigiam temperaturas
mais elevadas.} de muitas castas, e dele fazem pão, vinho, e se come
assado e com ele engordam os cavalos, porcos, galinhas, e umas
tajaobas,\footnote{ \textit{Tajaoba} ou \textit{tayoba}, da família
das Aroideáceas (\textit{Xanthosma violaceum}, Shott.), é uma planta de
folhas radicais, triangulares, violáceas e rizoma tuberoso, ambos
comestíveis, cultivada na América tropical. As folhas picadas e
cozidas são semelhantes à couve. O termo tupi provém de
\textit{taya"-oba} = ``folha picante'' de \textit{oba} = ``folha'' +
\textit{taya} = ``picante'' ou ``pimenta que arde''. Ocorre pela primeira
vez num texto português com o Pe. Fernão Cardim} que são como couves, e
fazem purgar, e uma erva por nome Jambig,\footnote{ \textit{Jambig,
jambi} ou \textit{nhambi} é o nome comum de duas plantas, a
\textit{Opilanthes acnella} da família das Compostas também conhecida
por \textit{agrião"-do"-pará} e a \textit{Eryngium foetidum} da família
das umbelíferas também chamada \textit{coentro"-de"-cabloco} e 
\textit{coentro"-do"-maranhão.} Ocorre pela primeira vez num texto
português com Cardim, com a primeira designação. Se esta planta é de
fato, como parece, o coentro, originário do Oriente, terá sido
introduzido no Brasil no início da colonização, como condimento e
aromatizante na culinária, além de ser também aplicado com fins
medicinais, por ser considerado como estimulante para o aparelho
digestivo.} único remédio para os doentes de fígado e pedra;
também há muitos gêneros de pimentas, que dão muito gosto ao 
comer.\footnote{ A \textit{pimenta} é uma das numerosas variedades picantes
do \textit{Capsicum annuum}, Linn. da família das Solanáceas. Como
especiaria utilizavam"-se também espécies vizinhas, em particular o
\textit{Capsicum frutescens}, Lin., de fruto mais pequeno e fusiforme,
conhecido sobretudo por \textit{piripiri} ou \textit{malagueta.} A mais
antiga referência que se conhece da existência das pimentas, no
Brasil, é a de um documento alemão de 1515, a \textit{Newen Zeitung
ausz Presillig Landt}, missiva enviada da Madeira para Antuérpia com
notícias do regresso de uma nau que fora reconhecer a costa brasileira,
onde se afirma: ``[\ldots{}] eles têm na terra uma qualidade de especiaria
que arde na língua como pimenta, e ainda mais; se cria em uma vagem,
com muitos grãozinhos dentro, sendo o grão do mesmo tamanho da
ervilha''. Pub. em tradução portuguesa in \textit{História da
Colonização Portuguesa do Brasil [\ldots{}]}, dir. de Carlos Malheiro Dias,
vol. \textsc{ii}, Porto, 1923, apêndice B, pp. 385--386. Pela descrição
corresponde melhor à pimenta"-de"-bruge ou pimenta"-de"-macaco, que é uma
espécie de \textit{Xylopia.} Cf. Luís Filipe Thomaz, ``Especiarias do
Velho e Novo Mundo'', in \textit{Arquivos do Centro Cultural Calouste
Gulbenkian}, vol. \textsc{xxxiv}, Lisboa/Paris, 1995, pp. 310--311. O primeiro
autor português a descrever detalhadamente as pimentas brasileiras é
Gabriel Soares de Sousa, em 1587, na sua \textit{Notícia do Brasil}, 
em que enumera e declara várias castas de pimenta"-da"-terra:
\textit{cuihem, sabaa, cuihemoçu, cujepia, cuihejurimu} e \textit{comari.}} 

\paragraph{Das ervas que servem para mezinhas}

\especie{Tetigcucu}\footnote{ \textit{Tetigcucu, ietigcucu} ou
\textit{jeticucu}, da família das Convolvuláceas (\textit{Ipomoea
hederacea}, Jacq.), é uma trepadeira de raiz tuberosa, folhas alternas e
flores azuis de corola branca. A raiz fornece fécula de ação purgante,
daí ser também conhecida por ``batata de purga'' ou ``tapioca de purga'' e
ainda, ``mechoacão'', como a isso se refere Cardim no seu texto, onde
ocorre pela primeira vez na língua portuguesa. Nicolás Monardes trata
na primeira parte da sua obra, já citada, do \textit{mechoacão} ou
\textit{mechoacan} (fls. 28v). Parece ter sido originária da Ásia. O
termo tupi vem de \textit{yetica} = ``batata'' + \textit{pucu} = ``longa'',
``comprida''.} Este é o Mechoacão das Antilhas; são umas
raízes compridas como rabãos, mas de boa grossura, serve de purga;
toma"-se esta raiz moída em vinho, ou água para febres, toma"-se em
conserva de açúcar como marmelada, coze"-se com galinha, faz muita sede,
mas é proveitosa, e obra grandemente.

\especie{Igpecacóaya}\footnote{ \textit{Igpecacóaya} ou \textit{
ipecacuanha}, da família das Rubiáceas (\textit{Cephaelis ipecuanha}, Baill.),
é uma erva perene, lenhosa, rasteira, de
folhas oblongas e pequenas flores brancas, com raízes, longas e
nodulosas, de onde se extrai o alcaloide emetina, com propriedades
medicinais. Ocorre pela primeira vez num texto português com o Pe.
Cardim.} Esta erva é proveitosa para câmaras de sangue: a
sua haste é de comprimento de um palmo, e as raízes de outro, ou mais;
deita somente quatro ou cinco folhinhas, cheira muito onde quer que
está, mas o cheiro é \textit{fartum}\footnote{ Em latim no manuscrito, ``odor nauseabundo''.} e
terrível; esta raiz moída, botada em um pouco de água se põe a serenar
uma noite toda, e pela manhã se aguenta a água com a mesma raiz moída,
e coada se bebe somente a água, e logo se faz purgar de maneira que
cessam as câmaras de todo.

\especie{Cayapiá}\footnote{ \textit{Cayapa, caapiá} ou \textit{capiá},
da família das Moráceas (\textit{Dorstenis brasiliensis}, Lam.), são
ervas tenras, leitosas, de flores muito pequenas inseridas num amplo
receptáculo discoide. Ocorre pela primeira vez num texto português com
Cardim. O termo tupi vem de \textit{caá} = ``erva'' + \textit{apiá} = ``testículos''.} 
Esta erva é pouco que é descoberta, é único
remédio para peçonha de toda a sorte, \textit{maxime} de cobras, e assim se chama
erva de cobra, e é tão bom remédio como unicórnio de Bada,\footnote{ O
\textit{unicórnio de Bada} é uma substância do corno do rinoceronte,
utilizada como afrodisíaco, quando reduzida a pó e cujo hábito é
originário da Ásia. Este animal, o unicórnio, despertou alguma surpresa
e curiosidade aos viajantes, desde Marco Polo, que percorreu a Ásia
entre 1275--1292, até mais tarde, no século \textsc{xvi}, aos portugueses que
descrevem o rinoceronte detalhadamente e que até importaram numerosos
exemplares para a Europa. Cf. \textit{O Livro de Marco Polo}, Lisboa,
1502, ed. de F. M. Esteves Pereira, Lisboa, 1922.} pedra de 
bazar,\footnote{ \textit{Pedra de bazar} ou \textit{pedra bezoar} era o
nome atribuído às concreções calcárias formadas em diversas partes
do corpo de certos animais, principalmente ruminantes, às quais se
atribuía na época quinhentista grande reputação. O próprio Nicolás
Monardes, citado diversas vezes pelo Pe. Fernão Cardim, escreveu um
\textit{Tratado de la piedra bezoar}, in \textit{Dos Libros, el uno que
trata de todas las cosas que traen de nuestras Indias Occidentales,
que sirven al uso de la Medicina, y el otro que trata de la piedra
bezoar, y de layerva escuerçonera}, Sevilha, en casa de Hernando Diaz,
1569, aplicando com proveito na Espanha como contraveneno,
mandando"-a vir expressamente de Lisboa.} ou coco de 
Maldiva.\footnote{ \textit{Coco"-de"-maldiva} 
ou \textit{coco"-das"-maldivas}, ou ainda
\textit{coco"-do"-mar} é o fruto da palmeira \textit{Ludoicea
seychellarum}, Labill., usual em algumas das ilhas Seicheles, e apesar
deste arquipélago só ter sido descoberto no século \textsc{xviii} e, como tal
esta palmeira, os cocos já eram conhecidos porque caindo ao mar,
flutuavam mercê das correntes e dos ventos e eram impelidos para as
praias das Maldivas, daí os nomes que adquiriu. Esta origem foi
apresentada pelos escritores portugueses quinhentistas como João de
Barros, nas \textit{Décadas}, Garcia da Orta, nos \textit{Colóquios}, 
entre outros. O próprio Luís de Camões refere"-se a estes cocos e às
suas qualidades antídotas, em \textit{Os Lusíadas}, Canto \textsc{x}, estr. 136:
 ``[\ldots{}] Nas ilhas de Maldiva nasce a planta/ No profundo das águas,
soberana/ Cujo pomo contra o veneno urgente/ É tido como antídoto
excelente.''} Não se aproveita dela mais que a raiz, que é delgada, e no
meio faz um nó como botão; esta moída, deitada em água e bebida mata a
peçonha da cobra; também é grande remédio para as feridas de flechas
ervadas, e quando algum é ferido fica sem medo, e seguro, bebendo a
água desta raiz; também é grande remédio para as febres, continuando"-a,
e bebendo"-a algumas manhãs; cheira esta erva à folha de figueira de Espanha.

\especie{Tareroquig}\footnote{ \textit{Tareroquig} ou \textit{tareroqui}: 
da família das Leguminosas (\textit{Cassia occidentalis}, L.) que
tem propriedades medicinais, forrageiras e ornamentais. Tem outros
nomes comuns, como \textit{tararacu, mangirióba, fedegoso, matapasto,
crista"-de"-gelo, lava"-pratos}, entre outros. O nome tupi é difícil de
identificar e ocorre pela primeira vez num texto português com Fernão
Cardim.} Também esta erva é único remédio para câmaras de
sangue: as raízes são todas retalhadas, os ramos muito delgadinhos, as
folhas parecem de alfavaca,\footnote{ \textit{Alfavaca} é uma
planta herbácea, muito aromática, da família das Labiadas, semelhante
ao \textit{manjericão}, também chamada \textit{alfádega} e 
\textit{alfávega.} No território brasileiro existem várias
espécies, como a Alfavaca"-cheirosa (\textit{Ocimum basilicum}) 
e a Alfavaca"-do"-campo (\textit{O. gratissimum}) e ainda a
Alfavaca"-de"-cobra (\textit{Monnieria trifolia}), que deve ser a
descrita por Cardim, já que a raiz é considerada diurética e eficiente
contra dores de ouvido e cólicas.} as flores são vermelhas, e tiram
algum tanto roxo, e dão"-se nas pontinhas. Desta há muita abundância,
quando se colhe é amarela, e depois de seca fica branca; toma"-se da
própria maneira que a precedente. Com esta erva se perfumam os índios
doentes para não morrerem, e para certa enfermidade que é comum nesta
terra, e que se chama doença do bicho, é grande remédio, serve para
matar os bichos dos bois, e porcos, e para postemas. Esta erva toda a
noite está murcha, e como dormente, e em nascendo o sol torna a abrir,
e quando se põe torna a fechar.

\especie{Goembegoaçu}\footnote{ \textit{Goembegoaçu, guembé"-guaçu} ou
\textit{imbéguaçu}, ou ainda \textit{imbé}: planta da família das
Aráceas (\textit{Philodendron, imbé}, Schott.), conhecida
vulgarmente por \textit{cipó"-de"-imbé}, tem longas raízes adventícias de
caule grosso marcado pelas cicatrizes das folhas que caíram; folhas
longo"-pecioladas, coriáceas, flores sem perianto e frutos que são bagas
presas ao tubo da espata. O termo tupi \textit{imbéguaçu} é formado de
\textit{ymbé} = ``planta rasteira'' + \textit{guaçu} = ``grande'', o que
confirma a descrição cardiniana, onde esta planta ocorre pela primeira
vez num texto português.} Esta erva serve muito para fluxo
de sangue, \textit{maxime} de mulheres; as raízes são muito compridas e algumas
de trinta, e quarenta braças. Tem uma casca rija, de que se fazem muito
fortes cordas, e amarras para navios, e são de muita dura, porque na
água reverdecem; esta tomando"-a, sc.~a casca dela, e defumando a pessoa
na parte do fluxo, logo estanca.

\especie{Caáobetinga}\footnote{ \textit{Caáobetinga}: planta difícil
de identificar cujo termo tupi traduz"-se por \textit{caá} = ``folha'' +
\textit{obi} = ``verde'' + \textit{tinga} = ``branca'', o que condiz com a
descrição cardiniana, onde ocorre pela primeira vez num texto
português.} Esta erva é pequena, deita poucas folhas, as
quais começa a lançar logo da terra, são brancas, de banda de baixo, e
de cima verdes, deitam uma flor do tamanho de avelãs; as raízes e
folhas pisadas são excelente remédio para chagas de qualquer sorte, e
também se usa da folha por pisar, a qual posta na chaga pega muito e sara.

\especie{Sobaúra}\footnote{ \textit{Sobaúra}: planta difícil de
identificar, cujo termo deve ter desaparecido da toponímia. Ocorre pela
primeira vez num texto português com Cardim e, mais tarde, em 1590, com
Francisco Soares, em \textit{Coisas Notáveis do Brasil}, cujo
testemunho coincide com o de Cardim, ``\textit{Cobaura} Serve em pós,
seca e verde, para feridas, e a própria folha é como a caubetinga,
ainda que a folha fica direita; há muita por grão, seca que vá. E há
outra como esta, tem o mesmo efeito''. Cf. Francisco Soares,
\textit{op. cit.}, in \textit{O Reconhecimento do Brasil}, dir. de Luís
de Albuquerque, Lisboa, Pub. Alfa, 1989, p. 168.} Esta erva
serve para chagas velhas, que já não têm outro remédio: deita"-se moída
e queimada na chaga, logo come todo o câncer, e cria couro o novo;
também se põe pisada e a folha somente para encourar.

\especie{Erva santa}\footnote{ \textit{Erva"-santa, fumo, tabaco,
petigma} ou \textit{petume} é a planta da família das Solanáceas
(\textit{Nicotiana tabacum}, L.) usualmente designada por ``tabaco''. É
uma planta herbácea, anual, que chega a atingir 2\,m de altura,
tomentosa, de folhas amplas, oblongas, acuminadas e macias, e flores
avermelhadas. A infusão das folhas é utilizada como insetífuga, e,
quando dessecadas, as folhas constituem o fumo ou tabaco. Ocorre pela
primeira vez num texto português em 1566, com Damião de Góis na
\textit{Crónica do Felicíssimo Rei Dom Manuel}, p.~1ª, cap. 56, fl. 52,
``[\ldots{}] A terra é muito viçosa, muito temperada e de muito bons ares,
muito sadia, [\ldots{}] há muitas ervas odoríferas e medicinais, delas
diferentes das nossas, entre as quais é a que chamamos de fumo, e eu
chamaria erva santa, e que dizem que eles chamam Betum\ldots{}''. Este dá
ainda a conhecer, no mesmo texto, que terá sido Luís de Góis que
trouxe o tabaco pela primeira vez, do Brasil para Portugal. Mas o
nome latino que define o lugar da planta na flora mundial,
\textit{Nicotiana tabacum}, sp., é atribuído a Jean Nicot (1559--1561), 
embaixador francês, em Lisboa, que a enviou para Paris, com destino à
regente Catarina de Médicis, com recomendações sobre as suas
virtualidades medicinais. Divulgou"-se, inicialmente, na França com a
designação de ``erva da rainha'' e, depois, em homenagem àquele diplomata
francês, passado a chamar"-se ``nicotina''. Cf. Edmond Falgairolle,
\textit{Jean Nicot: Sa Correspondance Diplomatique}, Paris, 1897, p.
66; Carlos França, ``Os Portugueses do século \textsc{xvi} e a História Natural
do Brasil'', in \textit{Revista de História} (Lisboa), \textsc{xv} (57--60),
1926, pp. 81--84 e Jorge Couto, \textit{op. cit.}, pp. 328--329. Cardim
elabora um texto mais pormenorizado sobre esta erva e a sua utilização,
no \textit{Tratado} referente aos índios do Brasil.} Esta
erva santa serve muito para várias enfermidades, como feridas,
catarros, e principalmente serve para doentes da cabeça, estômago e
asmáticos. Nesta terra se fazem umas cangueras\footnote{ A
\textit{canguera} ou \textit{cangoeira} é uma espécie de canudo,
confeccionado com folhas de palmeira, que os indígenas utilizavam para
fumar, como menciona Cardim neste texto e descreve de forma mais
cuidada no \textit{Tratado} referente aos índios. É também uma espécie
de flauta rústica, fabricada com ossos descarnados, utilizada pelos
índios nas suas festividades. O termo tupi é formado de \textit{cang} =
``osso'', com o sufixo de pretérito \textit{cuer}, passa a designar ``osso
já fora do corpo'', ``osso de canela'', ``tíbia'' e depois ``canudo'', ``tubo''.
Ocorre, com o sentido de ``canudo para fumar'', pela primeira vez num
texto português com o Pe. Fernão Cardim.} de folha de palma cheia desta
erva seca, e pondo"-lhe o fogo por uma parte põem a outra na boca, e
bebem o fumo; é uma das delícias, e mimos desta terra, e são todos os
naturais, e ainda os portugueses perdidos por ela, e têm por grande
vício estar todo o dia e noite deitados nas redes a beber fumo, e assim
se embebedam dela, como se fora vinho. 

\especie{Guaraquigynha}\footnote{ \textit{Guaraquigynha, guaraquim,
erva"-de"-bicho, erva"-moura, pimenta"-de"-rato} ou \textit{carachichú} 
é uma planta da família das Solanáceas (\textit{Solanum nigrum}, L.) ou
segundo outros autores trata"-se de uma planta da família das
Poligonáceas (\textit{Polygonum hydropiper}) que atinge 50\,cm de
altura, anual, de folhas lineares e pequenas flores brancas dispostas
em espigas. O caule, as folhas e a flor possuem sabor apimentado e são
utilizados no combate ao reumatismo, artrite e disenteria, sendo também
considerada abortiva.} Esta é a erva moura de Portugal, e
além de outras bondades que tem como a erva moura, tem somente que é
único remédio para lombrigas, e de ordinário quem as come logo as lança. 

\especie{Camará}\footnote{ \textit{Camará} ou \textit{cambará}, da
família das Verbenáceas (\textit{Lantana camara}, L.) ou da família das
Compostas (\textit{Eupatorium laevigatum}), é um arbusto com até 2\,m de
altura, de ramos lisos, folhas opostas e flores amareladas, cujas
folhas são consideradas úteis na preparação de unguentos. Existem várias
espécies, como o cambará"-branco (\textit{Lantana brasiliensis}), 
cambará"-de"-espinho (\textit{Lanatana camara}), cambará"-guaçu
(\textit{Vernonia polyanthes}), entre outros. Ocorre pela primeira vez
num texto português com Cardim. O termo tupi parece significar ``folha
pintada'', de \textit{caá=} ``folha'' + \textit{mbará} = ``pintada'' ou 
``variegada de várias cores''.} Esta erva se parece com silvas
de Portugal: coze"-se em água, e a dita água é único remédio para
sarnas, boubas, e feridas frescas, e quando as feridas se curam com as
folhas de figueira de que se disse no título das árvores, se lava a
ferida com a água desta erva, cuja flor é formosíssima, parece cravo
amarelo, e vermelho, almiscarado, e destas se fazem ramalhetes para os altares.

\especie{Aipo}\footnote{ \textit{Aipo}, da família das Umbeliferáceas
(\textit{Apium graveolens}, Linn., \textit{Apium australe} ou
\textit{Apium ranunculifolium}), adquirindo os nomes vulgares de
``aipo"-do"-Rio"-Grande'' ou ``aipo"-falso'', são ervas condimentares, de
pequenas flores brancas, também muito utilizadas no tratamento de
ferimentos de armas de fogo.} Esta erva é o próprio aipo
de Portugal, e tem todas as suas virtudes: acha"-se somente pelas
praias, principalmente no Rio de Janeiro, e por esta razão é mais
áspero, e não tem doce ao gosto, como o de Portugal: deve ser por causa
das marés. 

\especie{Malvaísco}\footnote{ \textit{Malvaísco}: da família das
Malváceas com várias espécies, como a malva"-branca (\textit{Althaea
officinalis}) originária da Europa, a malva"-rosa"-do"-campo 
(\textit{Pavonia hastata}) o malvaísco"-do"-sul (\textit{Sphaeralcea
cisplatina}) que produz umas flores rosadas e que tem aproveitamento
medicinal, anticatarral. Esta última espécie deve ser a que Cardim
menciona no seu texto.} Há grande abundância de malvaísco
nesta terra; tem os mesmos efeitos, tem umas flores do tamanho de um
tostão, de um vermelho gracioso, que parecem rosas de Portugal.

\especie{Caraguatá}\footnote{ \textit{Caraguatá, carauqatá, carautá,
crautá, crauá ou gravatá} é a designação comum a diversas plantas da
família das Bromeliáceas (\textit{Bromelia Karatas}, L.). Ocorre pela
primeira vez num texto português com Cardim. O termo tupi deve ser
formado de \textit{caá+ragua+ãtã} = ``erva de ponta fina'' ou ``folha de
ponta aguda''.} Este Caraguatá é certo gênero de cardos,
dão umas frutas de comprimento de um dedo, amarelas; cruas fazem
empolar os beiços; cozidas ou assadas não fazem mal; porém toda a
mulher prenhe que as come de ordinário morre logo.

 Há outros caraguatás que dão umas folhas como espadana muito comprida,
de duas ou três braças, e dão umas alcachofras como o naná, mas não são
de bom gosto. Estas folhas deitadas de molho dão um linho muito fino,
de que se faz todo gênero de cordas, e até linhas para cozer e pescar. 

\especie{Timbó}\footnote{ \textit{Timbó}, designação comum a várias
plantas da família das Sapindáceas (\textit{Paullinia pinnata}, L.) 
ou das Leguminosas, cuja seiva é tóxica, que contém o
alcaloide timboína, e que serve para atordoar os peixes e, por isso,
utilizada para a pesca. Este costume praticado pelos índios foi
transmitido aos portugueses e ainda hoje é praticado no interior do
Brasil. Ocorre pela primeira vez num texto português em 1560, numa
carta do Pe. José de Anchieta.} Timbó são umas
ervas maravilhosas, crescem do chão como cordões até o mais alto dos
arvoredos onde estão, e alguns vão sempre arrimados à árvore como era;
são muito rijos, e servem de atilhos, e alguns há que tão grossos como a
perna de homem, e por mais que os torçam não há quebrarem; a casca
destes é fina peçonha, e serve de barbasco para os peixes, e é tão
forte que nos rios onde se deita não fica peixe vivo até onde chega
com sua virtude, e destes há muitas castas, e proveitosas assim para
atilhos como para matar os peixes. 

 Outras ervas há que também servem para medicinas, como são
serralhas,\footnote{ \textit{Serralha}, nome comum a várias ervas da
família das Compostas, entre as quais a serralha"-lisa (\textit{Sonchus
oleraceus}) é uma erva anual, de folhas denteadas e flores amarelas,
cujas folhas novas, aromáticas e de leve sabor amargo, são consumidas
em saladas ou como espinafre, cuja raiz é medicinal e o látex é
utilizado contra inflamações dos olhos, e ainda a serralha"-de"-espinho
(\textit{S. asper}) que tem idêntica utilização.} 
beldroegas,\footnote{ \textit{Beldroegas} são plantas das famílias das
Urticáceas ou das Portulacáceas, com várias espécies e cujas folhas são
comestíveis, cruas ou cozidas, tendo algumas propriedades emolientes,
como a beldroega"-de"-Cuba (\textit{Tainum racemossum}) originária de
Cuba.} bredos,\footnote{ \textit{Bredos}, da família das
Amarantáceas (\textit{Amarantus hypocondriacus}), é uma erva grande, de
folhas avermelhadas, carnosas e flores roxas, pequenas e numerosas, tem
várias espécies como a bredo"-fedorenta (\textit{Cleome polygama}) da
família das Caparidáceas que é aromática e que tem fins medicinais, o
bredo"-verdadeiro (\textit{Amarantus graecizans}) cujas folhas depois de
secas são consideradas diuréticas e que é originária da Europa.} 
almeirões,\footnote{ \textit{Almeirão}, da família das Compostas
(\textit{Chicorium intybus}), é uma erva perene, de raiz oblonga, flores
azuis, grandes, às vezes brancas ou róseas que são comestíveis cruas ou
cozidas. É originária da Europa.} avencas,\footnote{ \textit{Avenca} 
é a designação comum a várias plantas criptogâmicas
da família das Polipodiáceas, dos gêneros \textit{Adiantum} e 
\textit{Asplenium} com longos pecíolos finos, negros, folhas pequenas e
recortadas.} e de tudo há grande abundância, ainda que não têm estas
ervas a perfeição das de Espanha, nem faltam amoras de silva brancas, e
pretas como as de Portugal,\footnote{ A \textit{amora}, pequena drupa
vermelho"-escuro comestível, da família das Rosáceas, das quais existem
no Brasil as espécies amora"-brava (\textit{Rubus imperialis}) e a
amora"-vermelha (\textit{R. rosaefolius}).} e muito bom 
perrexil\footnote{ \textit{Perrexil} é o nome vulgar do \textit{Chrithmum
maritimum}, da família das Umbelíferas, também conhecido por
\textit{funcho"-marítimo.}} pelas praias, de que se faz conserva muito
boa, nem falta macela.\footnote{ \textit{Macela} é a designação mais
usual para a camomila, asterácea cujas flores servem para infusões.} 

\paragraph{Das ervas cheirosas}

Nesta terra há muitos mentrastos,\footnote{ \textit{Mentrasto,
menstrate, mentraste} ou \textit{mentastro} são as designações
atribuídas a uma planta da família das Labiadas (\textit{Mentha
rottundifolia}) que é uma espécie de hortelã silvestre, com
propriedades medicinais.} principalmente em Piratininga: não cheiram
tão bem como os de Portugal; também há umas malvas\footnote{ \textit{Malva} 
é o nome vulgar de várias herbáceas emolientes, da
família das Malváceas, entre as quais se destaca a \textit{Malva
silvestris} e a Malva"-rosa (\textit{Althaea rosea).}} francesas de umas
flores roxas, e graciosas que servem de ramalhetes. Muitos Lírios, não
são tão finos, nem tão roxos como os do Reino, e alguns se acham brancos.

\especie{Erva que dorme}\footnote{ \textit{Erva que dorme, dormideira}
ou \textit{papoila}, da família das Papaveráceas (\textit{Papayer
somniferum}, L.) ou da família das Leguminosas"-Mimosáceas
(\textit{Mimosa pudica}), é uma planta que atinge 1\,m de altura, com
ramos armados de espinhos esparsos, folhas compostas e pequenas flores
lilases, cuja casca é utilizada como vermífugo e as folhas são
consideradas venenosas.} Esta erva se dá cá na Primavera, e
parece"-se com os Maios de Portugal,\footnote{ Cardim refere"-se a uma
espécie de lírio campestre, de flores amarelas.} e assim como eles se
murcha e dorme em se pondo o sol, e em nascendo torna a abrir e mostrar
sua formosura. O cheiro é algum tanto farto. Também há outra árvore que
dorme da mesma maneira, e dá umas flores graciosas, mas não cheiram muito.

\especie{Erva viva}\footnote{ \textit{Erva"-viva, sensitiva,
malícia"-de"-mulher} são designações atribuídas a uma planta da família 
das Leguminosas, subfamília das Mimosáceas (\textit{Mimosa
sp.).}} Estas ervas são de boa altura, e dão ramos, umas
folhas farpadas de um verde gracioso; chamam"-se erva"-viva, porque são
tão vivas e sentidas que em lhes tocando com a mão, ou qualquer outra
cousa, logo se engelham, murcham e encolhem como se as agravaram muito,
e daí a pouco tornam em sua perfeição tantas vezes lhes tocam, tantas
tornam a murchar"-se, e tornam em seu ser como dantes.

 Outras muitas ervas há, como orégãos, e poejos,\footnote{ \textit{Orégão} 
e \textit{poejo} são plantas da família das Labiadas
(\textit{Mentha piperita, L.}) e (\textit{M. Pulegium}, L.) utilizadas
usualmente na culinária como cheiros.} e outras muitas flores várias,
porém parece que este clima, ou pelas muitas águas, ou por causa do
sol, não influi nas ervas cheiro, antes parece que lho tira.

\paragraph{Das canas}

Nesta terra há muitas espécies de canas e tacoara;\footnote{ \textit{Tacoará} 
ou \textit{taquará}, da família das Gramíneas
(\textit{Chusquea gaudichaudii}, Kunth), também conhecida pela
designação de taboca ou bambu. Os indígenas utilizavam as taquaras,
particularmente para a confecção de suas flechas. O termo tupi
explica"-se por \textit{tâ"-quara} = ``haste furada'' ou ``cheia de buracos''
e ocorre a primeira vez num texto português com Cardim.} há de grossura
de uma coxa de um homem, outras que têm uns canudos de comprimento de
uma braça, outras de que fazem flechas e são estimadas; outras tão
compridas que têm três ou quatro lanças de comprimento; dão"-se estas
canas por entre os arvoredos, e assim como há muitas, assim há muitos e
compridos canaviais de muitas léguas, e como estão entre as árvores vão
buscar o sol, e por isso são tão compridas.

\paragraph{Dos peixes que há na água salgada}

\especie{Peixe boi}\footnote{ \textit{Peixe boi} é um cetáceo da
família dos Triquequídeos (\textit{Trichechus inunguis}, Natterer)
mamífero aquático que ocorre, ainda hoje, na Amazônia. O termo tupi
para este peixe de água salgada é \textit{Guaraguá}, que se
traduz por \textit{guara"-guara} = ``come"-come'', ``comilão'', ou ainda
por \textit{yguá"-ri"-guá} = ``morador em enseadas'', o que é um costume
deste cetáceo. A designação tupi ocorre pela primeira vez numa carta 
de 1560, do Pe. José de Anchieta, já que Cardim utiliza o termo
de \textit{Peixe boi.} A descrição cardiniana deste peixe está
rigorosa, mesmo em relação ao peso que pode chegar a atingir cerca de
1200 a 1500\,kg, com 4\,m de comprimento. Tinha o seu habitat nas águas
quentes da costa norte e nordeste até às imediações de Ilhéus, mas já
era raro no litoral do Espírito Santo e constituía um dos principais
recursos alimentares dos Tupis da orla marítima.} Este
peixe é nestas partes real, e estimado sobre todos os demais peixes, e
para se comer muito sadio, e de muito bom gosto, ora seja salgado, ora
fresco; e mais parece carne de vaca que peixe. Já houve alguns
escrúpulos por se comer em dias de peixe; a carne é toda de febras,
como a de vaca, e assim se faz em taçalhos e chacina; e cura"-se ao
fumeiro como porco ou vaca, e no gosto se se coze com couves, ou outras
ervas, sabe à vaca, e concertada com adubos sabe a carneiro, e assada
parece, no cheiro, a gordura de porco, e também tem toucinho.

 Este peixe nas feições parece animal terrestre, e principalmente boi: a
cabeça é toda de boi com couro, e cabelos, orelhas, olhos, e língua; os
olhos são muito pequenos em extremo para o corpo que tem; fecha"-os, e
abre"-os, quando quer, o que não têm os outros peixes; sobre as ventas
tem dois courinhos com que as fecha, e por elas resfolega; e não pode
estar muito tempo debaixo de água sem resfolegar; não tem mais
barbatana que o rabo, o qual é todo redondo e fechado; o corpo é de
grande grandura, todo cheio de cabelos ruivos; tem dois braços de
comprimento de um côvado com suas mãos redondas como pás, e nelas tem
cinco dedos pegados todos uns com os outros, e cada um tem sua unha
como humana; debaixo destes braços têm as fêmeas duas mamas com que
criam seus filhos, e não parem mais que um; o interior deste peixe, e
intestinos são propriamente como de boi, com fígados, bofes etc.

 Na cabeça sobre os olhos junto aos miolos tem duas pedras de bom
tamanho, alvas, e pesadas; são de muita estima, e único remédio para
dor de pedra, porque feita em pó e bebida em vinho, ou água, faz deitar
a pedra, como aconteceu que dando"-a a uma pessoa, deixando outras
muitas experiências, antes de uma hora botou uma pedra como uma
amêndoa, e ficou sã, estando dantes para morrer. Os ossos deste peixe
são todos maciços, e brancos como marfim; faz"-se dele muita manteiga,
e tiram"-lhe duas banhas, como de porco; e o mais da manteiga; é muito
gostosa, e para cozinhar e frigir peixe, para a candeia serve muito, e
também para mezinhas, como a do porco; é branca e cheirosa; nem tem
cheiro de peixe. Este peixe se toma com arpoeiras; e acham"-se nos rios
salgados junto de água doce: comem uma certa erva que nasce pelas
bordas, e dentro dos rios, e onde há esta erva se matam, ou junto de
olhos de água doce, a qual somente bebem; são muito grandes; e alguns
pesam dez, e outros quinze quintais, e já se matou peixe que cem
homens o não puderam tirar fora de água, e nela o desfizeram.

\especie{Bigjuipirá}\footnote{ \textit{Bigjupirá, bijupirá} ou
\textit{bejupirá}, da família dos Racicentrídeos (\textit{Rachycentrus
canadus}, L.), era também designado por ``peixe"-rei''. O termo tupi
parece ser formado por \textit{mbeyú"-pirá} = ``peixe de bolo'', por causa
da qualidade da sua carne. Ocorre pela primeira vez num texto português
com Cardim.} Este peixe Bigjuipirá se parece com solho de
Portugal, e assim é cá estimado, e tido por peixe real; é muito sadio,
gordo, e de bom gosto; há infinidade deles, e algumas ovas têm em
grosso um palmo de testa. Tomam"-se estes peixes no mar alto à linha
com anzol; o comprimento será de seis ou sete palmos, o corpo é
redondo, preto pelas costas, e branco pela barriga.

\especie{Olho de boi}\footnote{ \textit{Olho"-de"-boi}, peixe de água
salgada da família dos Carangídeos (\textit{Seriola lalandei}, Cuv.\&
Val.), atinge grandes dimensões, até 2\,m de comprimento e 50\,kg de peso.
Ocorre no oceano Atlântico, das Antilhas até ao Uruguai, em locais
pedregosos. O termo tupi parece ser \textit{Tapyrsiçá} ou
\textit{tapireçá}, que quer dizer ``olho de boi'', de \textit{tapyra} = 
``boi'' + \textit{eçá} = ``olho''. Este termo tupi ocorre apenas com
Gabriel Soares de Sousa, em 1587, na \textit{Notícia do Brasil}, que o
denomina de ``olho"-de"-boi'', tal como Cardim.} 
Parece"-se este peixe com os atuns de Espanha, assim no tamanho como
nas feições, assim interiores como exteriores; é muito gordo, tem as
vezes entre folha, e folha gordura de grossura de um tostão:
tiram"-se"-lhe lombos e ventrechas como aos atuns, e deles se faz muita e
boa manteiga, e lhe tiram banhas com a um porco; é peixe estimado, e de
bom gosto, bem merece o nome de peixe boi assim na formosura, como
grandura; os olhos são propriamente como de boi, e por esta razão tem este nome.

\especie{Camurupig}\footnote{ \textit{Camuripig, camurupi} ou
\textit{camurupim}, da família dos Megalopídeos (\textit{Megalops
trissoides}, BI.\& Schn.). Ocorre pela primeira vez num texto português
em 1576, na \textit{História da Província Santa Cruz}, de Pêro de
Magalhães de Gândavo. É o \textit{pirapema} do litoral do norte do Brasil.
O termo tupi é difícil de explicar até porque aparece com várias
terminologias.} Este peixe também é um dos reais e
estimados nestas partes: a carne é toda de febras em folha, cheia de
gordura e manteiga, e de bom gosto; tem muita espinha por todo o corpo
e é perigoso ao comer. Tem uma barbatana no lombo que sempre traz
levantada para cima, de dois, três palmos de comprimento; é peixe
comprido de até doze e treze palmos, e de boa grossura, e tem bem que
fazer dois homens em levantar alguns deles; tomam"-se com arpões; há
muitos, e faz"-se deles muita manteiga.

\especie{Peixe selvagem}\footnote{ \textit{Peixe selvagem}, da família
dos Hemulídeos (\textit{Conodon nobilis}, L.). O termo tupi
\textit{pirambá} significa ``peixe"-roncador'' o que se coaduna com a
descrição cardiniana e ocorre pela primeira vez num texto português com
Cardim.} Este peixe selvagem, aqui os índios chamam
\textit{Pirambá}, sc.~peixe que ronca; a razão é porque onde andam
logo se ouvem roncar, são de boa grandura até oito e nove palmos; a 
carne é de bom gosto, e são estimados; têm na boca duas pedras de
largura de uma mão, rijas em grande extremo, com elas partem os búzios
de que se sustentam; as pedras estimam os índios, e as trazem ao
pescoço como jóias.

 Há outros peixes de várias espécies que não há em Espanha, e comumente
de bom gosto, e sadios. Dos de Portugal também por cá há muitos, sc.
tainhas em grande multidão, e tem"-se achado que a tainha fresca posta a
carne dela em mordedura de cobra é outro unicórnio.\footnote{ Utilizar 
o termo \textit{unicórnio} deve ser encarado como
``bálsamo''. Vide nota supra.} Não faltam garoupas, peixe agulha,
pescada, mas são raras; sardinhas com as de Espanha se acham em alguns
tempos no Rio de Janeiro, e mais partes do sul; cibas, e arraias; estas
arrais algumas delas têm na boca dois ossos tão rijos que quebram os búzios com eles. 

 Todo este peixe é sadio cá nestas partes que se come sobre leite, e
sobre carne, e toda uma quaresma, e de ordinário sem azeite nem
vinagre, e não causa sarna nem outras enfermidades como na Europa,
antes se dá aos enfermos de cama, ainda que tenham, ou estejam muito no cabo.

\especie{Balêa}\footnote{ \textit{Baleia} é o nome comum aos grandes
cetáceos da família dos Baleanídeos. Existem cerca de sete espécies na
costa brasileira, como a Baleia"-azul (\textit{Balaenoptera musculus}), 
Baleia"-lisa (\textit{Eubalaena australis}), Baleia"-mink 
(\textit{Balaenoptera acutorostrata}) e a \textit{Megaptera nodosa}, 
que era a que existia em maior quantidade nos recortes do litoral
brasílico. O termo tupi para este animal era \textit{pirapuã}, que
significa ``peixe que empina''. Este termo ocorre pela primeira vez num
texto português em 1587, com Gabriel Soares de Sousa, na
\textit{Notícia do Brasil.}} Por esta costa ser cheia de
muitas baías, enseadas e esteiros acodem grande multidão de baleias a
estes recôncavos, principalmente de Maio até Setembro, em que parem, e
criam seus filhos, e também porque acodem ao muito tempo que nestes
tempos é nestes remansos; são tantas as vezes que se vêm quarenta, e
cinquenta juntas, querem dizer que elas deitam o âmbar que acham no
mar, e de que também se sustentam, e por isso se acha algum nesta
costa; outros dizem que o mesmo mar o deita nas praias com as grandes
tempestades e comumente se acha depois de alguma grande. Todos os
animais comem este âmbar, e é necessário grande diligência depois das
tempestades para que o não achem comido. É muito perigoso navegar em
barcos pequenos por esta costa, porque, além de outros perigos, as
baleias soçobram muitos, se ouvem tanger, assim se alvoraçam como se
foram cavalos quando ouvem tambor, e arremetem como leões, dão muitas à
costa e delas se fazem muito azeite. Tem o toutiço furado, e por ele
resfolegam, e juntamente botam grande soma de água, e assim a espalham
pelo ar como se fosse chuveiro.

\especie{Espadarte}\footnote{ \textit{Espadarte}, peixe da família dos
Xifídeos (\textit{Xiphias glaudius}, Linn.). Os índios denominavam"-no
de \textit{pirapicu}, que significa ``peixe comprido'', o que condiz
com as características deste peixe. O termo tupi ocorre pela primeira
vez num texto português em 1587, com Gabriel Soares de Sousa, na
\textit{Notícia do Brasil.}} Destes peixes há grande
multidão, são grandes, e ferozes, porque têm uma tromba como espada,
toda cheia de dentes ao redor, muito agudos, tão grandes como de cão,
os maiores, são de largura de uma mão travessa, ou mais, o comprimento
é segundo a grandura do peixe; algumas trombas, ou espadas destas são
de oito e dez palmos; com estas trombas fazem cruel guerra às baleias,
porque alevantando"-a para cima, dando tantas pancadas em elas, e tão a
miúde que é cousa de espanto, acodem ao sangue os tubarões, e as chupam
de maneira até que morrem, e desta maneira se acham muitas mortas, em
pedaços. Também com esta tromba pescam os peixes de que se sustentam.
Os índios usam destas trombas quando são pequenos para açoutarem os
filhos, e lhes meterem medo quando lhes são desobedientes.

\especie{Tartaruga}\footnote{ \textit{Tartaruga} é o nome comum aos
quelônios marinhos. Cardim coloca"-a como peixe, mas é um réptil.
Existem no Brasil várias espécies, como a Tartaruga"-de"-couro
(\textit{Dermochelys coriacea}) que chega a atingir cerca de 2\,m de
comprimento e que não dispõe de uma carapaça, mas coberta por uma
espécie de couro; a Tartaruga"-de"-pente (\textit{Eretmochelys
imbricata}) que é muito caçada por causa da carapaça que é preciosa; a
Tartaruga"-verde (\textit{Cahelonia mydas}) que chega a atingir cerca
de 1\,m de comprimento e a Tartaruga"-da"-Amazônia (\textit{Podocnemis
expansa).} Os índios apresentam vários termos para designarem estes
animais, como \textit{jurará} que ocorre pela primeira vez num texto
português em 1624, com Simão Estácio da Silveira, na \textit{Relação do
Maranhão;} \textit{jurarapeba}, que ocorre em 1631, com Frei Cristóvão
de Lisboa, na \textit{História dos Animais e Árvores do Maranhão;} 
\textit{tracajá}, que só ocorre em 1777, com Francisco Xavier Ribeiro
de Sampaio, na \textit{Relação Geográfica Histórica do Rio Branco da
América Portuguesa} e ainda \textit{matamatá} que ocorre na mesma obra
de Frei Cristóvão de Lisboa.} Há nesta costa muitas
tartarugas; tomam"-se muitas, de que se fazem cofres, caixas de hóstias,
copos etc. Estas tartarugas põem ovos nas praias, e põem logo duzentos
e trezentos, são tamanhos como de galinhas, muito alvos, e redondos
como pélas;\footnote{ O autor utiliza no texto \textit{péla} no sentido
de bola. Na Europa, entre os séculos \textsc{xv} e \textsc{xviii}, desenvolveu"-se a
prática do ``jogo da péla'', que era muito semelhante ao tênis atual. No
México e Guatemala os índios desenvolveram também um jogo com o mesmo
nome, em que se procurava representar o percurso do sol pelo céu, o
qual era praticado com uma bola maciça, de couro, impulsionada com as
ancas, os cotovelos e os joelhos.} escondem estes ovos debaixo da
areia, e como tiram os filhos logo começam de ir para água donde se
criam. Os ovos também se comem, têm esta propriedade que ainda se
cozam, ou assem sempre a clara fica mole: os intestinos são como de
porco, e têm ventas por onde respiram. Tem outra particularidade que
pondo"-lhe o focinho para a terra logo viram para o mar, nem podem
estar doutra maneira. São algumas tão grandes que se fazem das conchas
inteiras adargas; e uma se matou nesta costa tão grande que vinte
homens a não podiam levantar do chão, nem dar"-lhe vento.

\especie{Tubarões}\footnote{ \textit{Tubarões} existem várias
espécies, que são os maiores da ordem dos Seláceos, no território
brasileiro. Entre essas espécies salientam"-se o Tubarão"-azul
(\textit{Prionace glauca}) que chega a alcançar cerca de 3,5\,m de
comprimento, o Tubarão"-baleia (\textit{Rhincodon typus}) que é de
maior porte, chegando aos 18\,m de comprimento, o Tubarão"-martelo
(\textit{Sphyrna zygaena, S. diplana} ou \textit{S. tudes}) que
atinge os 5\,m de comprimento e que apresenta duas expansões laterais
na cabeça e o Tubarão"-seis"-fendas (\textit{Hexanchus griseus}) que é
um tubarão que apresenta seis pares de fendas branquiais. O termo tupi
para este animal é \textit{iperu} que significa ``o que dilacera'' e
que ocorre, com a variante de \textit{uperu}, pela primeira vez num
texto português em 1587, com Gabriel Soares de Sousa, na
\textit{Notícia do Brasil.}} Há muitos gêneros de tubarões
nesta costa: acham"-se nelas seis, ou sete espécies deles: é peixe muito
cruel e feroz, e matam a muitas pessoas, principalmente aos que nadam.
Os rios estão cheios deles, são tão cruéis que já aconteceu correr um
após de um índio que ia numa jangada, e pô"-lo em tanto aperto que
saltando o moço em terra o tubarão saltou juntamente com ele, e
cuidando que o apanhava ficou em seco onde o mataram. No mar alto onde
também há muitos se tomam com laço, e arpões por serem muito gulosos,
sôfregos, e amigos de carne e são tão comilões que se lhes acham na
barriga couros, pedaços de pano, camisas, e ceroulas, que caem aos
navegantes; andam de ordinário acompanhados de uns peixes muito
galantes, formosos de várias cores que se chamam romeiros;\footnote{ Cardim 
refere"-se ao \textit{Romeiro} ou \textit{Romeirinho} que é o
nome vulgar de peixes do gênero \textit{Naucrates}, também denominados
de ``peixe"-piolho''.} faz"-se deles muito azeite, e dos dentes usam os
índios em suas flechas por serem muito agudos, cruéis, e peçonhentos, e
raramente saram as feridas, ou com dificuldade.

\especie{Peixe voador}\footnote{ \textit{Peixe voador}, da família dos
Cafelacantídeos (\textit{Cephalacanthus volitans}, L.) ou também da
família dos Exocetídeos (\textit{Exocoertus volitans}), é um peixe
teleósteo, ateriniforme, cujas nadadeiras peitorais e ventrais são
amplas, permitindo"-lhe longas planagens acima da superfície da água. O
termo tupi para este peixe é \textit{pirabebe}, de \textit{pirá} = 
``peixe'' + \textit{bêbê} = ``volante'', ``que voa'' e ocorre pela primeira
vez num texto português em 1631, com Frei Cristóvão de Lisboa, in
\textit{op. cit.} Atualmente é designado também por
\textit{coió.}} Estes peixes são de ordinário de um palmo,
ou pouco mais de comprimento; têm os olhos muito formosos, galantes de
certas pinturas que lhes dão muita graça, e parecem pedras preciosas; a
cabeça também é muito formosa. Têm asas como de morcegos, mas muito
prateadas, são muito perseguidos dos outros peixes, e para escaparem
voam em bandos como de estorninhos, ou pardais, mas não voam muito
alto. Também são bons para comer, e quando voam alegram os mareantes, e
muitas vezes caem dentro das naus, e entram pelas janelas dos
camarotes.

\especie{Botos} e \textit{Tuninhas}\footnote{ \textit{Botos} e 
\textit{Toninhas} são mamíferos cetáceos de água salgada e marinha da
família dos Platanistídeos (boto"-branco e toninha) e Delfinídeos
(golfinho e tucuxi) da ordem dos Odontocetos. Existem várias espécies
no território brasileiro, como o Boto"-cinzento (\textit{Grampus
griseus}) usualmente conhecido como o ``Golfinho"-de"-riso'', o
Boto"-cor"-de"-rosa (\textit{Inia geoffrensis}) que ocorre na bacia do
rio Amazonas e que tem uma coloração do corpo rosada e o Boto"-preto (\textit{Sotalia fluviatilis).} Este último é designado em tupi de
\textit{tucuxi}, \textit{jaguara} ou ``peixe"-cão''.} Destes
peixes há grande multidão como em Europa. 

\especie{Linguados}\footnote{ \textit{Linguado} é um teleósteo,
pleuronectiforme (\textit{Paralichthys brasiliensis}) da família dos
Botídeos, que chega a atingir cerca de 1\,m de comprimento e
12\,kg de peso. Ocorre da Bahia para o Sul, sendo comum no
litoral do Rio de Janeiro.} e \textit{Salmonetes}\footnote{ \textit{Salmonete} 
é a designação comum a dois peixes marinhos, da
família dos Mulídeos, o \textit{Pseudopeneus maculatus} e o
\textit{Mullus surmuletus}, muito apreciados e munidos de barbilhões na
mandíbula inferior.} Também se acham nesta costa
salmonetes, mas são raros, e não tão estimados, nem de tão bom gosto
como os da Europa; os linguados de cá são raros; têm propriedade que
quando se hão de cozer, ou assar os açoutam, e quando mais açoutam lhes
dão tanto mais tesos ficam, e melhores para comer, e se os não açoutam
não prestam e ficam moles.

\paragraph{Dos peixes peçonhentos}

Assim como esta terra do Brasil há muitas cobras, e bichos
peçonhentos de que se dirá adiante, assim também há muitos peixes muito
peçonhentos.

\especie{Peixe sapo}, pela língua \textit{Guamayaçu}\footnote{ \textit{Peixe sapo}, 
hoje designado por \textit{baiacu}, é da família
dos Tetrodontídeos, do qual existem várias espécies, como o
\textit{Baiacu"-de"-espinho (Chilo mycterus spinosus}, L.) da família dos
Diodontídeos. O termo que Cardim diz ser na língua tupi
\textit{guamayaçu} é difícil de identificar.} É peixe
pequeno, de comprimento de um palmo, pintado, tem olhos formosos; em o
tirando da água ronca muito e trinca muito os anzóis, e em o tirando da
água incha muito. Toda a peçonha têm na pele, e tirando"-lha, come"-se,
porém comendo"-se com a pele mata. Aconteceu que um moço comeu e morreu
quase subitamente; disse o pai: hei"-de comer o peixe que matou meu
filho, e comendo dele também morreu logo. É grande mezinha para os
ratos, porque os que o comem logo morrem.

 Há outro peixe sapo da própria feição que o atrás, mas tem muitos e
cruéis espinhos, como ouriço, ronca e incha tirando"-o da água; a pele
também mata, \textit{maxime} os espinhos, por serem muito venenosos; esfolado se
come, e é bom para câmaras de sangue.

 Há outro peixe sapo que na língua se chama 
 \textit{Itaoca}\footnote{ \textit{Itaoca} ou \textit{taoca} é outra espécie de peixe"-sapo
(\textit{Lactophrys tricornis}, L.) da família dos Ostraciontídeos, que
ocorre pela primeira vez, com este termo tupi, num texto português com
Cardim.}; tem três quinas em o corpo que todo ele parece punhal; é
formoso, tem os olhos esbugalhados, e esfolado se come; consiste a
peçonha na pele, fígados, tripas, e ossos, e qualquer animal que o come
logo morre.

 Há outro que se chama \textit{Carapeaçaba},\footnote{ \textit{Carapeaçaba} 
ou \textit{carapiaçaba} é uma espécie de Sargo
do rio, que ocorre a primeira vez num texto português com Fernão
Cardim. O termo tupi é difícil de identificar.} de cor gateado, pardo,
preto, e amarelo; é bom peixe e dá"-se aos doentes; os fígados, e tripas
têm tão forte peçonha que a todo o animal mata; e por esta causa os
naturais em o tirando deitam as tripas e fígado no mar.

\especie{Purá}\footnote{ \textit{Purá, puraquê} ou
\textit{poraquê} são as designações para o \textit{peixe"-elétrico}
da família dos Electroforídeos (\textit{Electrophorus electricus}, L.).
O termo tupi \textit{purá} ocorre pela primeira vez num texto
português com Cardim e \textit{poraquê} significa ``fazer dormir'',
``entorpecer'', o que está de acordo com os efeitos do peixe"-elétrico
cuja descarga elétrica adormece quem o apanha.} Este peixe
se parece com arraia: tem tal virtude que quem quer que o toca logo
fica tremendo, e tocando"-lhe com algum pau, ou com outra qualquer
cousa, logo adormece o que lhe põem, e enquanto lhe tem o pau posto em
cima fica o braço com que o toma o pau adormecido, e adormentado.
Tomam"-se com redes de pé, e se se tomam com redes de mão todo o corpo
faz tremer, e pasmar com a dor, mas morto come"-se, e não tem peçonha.

\especie{Caramuru}\footnote{ \textit{Caramuru} ou \textit{moreia} é a
designação comum a peixes ósseos, marinhos, serpentiformes,
anguiliformes, da família dos Muraenídeos (\textit{Gymnothorax
moringua}). O corpo é muito musculoso, roliço, sem nadadeiras peitorais
e possui dentes fortes, na base dos quais se situam glândulas de
veneno. O termo tupi ocorre pela primeira vez num texto português com
Fernão Cardim. \textit{Caramuru}, em alusão ao peixe do mesmo nome, foi
o apelido dado pelos índios Tupinambás ao português Diogo Álvares (Viana
do Castelo?--Salvador da Bahia 1557) que escapou a nado de um
naufrágio nas costas da Baía de Todos os Santos, em 1510, foi o
primeiro lusitano a fixar"-se na Bahia. Passou a viver entre os índios
que lhe deram essa alcunha. Dominando a língua e os costumes indígenas,
auxiliou Tomé de Sousa e os Jesuítas na fundação dos primeiros
estabelecimentos e na aproximação com os índios. Casou com a filha de
um dos chefes Tupinambás, tendo deixado quatro filhos. A lenda segundo a
qual Diogo Álvares foi apelidado de \textit{Caramuru} por haver
maravilhado os indígenas com um tiro de espingarda parece ter sido
divulgada por Frei José de Santa Rita Durão, no poema épico com a mesma
designação, \textit{Caramuru}, publicado em Lisboa, em 1781.} 
Estes peixes são como as amoreias de Portugal, de comprimento de dez,
e quinze palmos; são muito gordos, e assados sabem a leitão; estes têm
estranha dentadura, e há muitos homens aleijados de suas mordeduras, de
lhe apodrecerem as mãos ou pernas onde foram mordidos; têm por todo o
corpo muitos espinhos, e dizem que os naturais que têm ajuntamento com
as cobras, porque os acham muitas vezes com elas enroscados, e nas
praias esperando as ditas moreias.

\especie{Amoreati}\footnote{ \textit{Amoreati} ou \textit{moreiatim} 
é o niquim, niquim"-de"-areia ou niquim"-do"-mar da família dos
Batracoidídeos (\textit{Thalassophyne branniere}, Starks). Este animal,
tal como descreve o Pe. Fernão Cardim, pode esconder"-se debaixo da
areia e ferir as pessoas com o acúleo dorsal, que se comunica com uma
glândula de veneno. O termo tupi ocorre a primeira vez num texto
português com Cardim.} Este peixe se parece com o peixe
sapo; está cheio de espinhos, e mete"-se debaixo da areia nas praias, e
picam por debaixo do pé ou mão que lhes toca, e não tem outra cura
senão fogo.

\especie{Guamaiacucurub}\footnote{ \textit{Guamaiacucurub} ou
\textit{baiacu"-curuba} é uma espécie dos Tetrodontídeos, mas difícil
de determinar. O termo \textit{baiacu} designa pequeno peixe venenoso,
que, quando tem o ventre atritado, incha"-se todo, chegando a rebentar,
desprendendo um fel que é venenoso. O termo \textit{curubá} designa,
por sua vez, ``espinha do rosto'', ``bolha da pele'', ``sarna'', ``bolota'' ou
``caroço'', o que vai ao encontro da explicação de Cardim.} 
Estes peixes são redondos, e do tamanho dos bugalhos de Espanha, e são
muito peçonhentos. O corpo tem cheio de verrugas, e por isso se chama
curub,\footnote{ \textit{Curub} ou \textit{Curubá}: vide nota supra.} 
sc.~na língua verruga.

\especie{Terepomonga}\footnote{ \textit{Terepomonga}: pela descrição
do Pe. Fernão Cardim parece tratar"-se da Sanguessuga, que é um verme da
família dos Hirudinídeos. São animais anelídeos, hirudíneos, marinhos,
dulcícolas ou terrestres, que não apresentam tentáculos, parapódios ou
cerdas, sendo frequentemente ectoparasitas de vertebrados. Existem
cerca de 500 espécies. Ocorre pela primeira vez num texto português com
Cardim. O adjetivo tupi \textit{pomong} = ``pegajoso'', ``viscoso'', ``que
pega'' ou ``gruda'' coincide com a descrição cardiniana.} É uma
cobra que anda no mar; o seu modo de viver é deixar"-se estar muito
queda e qualquer cousa viva que lhe toca nela tão fortemente apegada,
que de nenhuma maneira se pode bulir, e desta maneira come, e se
sustenta; algumas vezes sai fora do mar, e torna"-se muito pequena, e
tanto que a tocam, pega, e se vão com a outra mão para desapegarem
ficam também pegados por ela, e depois faz"-se tão grossa como um bom
tirante, e assim leva a pessoa para o mar e a come; e por pegar muito
se chama terepomonga, sc.~cousa que pega. 

 Finalmente, há muitas espécies de peixes mui venenosos no salgado que
tem veemente peçonha, que de ordinário não escapa quem os come, ou toca.

\paragraph[Homens marinhos, e monstros do mar]{Homens marinhos, 
e monstros do mar\protect\footnote{ O Pe. Fernão
Cardim, tal como muitos outros escritores quinhentistas e seiscentistas
escreveram sobre os homens marinhos e os monstros do mar, inserindo"-se
no ciclo de ideias que produziu os tritões, as sereias, as mães"-de"-água
e outros seres fantásticos. A eles se referem Pêro de Magalhães de Gândavo,
Gabriel Soares de Sousa, Frei Vicente do Salvador, entre outros, que os
descrevem de uma forma muito semelhante à de Cardim. Esta antropologia
fantasista já tinha tradição na Península Ibérica, quer através dos
textos de Solino e Plínio, quer através das \textit{Etimologias} de
Santo Isidoro de Sevilha, muito frequentes nas bibliotecas da época
medieval. Qualquer um destes autores dever"-se"-ia querer referir ao
\textit{leão"-marinho} e \textit{lobo"-do"-mar (Otaria jubata}, Forst.)
que é um carnívoro pinípede.}}

Estes homens marinhos se chamam na língua \textit{Igpupiara}\footnote{ \textit{Igpupiara} 
ou \textit{ipupiara} é um termo tupi
formado por vários nomes, nomeadamente \textit{y} = ``água'' +
\textit{pypyara} = ``de dentro'', ``do íntimo'', ou seja, ``o que é de
dentro de água'', ``o que vive no fundo da água'', ``o aquático''. Este
nome tupi ocorre a primeira vez, em 1560, numa carta em latim do Pe.
José de Anchieta, e em português, em 1576, com Pêro de Magalhães de Gândavo,
na \textit{História da Província Santa Cruz}, em que conta a
história do monstro marinho que se matou na capitania de São Vicente,
em 1564, que ``[\ldots{}] os índios da terra lhe chamam em sua língua
Hipupiara, que quer dizer demônio de água''.}; têm"-lhe os naturais tão
grande medo que só de cuidarem nele morrem muitos, e nenhum que o vê
escapa; alguns morreram já, e perguntando"-lhes a causa, diziam que
tinham visto este monstro; parecem"-se com homens propriamente de boa
estatura, mas têm os olhos muito encovados. As fêmeas parecem mulheres,
têm os cabelos compridos, e são formosas; acham"-se estes monstros nas
barras dos rios doces. Em Jagoarigpe\footnote{ \textit{Jaguarigpe} é
um rio da região da Bahia. Nesta região surgiu a Santidade do
Jaguaripe, entre os anos de 1580 e 1585, um movimento de caráter
messiânico do Recôncavo Baiano, que reuniu um número considerável de
índios, entre os quais alguns já batizados e que viviam entre os
padres da Companhia. O seu líder encarnava as características de um
autêntico caraíba, de acordo com a tradição ameríndia, e a quem foi
atribuído o nome cristão de António, tinha vivido entre os inacianos no
aldeamento da ilha de Tinharé, de onde fugiu para dirigir os índios, e
que se intitulava de ``papa'' do seu próprio movimento. A sua mensagem
principal era a busca da Terra sem Mal acreditando que o fim do mundo
estava próximo e que no dia do Juízo Final se tornariam senhores e os
portugueses seus escravos, o que fez com que tivesse numerosos
seguidores por parte de escravos índios que fugiam das plantações do
Recôncavo. Os relatos sobre esta Santidade surgiam já em cartas
escritas pelos Jesuítas em 1584, mas foi a chegada do inquisidor Heitor
Furtado de Mendonça que despoletou as inúmeras denúncias e confissões.
Entre os membros da mesa inquisitorial contava"-se o Padre Fernão
Cardim. Cf. \textit{Primeira Visitação do Santo Ofício às partes do
Brasil pelo Licenciado Heitor Furtado de Mendonça. Confissões da Bahia
1591--1592}, Rio de Janeiro, F. Briguiet, 1935 e \textit{Primeira
Visitação do Santo Ofício às partes do Brasil pelo Licenciado Heitor
Furtado de Mendonça. Denunciações da Bahia 1591--1593}, São Paulo, Paulo
Prado, 1925. Entre os vários estudos sobre esta Santidade são de
referir as obras: José Calazans, \textit{A Santidade de Jaguaripe}, 
Bahia, 1952; Laura de Mello e Souza, \textit{O Diabo e a Terra de
Santa Cruz: Feitiçaria e Religiosidade Popular no Brasil Colonial}, 
São Paulo, Companhia das Letras, 1986; Ronaldo Vainfas, ``A Heresia do 
Trópico: Santidades Ameríndias no Brasil Colonial'', Tese Titular:
Universidade Federal Fluminense, 1993; o artigo de Alida C. Metcalf, ``Os 
Limites da Troca Cultural: o Culto da Santidade no Brasil Colonial'',
in \textit{Cultura Portuguesa na Terra de Santa Cruz}, coord. de Maria
Beatriz Nizza da Silva, Lisboa, Ed. Estampa, 1995, pp. 35--52 e Ronaldo
Raminelli, \textit{Imagens da Colonização. A Representação do Índio de
Caminha a Vieira}, São Paulo/Rio de Janeiro, Ed\textsc{usp}/Jorge Zahar Ed.,
1996.} sete ou oito léguas da Bahia se têm achado muitos; em o ano de
oitenta e dois indo um índio pescar, foi perseguido de um, e
acolhendo"-se em sua jangada o contou ao senhor; o senhor para animar o
índio quis ir ver o monstro, e estando descuidado por uma mão fora da
canoa, pegou dele, e o levou sem mais aparecer, e no mesmo ano morreu
outro índio de Francisco Lourenço Caeiro. Em Porto Seguro se vêm
alguns, e já têm morto alguns índios. O modo que têm em matar é:
abraçam"-se com a pessoa tão fortemente beijando"-a, e apertando"-a
consigo que a deixam feita toda em pedaços, ficando inteira, e como a
sentem morta dão alguns gemidos como de sentimento, e largando"-a fogem;
e se levam alguns comem"-lhes somente os olhos, narizes e pontas dos
dedos dos pés e mão, e as genitálias, e assim os acham de ordinário
pelas praias com estas cousas menos.

\paragraph{Dos mariscos}

\especie{Polvos}\footnote{ \textit{Polvo}: molusco
cefalópode, da ordem \textit{Octopoda}, com oito tentáculos guarnecidos
de ventosas que vive nos orifícios dos rochedos perto das costas. No
Brasil é comum a espécie \textit{Octopus tehuelchus}, que se alimenta
de caranguejos e mariscos utilizando os tentáculos para a captura e um
veneno secretado por glândulas especiais para imobilizar a presa, tem
um bico recurvado, semelhante ao do papagaio, localizado entre os
tentáculos, no centro do corpo ovoide, desprovido de concha. A sua
carne é comestível e muito apreciada.} O mar destas partes
é muito abundante de polvos; tem este marisco um capelo, sempre cheio
de tinta muito preta; e esta é sua defesa dos peixes maiores, porque
quando vão para os apanhar, botam"-lhes aquela tinta diante dos olhos, e
faz"-se água muito preta, então se acolhem. Tomam"-se à flecha, e
assobiam"-lhe primeiro; também se tomam com fachos de fogo de noite.
Para se comerem os açoitam primeiro, e quanto mais lhe derem mais então
ficam mais moles e gostosos.

\especie{Azula}\footnote{ \textit{Azula} ou \textit{apula} é um
molusco difícil de identificar pela descrição cardiniana e o termo não
consta do Vocabulário Tupi"-Guarani.} Este marisco é como um canudo de cana; 
é raro, come"-se, e para o baço bebido em pó e em jejum é único remédio.

\especie{Águas mortas}\footnote{ \textit{Águas mortas}, hoje
denominadas mais por ``águas"-vivas'', ``alforrecas'', ``urtiga"-do"-mar'',
``chora"-vinagre'' ou ainda ``mãe de água'', são as medusas ou celenterados
marinhos da classe dos Cifozoários, de corpo mole, semelhante à
gelatina, transparente. Muitos desses animais apresentam células
urticantes que causam queimaduras dolorosas. Os mais comuns no litoral
brasileiro são do gênero \textit{Rhizoztoma}.} Destas águas mortas há 
infinitas nestas partes e são grandes e são do tamanho
de um barrete; têm muitas dobras, com que tomam os peixes, que parecem
bolsos de atarrafa; não se comem, picando em alguma pessoa causam
grandes dores, e fazem chorar, e assim dizia um índio a quem uma mordeu
que tinha recebido muitas flechadas, e nunca chorara senão então. Não
aparecem senão em águas mortas. 


\paragraph[Dos caranguejos]{Dos caranguejos\protect\footnote{ Neste capítulo Cardim apresenta
os Crustáceos, que intitula de \textit{Caranguejos} ainda que inclua
também alguns moluscos, o que não é de estranhar já que para a Ciência
do Renascimento, as duas classes de animais fazem parte da classe dos
peixes da mesma forma que os crocodilos e os hipopótamos, na medida em
que vivem na água. Cf. Frank Lestringant, notas a, \textit{Le Brésil
d'André Thevet, Les Singularités de la France Antarctique (1557}), Paris, Ed. Chandeigne, 1997.}}

\especie{Uçá}\footnote{ \textit{Uçá} é um caranguejo,
crustáceo decápode, da família dos Gecarcinídeos (\textit{Ucides
cordatus}, L.) que vive usualmente nos mangues. O termo tupi ocorre
pela primeira vez num texto português com Cardim, e significa ``olhos
de pernas'' de \textit{ub} = ``perna'' + \textit{eçá} = ``olho''.} 
Uçá é um gênero de caranguejos que se acham na
lama, e são infinitos, e o sustentamento de toda esta terra, \textit{maxime} dos
escravos de Guiné, e índios da terra; são muito gostosos, sobre eles é
boa água fria. Têm uma particularidade de notar, que quando mudam a
casca se metem em suas covas, e aí estão dois, três meses, e perdendo a
casca, boca, e pernas, saem assim muito moles, e tornam"-lhe a nascer como dantes.

\especie{Guanhumig}\footnote{ \textit{Guanhumig, guayamú} ou
\textit{guayamum} são as designações para outro caranguejo da mesma
família do anterior (\textit{Cardisoma guanhumi}, Latr.) de tamanho
grande e que vivem em grupo no mato. O termo tupi ocorre pela primeira
vez num texto português nesta obra, mas é de difícil explicação, ainda
que haja alguns autores que o vejam associado ao nome de uma
constelação na astronomia dos Tupis maranhenses, que é semelhante:
\textit{ouegnonmoin.}} Este gênero de caranguejos são tão
grandes que uma perna de homem lhe cabe na boca; são bons para comer;
quando fazem trovões saem de suas covas, e fazem tão grande matinada
uns com os outros, que já houve pessoas que acudiram com suas armas,
parecendo que eram inimigos; se comem uma certa erva, quem então os
come morre. Estes são da terra, mas vivem em buracos à borda do mar.

\especie{Aratú}\footnote{ \textit{Aratú}: esta variedade de caranguejo
é da família dos Grapsídeos (\textit{Aratus pisoni}, M.~Edw.). Vivem nos
mangues e são usualmente de cor vermelha. O termo tupi ocorre pela
primeira vez precisamente neste texto.} Estes caranguejos
habitam nas tocas das árvores, que estão nos lamarões do mar; quando
acham algumas amêijoas tem a boca aberta, buscam logo alguma pedrinha,
e sutilmente dão com ela na amêijoa. A amêijoa logo se fecha e não
podendo fechar bem, por causa da pedrinha que tem dentro, eles com suas
mãos lhe tiram de dentro o miolo, e o comem.

 Há dez ou doze espécies de caranguejos nesta terra, e como tenho dito,
são tantos em número, e tão sadios que todos os comem, \textit{maxime} os
índios etc.

\especie{Ostras}\footnote{ \textit{Ostras}, moluscos bivalves,
marinhos, lamelibrânquios da família dos Ostreídeos, dos gêneros
\textit{Ostrea} e \textit{Crassostrea}, são comestíveis e vivem em
colônias fixas em rochas. Existem várias espécies como a
``ostra"-do"-mangue'' (\textit{Ostrea arborea}) e a ``ostra"-gigante"-do"-mangue'' 
(\textit{Crassostrea rhizophorae}). A espécie
brasileira mais comum é a \textit{Ostrea virginica} que pode atingir
até 20\,cm de comprimento que os indígenas denominavam de \textit{guerini}.} 
As ostras são muitas, algumas delas são muito grandes,
e têm o miolo como uma palma da mão; nestas se acham algumas pérolas
muito ricas; em outras mais pequenas também se acham pérolas mais
finas. Os índios naturais antigamente vinham ao mar às ostras, e
tomavam tantas que deixavam serras de cascas, e os miolos levavam de
moquém para comerem entre ano; sobre estas serras pelo discurso do
tempo se fizeram grandes arvoredos muito espessos, e altos, e os
portugueses descobriram algumas, e cada dia se vão achando outras de
novo, e destas cascas fazem cal, e de um só monte se fez parte do
Colégio da Bahia, os paços do Governador, e outros muitos edifícios, e
ainda não é esgotado: a cal é muito alva, boa para guarnecer, e caiar,
se está à chuva faz preta, e para vedar água em tanques não é tão
segura, mas para o mais tão boa como a pedra em Espanha.\footnote{ Cardim 
refere"-se à cal de ostra, que era utilizada como a cal viva ou
anidra e que provinha das denominadas \textit{ostreiras}, que existiam
em grande quantidade na faixa litoral brasileira, a par dos
\textit{sambaquis} que eram depósitos de refugos geralmente de ossos,
conchas e resíduos diversos acumulados pelo homem desde o Paleolítico
na mesma região litoral.} 

\especie{Mexilhões}\footnote{ \textit{Mexilhões} são os moluscos
lamelibrânquicos, pertencentes à família dos Mitilídeos, entre os quais
o \textit{sururu} (\textit{Mytellus falcatus}) e o \textit{bacuçu} (\textit{Modiolus
brasiliensis}) que é o ``mexilhão"-do"-mangue''. São comestíveis e vivem
fixados em rochas marinhas pelo bisso. Entre eles destaca"-se a espécie
\textit{Mytilus perna}, que mede até 8\,cm, tem concha alongada e ocorre
no litoral do Rio de Janeiro, São Paulo e Santa Catarina. O termo tupi
\textit{sururu} ocorre pela primeira vez num texto português em 1587,
com Gabriel Soares de Sousa, na \textit{Notícia do Brasil.}} 
Não faltam mexilhões nesta terra; servem aos naturais e portugueses de
colheres, e facas; têm uma cor prateada graciosa, neles se acha algum
aljofre.\footnote{ O \textit{aljofre} ou \textit{aljôfar} é uma planta
herbácea, da família das Borragináceas, cujas sementes se parecem com
pequenas pérolas, por isso também chamada de ``erva"-pérola'' ou
``milho"-do"-sol''.} Há um gênero deles pequenos, de que as gaivotas se
sustentam, e porque não o podem quebrar, têm tal instinto natural que
levantando"-o no bico ao ar o deixam cair tantas vezes no chão que o quebram. 

\especie{Berbigões}\footnote{ \textit{Berbigões} são os moluscos
bivalves da família dos Cardídeos (\textit{Anomalocardia brasiliana}) 
também denominados de ``sarro"-de"-pito'' ou ``cernambitinga''.} 
Os berbigões são gostosos e bons nesta terra, e neles se acham alguns
grãos de aljofre, e assim dos berbigões, como dos mexilhões há grande
número de muitas e várias espécies.

\especie{Búzios}\footnote{ \textit{Búzios} são também moluscos
gastrópodes marinhos da família dos Xenoforídeos, do gênero
\textit{Strombus}, de concha retorcida em forma de corneta que os
índios aproveitavam para vários fins como menciona Fernão Cardim.} 
Os maiores que há se chamam \textit{Guatapiggoaçu},\footnote{ \textit{Guatapiggoaçu, uatapuguaçu} ou \textit{atapú} é
outra espécie de búzio da família dos Cassidídeos (\textit{Cassis
tuberosa}). O termo tupi não aparece referenciado no Vocabulário de
Tupi"-Guarani, mas Cardim traduz por ``búzio grande'', que é onde este
ocorre pela primeira vez num texto português.} sc.~búzio grande; são
muito estimados dos naturais, porque deles fazem suas trombetas,
jaezes, contas, metaras, e arrecadadas, e luas, para os meninos, e são
entre eles de tanta estima que por um dão uma pessoa das que têm
cativas; e os portugueses davam antigamente um cruzado por um; são tão
alvos como marfins, e de largo muitos deles têm dois palmos, e um de comprimento.

\especie{Piriguay}\footnote{ \textit{Piraguay, perigoari} ou
\textit{preguari} é uma variedade de búzio, molusco gastrópode
prosobrânquio marinho da família dos Estrombídeos (\textit{Strombus
pugilis}, Linn.). O termo tupi ocorre pela primeira vez num texto
português com Cardim.} Estes se comem também, e das cascas
fazem sua contaria, e por tantas braças dão uma pessoa; destes bota às
vezes o mar fora serras, cousa muito para ver. 

 De búzios e conchas há muita quantidade nesta terra, muito galantes, e
para estimar, e de várias espécies.

\especie{Coral branco}\footnote{ \textit{Coral branco}: trata"-se de
uma estrutura calcária produzida por uma colônia de pólipos cnidários
marinhos antozoários, da ordem dos Escleractíneos, de forma, tamanho e
cor variáveis, conforme a espécie relacionada, que se depositam sobre
esqueletos calcários em geral arborescentes. Neste caso trata"-se do
``coral branco'' (\textit{Millepora nitidae}).} Acha"-se muita
pedra de coral branco debaixo do mar; nasce com as arvorezinhas toda em
folhas e canudos, como coral vermelho da Índia, e se este também o
fora, houvera grande riqueza nesta terra pela muita abundância que é
dele. É muito alvo, tira"-se com dificuldade, e também se faz cal dele.

\especie{Lagostins}\footnote{ \textit{Lagostins} é o nome comum a
duas espécies de crustáceos decápodes (\textit{Scyllarides
aequinoctialis} e \textit{S. brasiliensis}), da família dos Cilarídeos,
marinhos e semelhantes à lagosta, diferindo da mesma pela presença de
antenas largas e achatadas.} Há grande quantidade de
lagostins, por esta costa estar quase toda cercada de arrecifes, e
pedras; também se acham muitos ouriços e outros monstros, pelas
concavidades das mesmas pedras [\ldots{}]\footnote{ Palavra difícil de
identificar no manuscrito e na tradução inglesa do texto de Cardim,
onde aparece ``[\ldots{}] e outros monstros nas concavidades das rochas,
grandes (\textit{Cravesses}) ou caranguejos como os da Europa\ldots{}''.
Tradução da autora do texto in \textit{Purchas his Pilgrimes}, vol. \textsc{iv},
p. 1316.} ou lagostas grandes, como as da Europa, parece que não há por cá.


\paragraph{Das árvores que se criam na água salgada}

\especie{Mangues}\footnote{ \textit{Mangue} é a designação
comum a diversas plantas próprias das formações vegetais de águas
salobras. O Pe. Fernão Cardim menciona alguns destes conjuntos
vegetais, cujos componentes principais são: o ``mangue"-vermelho'', 
``mangue"-preto'' ou ``mangue"-verdadeiro'' (\textit{Rhizophora mangle}, L.)
da família das Rizoforáceas, cujas árvores atingem 5\,m de altura, de
casca fina e flores púrpura; fornece madeira clara, de pouca
resistência e a casca é tanífera e útil em curtume e na fabricação de
tintas e o ``mangue"-manso'', ``mangue"-branco'', ``mangue"-amarelo'' ou
``tinteira'' (\textit{Laguncularia racemosa}, Gaertn.) da família das
Combretáceas, que é uma árvore de folhas opostas e pequenas flores
brancas, que fornece madeira própria para carpintaria e a casca é
também tanífera. Existe ainda a \textit{Siriúba} (\textit{Avicennia
nitida}, Jacq.) da família das Verbenáceas. Cardim parece referir"-se
sobretudo à primeira espécie que é a que despede grandes raízes
adventícias em forma de trempes.} Estas árvores se parecem
com salgueiros ou sinceiros\footnote{ \textit{Sinceiro} é o mesmo que
\textit{salgueiro"-branco}, que é o nome vulgar e genérico das árvores e
arbustos salicáceos do gênero \textit{Salix}, de folhas lanceoladas,
que crescem à beira dos cursos de água. Vulgarmente são denominados de
``chorão''.} da Europa, deles há tanta quantidade pelos braços e esteiros
que o mar deita pela terra dentro, que há léguas de terra todas deste
arvoredo, que com as enchentes são regadas do mar; caminhamos logo
léguas por estes esteiros e dias inteiros pelos rios onde há estes
arvoredos; estão sempre verdes, e são graciosos, e aprazíveis, e de
muitas espécies; a madeira é boa para queimar, e para emadeirar casas;
é muito pesada, e rija como ferro; da casca se faz tinta, e serve de
casca para curtir ouros; são de muitas espécies: um certo gênero deita
uns gomos de cima de comprimento às vezes de uma lança até chegar à
água, e logo deitam muitas trempes, e raízes na terra, e todas estas
árvores estão encadeadas e feitas em trempes, e assim as raízes, e
estes ramos tudo fica preso na terra; enquanto são verdes estes gomos
são tenros, e porque são vãos por dentro se fazem deles boas frautas.

 Nestes mangues há um certo gênero de mosquitos que se chamam de
\textit{Mariguis},\footnote{ \textit{Mariguis} são os mosquitos
hematófagos que se desenvolvem nos mangues. Existem várias espécies
como o \textit{maruim, merium} ou \textit{muruim} que são da família
dos Ceratopogonídeos, do gênero \textit{Culicoides}, sendo uma das mais
conhecidas que ocorre desde o litoral baiano até Santos, o
\textit{Culicoides maruim}, Lutz. O termo tupi, que procede de
\textit{mberu} = ``mosca'' + \textit{î} = ``pequena'', ocorre pela primeira
vez num texto português em 1560, numa carta do Pe. José de
Anchieta.} tamaninos como piolhos de galinha: mordem de tal maneira e
deixam tal pruído, ardor e comichão, que não há valer"-se uma pessoa,
porque até os vestidos passam, e é boa penitência e mortificação
sofrê"-los numa madrugada, ou uma noite; para se defenderem deles não há
remédio senão untar"-se de lama, ou fazer grande fogo, e fumaça.

 Nestes mangues se criam muitos caranguejos, e ostras, e ratos, e há um
gênero destes ratos cousa monstruosa, todo o dia dormem e vigiam de noite.

 Nestes mangues criam os papagaios que são tantos em número, e gritam de
tal maneira, que parece gralheado de pardais, ou gralhas.

 Nas praias se acha muito perrexil, tão bom e melhor que de Portugal, 
que também se faz conserva.


\paragraph{Dos pássaros que se sustentam e acham na água salgada}

\especie{Guigratinga}\footnote{ \textit{Guigratinga,
guiratinga} ou \textit{graça"-branca} são as designações para as
espécies das aves da ordem Ciconiformes, da família dos Ardeídeos
(\textit{Herodias egretta}, Gm.). De porte elegante, com pernas e dedos
compridos, pescoço fino, bico longo e pontiagudo. O nome tupi vem de
\textit{guirá} = ``pássaro'' + \textit{tonga} = ``branco'' e ocorre pela
primeira vez em português neste texto de Cardim.} Este
pássaro é branco, do tamanho dos grous\footnote{ Cardim refere"-se ao
\textit{grou} que é uma ave pernalta, cultirrostra, da família das
Gruídeas, que aparece durante o inverno no Ribatejo e Alentejo.} de
Portugal, são em extremo alvos, os pés têm muito compridos, o bico
muito cruel, e agudo, e muito formoso por ser de um amarelo fino; as
pernas também são compridas entre vermelhas e amarelas. No pescoço têm
os melhores panachos e finos que buscar se pode, e parecem"-se com os
das Emas africanas.\footnote{ A \textit{ema} é uma ave sul"-americana,
subclasse dos ratites, da família dos Reídeas, com três dedos em cada
pata, que os ameríndios denominavam de \textit{nandu} ou
\textit{nhandu} e que impropriamente é chamada de avestruz. Este termo
tupi ocorre pela primeira vez num texto português com Cardim.} 

\especie{Caripirá}\footnote{ \textit{Cariripirá} ou \textit{grapirá}, 
que vulgarmente é denominado de ``tesoura'' ou ``alcatraz'', pertence à
família dos Fregatídeos (\textit{Fregata aquila}, L.), piscívora, é
bastante comum nas costas brasílicas. Possui uma cauda comprida e
bifurcada, lembrando uma tesoura, daí os nomes atribuídos de
``rabiforcado'' e ``tesoura''. O hábito que Cardim alude de acompanharem os
barcos em alto"-mar era muito importante já que indicavam aos navegantes
a proximidade da terra firme. O termo tupi ocorre pela primeira vez num
texto português com o Pe. Fernão Cardim e a palavra vem de
\textit{guirá} = ``pássaro'' + \textit{pirá} = ``peixe'', o que coincide
com os hábitos alimentares desta ave.} Por outro nome se
chama, Rabiforcado; estes pássaros são muitos, chamam"-se rabiforcado
por ter o rabo partido pelo meio; das penas fazem muito caso os índios
para empenaduras das flechas, e dizem que duram muito; em algum tempo
estão muito gordos, as enxudias são boas para corrimentos; costumam
estes pássaros trazer novas dos navios à terra, e são tão certos nisto
que raramente faltam, porque como se veem, de ordinário daí a dois ou
três dias chegam os navios.

\especie{Guacá}\footnote{ \textit{Guacá} é a \textit{gaivota} de
Portugal como afirma Cardim. Trata"-se de uma ave oceânica da família
dos Larídeos (\textit{Thaethusa magnirostris}, Licht.), também
conhecida por ``andorinha"-do"-mar''. Alimenta"-se usualmente de peixes, é
de coloração branco"-acinzentada, bico e pés vermelhos. O termo tupi
desapareceu da toponímia e ocorre pela única vez, num texto português
nesta obra de Cardim.} Este pássaro é a própria Gaivota de
Portugal; seu comer ordinário são amêijoas, e porque são duras, e as
não podem quebrar, levam"-nas no bico ao ar, e deixando"-as cair muitas
vezes as quebram e comem. Destas gaivotas há infinidade de espécies que
coalham as árvores e praias.

\especie{Guigratéotéo}\footnote{ \textit{Guigratéotéo, téu"-téu}, ave
da família dos Caradrídeos (\textit{Belonopterus cayanensis}, Gm.), é
vulgarmente denominada de ``téu"-téu'' ou ``quero"-quero'', que é
onomatopaico do grito da ave. O termo tupi ocorre pela primeira vez num
texto português com Fernão Cardim.} Esta ave se chama em
português Tinhosa, chama"-se Guigratéotéo, sc.~pássaro que tem
acidentes de morte, e que morre e torna a viver, como quem tem gota
coral, e são tão grandes estes acidentes que muitas vezes os acham os
índios pelas praias, os tomam nas mãos, e cuidando que de todo estão
mortos os botam por aí, e eles se caindo se alevantam e se vão embora;
são brancos e formosos, e destes há outras espécies que têm os mesmos acidentes.

\especie{Calcamar}\footnote{ \textit{Calcamar}, também denominadas de
\textit{corta"-mar, talha"-mar} ou \textit{bico"-rasteiro}, são aves da
família dos Larídeos (\textit{Rynchops intercedens}, Saunders). O nome
atribuído por Cardim não parece identificado como sendo tupi, mas
apenas uma atribuição generalizada pela forma como se deslocava junto
ao mar.} Estes pássaros são pardos do tamanho de Rolas, ou
Pombas; dizem os índios naturais que põem os ovos, e aí os tiram, e
criam seus filhos; não voam, mas com as asas e pés nadam sobre o mar
ligeiramente, e adivinham muito calmarias e chuveiros, e são tantos nas
calmarias ao longo dos navios que se não podem os marinheiros valer e
são a própria mofina e melancolia.

\especie{Ayaya}\footnote{ \textit{Ayaya, ajajá} ou \textit{colhereiro}
é um pássaro da família dos Tresquiornitídeos (\textit{Ajaja
ajaja}, L.). O termo tupi ocorre pela primeira vez num texto português
com o Pe. Fernão Cardim.} Estes pássaros são do tamanho de
Pegas, mais brancos que vermelhos, têm cor graciosa de um branco
espargido de vermelho, o bico comprido, e parece uma colher; para tomar
o peixe tem este artifício: bate com o pé na água, e tendo o pescoço
estendido espera o peixe e o toma, e por isso dizem os índios que tem
saber humano.

\especie{Saracura}\footnote{ \textit{Saracura}, nome comum a diversas
aves gruiformes da família dos Ralídeos. Vivem entre a vegetação
aquática e alimentam"-se de vermes, moluscos, crustáceos e insetos.
Entre algumas das espécies mais conhecidas no território brasileiro
contam"-se a ``saracura"-do"-banhado'' (\textit{Rallus sanguinolentus}), 
``saracura"-carijó'' (\textit{R. maculatus}), ``saracuraçu'' 
(\textit{Aramides ypacaha}), entre outras. O termo tupi ocorre pela
primeira vez num texto português com Cardim, e significa ``o que come
ou traga espiga'', de \textit{çara} = ``espiga'' + \textit{cur} = 
``comer'', ``tragar''.} Este pássaro é pequeno, pardo, tem os
olhos formosos com um círculo vermelho muito gracioso; tem um cantar
estranho, porque quem o ouve cuida ser de um pássaro muito grande,
sendo ele pequeno, porque canta com a boca e juntamente com a traseira,
faz outro tom sonoro, rijo, e forte, ainda que pouco cheiroso, que é
para espantar; faz esta música suave duas horas ante manhã, e à tarde
até se acabar o crepúsculo vespertino, e quando canta de ordinário
adivinha bom tempo.

\especie{Guará}\footnote{ \textit{Guará} é uma ave ciconiforme, da
família dos Tresquiornitídeos (\textit{Eudocimus ruber}, L.), que
habita os manguezais, arrozais e margens alagadas dos rios. Tem
plumagem vermelho"-carmesim, anda dentro de água, com o bico submerso,
abrindo e fechando as mandíbulas rapidamente, em busca de caranguejos,
caramujos e insetos, dos quais se alimenta. Vive em colônias no Pará,
Maranhão, Piauí, São Paulo, pantanal do Mato Grosso e Paraná.
Normalmente aparece traduzido do tupi por ``garça''. O termo tupi ocorre
pela primeira vez em latim, em 1560, numa carta de José de
Anchieta e num texto português, em 1576, na \textit{História da
Província Santa Cruz}, de Pêro de Magalhães de Gândavo.} Este
pássaro é do tamanho de uma Pega, tem o bico muito comprido com a ponta
revolta, e os pés de comprimento de um grande palmo; quando nasce é
preto, e depois se faz pardo; quando já voa faz"-se todo branco mais que
uma pomba, depois faz"-se vermelho claro, \textit{et tandem} torna"-se vermelho mais que a mesma grã, e
nesta cor permanece até à morte; são muitos em quantidade, mas não têm 
mais que esta espécie; criam"-se bem em casa, o seu comer é peixe,
carne, e outras cousas, e sempre hão"-de ter o comer dentro na água; a
pena destes é muito estimada dos índios, e delas fazem diademas,
franjas, com que cobrem as espadas com que matam; e fazem braceletes
que trazem nos braços e põem"-nas nos cabelos como botões de rosas, e
estas suas jóias e cadeias de ouro com que se ornam em suas festas, e 
estimam"-nas tanto que, com serem muito amigos de comerem carne humana,
dão muitas vezes os contrários que têm para comer em troco das ditas 
penas: andam em bando estes pássaros, e se lhe dá o sol nas praias, ou
indo pelo ar é cousa formosa de ver.

 Há outros muitos pássaros que do mar se sustentam, como Garças,
Gaviões, e certo gênero de águias, e outros muitos que seria largo contar.

\paragraph{Dos rios de água doce, e cousas que neles há}

 Os rios caudais de que esta província é regada são inumeráveis, e
alguns mui grandes, e mui formosas barras, não falando em as ribeiras,
ribeiros e fontes de que toda a terra é muito abundante, e são as águas
de ordinário mui formosas, claras, e salutíferas, e abundantes de
infinidade de peixes de várias espécies, dos quais há muitos de notável
grandura e de muito preço, e mui salutíferos, e dão"-se aos doentes por
medicina. Estes peixes pescam os índios com redes, mas o ordinário é a
linha com anzol. Entre estes há um peixe real de bom gosto e sabor que
se parece muito com o solho de Espanha; este se chama 
\textit{Jaú}\footnote{ \textit{Jaú} ou \textit{Jahú} é um peixe siluriforme,
de água doce, da família dos Pimelodídeos (\textit{Pauliceia lutkeni}, Steind.). 
É um dos maiores peixes brasileiros podendo atingir até
1,50\,m de comprimento e 120\,kg de peso. Existe nas bacias dos rios
Amazonas e Paraná. O termo tupi ocorre pela primeira vez num texto
português com Cardim.} são de quatorze, e quinze palmos, e às vezes
maiores, e muito gordos, e deles se faz manteiga. Em alguns tempos são
tantos os peixes que engordam os porcos com eles. Em os regatos
pequenos há muitos camarões,\footnote{ \textit{Camarão} é a
denominação geral atribuída aos crustáceos da ordem dos decápodes
marinhos e de água doce. Neste caso o Pe. Fernão Cardim refere"-se aos
camarões de água doce também designados por \textit{pitus}, que
pertencem à família dos Palemonídeos (\textit{Macrobrachium carcinus}). 
O termo tupi ocorre pela primeira vez num texto português em 1817, na
\textit{Corografia Brazílica}, do Pe. Manuel Aires de Casal.} e alguns
de palmo e mais de comprimento, e de muito bom gosto e sabor.

\paragraph{Das cobras de água doce}

\especie{Sucurijuba}\footnote{ \textit{Sucurijuba, sucuriju}
ou \textit{sucuri} são as designações para um réptil ofídio,
ovovivíparo, não venenoso, da família dos Boídeos (\textit{Eunectes
murinus}, L.) que chega a atingir 8\,m de comprimento, segundo alguns
autores mesmo 12\,m. Existe em quase todos os rios do território
brasileiro. O termo tupi ocorre pela primeira vez num texto português
em 1560, numa carta do Pe. José de Anchieta.} Esta
cobra é a mór, ou das maiores que há no Brasil, assim na grandeza como
na formosura; tomam"-se algumas de vinte e cinco pés, e de trinta em
comprimento, e quatro palmos em roda. Tem uma cadeia pelo lombo de
notável pintura e formosa, que começa da cabeça e acaba na cauda; tem
dentes como cão, e aferra em uma pessoa, vaca, veado, ou porco, e
dando"-lhes algumas voltas com a cauda, engole a tal cousa inteira, e
depois que assim a tem na barriga deixa"-se apodrecer, e os corvos a
comem toda de modo que não ficam senão os ossos, e depois torna a criar
carne nova, e ressurgir como dantes era, e a razão dizem os índios
naturais é, porque no tempo que apodrece tem a cabeça debaixo da lama,
e porque têm ainda em o toutiço tornam a viver; e porque já se sabe
isto quando as acham podres lhe buscam a cabeça, e as matam. O modo de
se sustentarem é esperarem os animais, ou gente estendidas pelos
caminhos, e em perpassando se enviam a eles, e os matam, e comem;
depois de fartas dormem de tal modo que às vezes lhe cortam o rabo
duas, três postas sem acordarem, como aconteceu que depois de cortarem
duas postas a uma destas, ao dia seguinte a acharam morta com dois
porcos monteses na barriga, e seria de cinquenta palmos.

\especie{Manima}\footnote{ \textit{Manima}: o termo tupi parece
designar, segundo a descrição de Cardim, uma cobra de água. Ocorre pela
primeira vez em português neste texto, mas para o qual não conseguimos
encontrar o seu equivalente científico ou atual na fauna
brasileira.} Esta cobra anda sempre na água, é ainda maior
que a sobredita, e muito pintada, e de suas pinturas tomaram os gentios
deste Brasil pintarem"-se; têm"-se por bem"-aventurado o índio a quem ela
se amostra, dizendo que hão"-de viver muito tempo, pois a Manima se lhes
mostrou\footnote{ No manuscrito existente em Évora falta uma parte
do texto que se encontra em inglês, na edição de Purchas: ``[\ldots{}] 
Muitas outras espécies de cobras existem nos rios de água fresca, que
eu deixo por breve descrição e, porque não há nada em particular que
possa ser dito sobre elas'', in \textit{Purchas his Pilgrimes}, vol.
\textsc{iv}, p. 1318. Tradução da Autora.} 


\paragraph{Dos lagartos de água}

\especie{Jacaré}\footnote{ \textit{Jacaré} é a denominação
comum aos répteis crocodilianos, da família dos Aligatorídeos, que têm
a pele grossa e coriácea, formada por escudos rijos, pernas curtas e
terminando em dedos providos de garras e de membranas natatórias.
Existem várias espécies no Brasil, dos gêneros \textit{Caiman} e 
\textit{Jacarétinga}, nomeadamente o ``jacaré"-coroa''
(\textit{Paleosuchus palpelrosus}), o ``jacaré"-curuá'' (\textit{P.
trigonatus}) de pequeno porte, o ``jacaré"-de"-lunetas'' (\textit{caiman
jacare}), o ``jacaré"-de"-papo"-amarelo'' (\textit{C. latirostris}) que é o
mais comum da Bahia para o Sul e que deve ter sido este que Fernão
Cardim mais observou. O termo tupi que significa \textit{ya"-caré =}
``aquele que olha de lado'' ou ``aquele que é torto'', ocorre pela primeira
vez num texto português em 1560, numa carta do Pe. José de
Anchieta.} Estes lagartos são de notável grandura, e alguns
há tão grandes como cães; tem o focinho como de cão muito comprido, e
assim têm os dentes. Têm por todo o corpo umas lâminas como cavalo
armado, e quando se armam não há flecha que os passe; são muito
pintados de várias cores; não fazem mal à gente, mas antes os tomam com
laços facilmente, e alguns se tomaram de doze, quinze palmos, e os
estimam muito, e os tem por estado os índios como rembabas,\footnote{ Não 
conseguimos identificar este nome tupi \textit{rembaba}, que
Cardim apresenta como sendo o cão.} sc.~cães, ou outra cousa de estado;
andam na água, e na terra põem ovos tão grandes como de patas, e tão
rijos que dando uns nos outros tinem como ferro; aonde estes andam logo
são sentidos pelos grandes gritos que dão; a carne destes cheira muito,
\textit{maxime} os testículos, que parecem almiscre, e são de estima: o esterco
tem algumas virtudes, em especial é bom para belidas.\footnote{ \textit{Belidas}: 
interpretamos como manchas na córnea do olho.} 


\paragraph[Dos lobos de água]{Dos lobos de água\protect\footnote{ O Pe. Fernão Cardim aborda
neste capítulo, que denomina ``Dos Lobos de água'', diversos animais
anfíbios que habitam em águas doces ou junto aos rios, e não só os
denominados ``lobos"-do"-mar''.}}

\especie{Jaguaruçu}\footnote{ \textit{Jaguaruçu}: este animal
aparece referido duas vezes pelo Pe. Fernão Cardim neste
\textit{Tratado} para designar duas espécies distintas: o \textit{guará}
e o \textit{jaguar} ou \textit{onça.} É precisamente com este autor
que este termo tupi é mencionado pela primeira vez num texto português.
O vocábulo tupi significa ``cão grande'', de \textit{jaguar} = ``cão'' +
\textit{uçu} = ``grande''. Por sua vez o termo tupi \textit{guará}, 
atribuído a um mamífero carnívoro da família dos Canídeos
(\textit{Chrysocyon brachyrus}), ocorre pela primeira vez num texto
português em 1618, no \textit{Diálogo das Grandezas do Brasil.}}
 Este animal é maior que nenhum boi; tem dentes de grande
palmo, andam dentro e fora de água, e matam gente; são raros, alguns
deles se acham no rio de S. Francisco, e no Paraguaçu.

\especie{Atacape}\footnote{ \textit{Atacape} é outro animal de
difícil identificação, que aparece referido no Vocabulário
Tupi"-português como \textit{taçape} e que é possivelmente semelhante ao
anterior pela descrição cardiniana, ou seja, \textit{lobos"-do"-mar} que
são carnívoros pinípedes, da família dos Otariídeos, habitantes da
região Antárctica, mas que por arribação atingem as costas do Rio de
Janeiro.} Estes lobos são mais pequenos, mas muito mais
daninhos, porque saem da água a esperar a gente, e por serem muito
ligeiros matam algumas pessoas, e as comem.

\especie{Pagnapopeba}\footnote{ \textit{Pagnapopeba, jaguapopeba} ou
\textit{iaguapopeba}: Por erro de transcrição do manuscrito em inglês
aparece a primeira designação, no entanto o termo tupi que ocorre no
manuscrito de Évora é \textit{jaguapopeba} para designar a lontra. Esta
expressão tupi ocorre pela primeira vez num texto português com Cardim,
mas passa a ser designado desta forma em outras obras contemporâneas.
Trata"-se da lontra, mamífero carnívoro da família dos Mustelídeos, de
patas espalmadas, que vive à beira da água e tem hábitos aquáticos. No
Brasil encontram"-se duas espécies: a \textit{Lutra platensis}, que
ocorre de São Paulo para o Sul, e a \textit{L. enudris}, no Rio de
Janeiro, Pará e Guiana.} Estas são as verdadeiras
lontras de Portugal. Há outro animal pequeno do tamanho de doninha,
chama"-se \textit{Sariguey beju},\footnote{ \textit{Sariguey"-beju} ou
\textit{cariguemeiu} é um mamífero carnívoro da família dos
Mustelídeos, semelhante à doninha, mas de maior dimensão. O termo tupi
ocorre pela primeira vez num texto português com Cardim.} este tem
ricas peles para forros; e destes animais de água há outras muitas
espécies, alguns não fazem mal, outros são muito ferozes.

\especie{Baéapina}\footnote{ \textit{Baéapina}: monstro marinho, cujo
termo tupi \textit{Igbaheapiná} significa ``Diabo pelado'' pois
\textit{Igbahé} = ``coisa má'' ou ``sobrenatural''. Este ocorre pela
primeira vez num texto português com Cardim.} Estes são
certo gênero de homens marinhos do tamanho de meninos, porque nenhuma
diferença têm deles; destes há muitos, não fazem mal.

\especie{Capijuara}\footnote{ \textit{Capijuara, capibara} ou
\textit{capivara} é um mamífero da ordem dos roedores, da família
dos Cavídeos (\textit{Hydrochoerus hydrochaeris}, Erxl.). É
considerado o maior dessa espécie, atingindo cerca de 1\,m de comprimento
e 50\,kg de peso, tem pelagem castanho"-avermelhado no dorso e
amarelo"-acastanhado no ventre. Vive em bandos, nas margens dos rios e
beira dos lagos, sempre na proximidade de matas ou cerrados. É de
hábitos diurnos, mas torna"-se muito ativo à noite, quando é perturbado
pela presença humana. Nada muito bem e apresenta a pata provida de
membranas interdigitais e cada dedo tem uma garra. É herbívoro e ocorre
na região cisandina da América do Sul. O termo tupi ocorre pela
primeira vez num texto português, em 1560, numa carta do Pe.
José de Anchieta e significa ``comedor de capim'' de \textit{capyi} = 
``erva'', ``capim'' + \textit{eguara =} particípio do verbo \textit{û} = 
``comer''.} Destes porcos de água há muitos e são do mesmo
tamanho dos porcos, mas diferem nas feições; no céu da boca têm pedra
muito grossa que lhes serve de dentes queixais. Esta os índios têm por
jóia para os filhos e filhas; não têm rabo, andam muito tempo debaixo
de água, porém habitam na terra, e nela criam seus filhos; seu comer é
erva e frutas que ao longo dos rios acham.

\especie{Itã}\footnote{ \textit{Itã} ou \textit{itan} são conchas
bivalvas de mexilhões que se encontram nas areias dos rios, às quais
ainda hoje se dá a aplicação que o Pe. Fernão Cardim refere no texto,
como cuia e colher. O termo tupi significa precisamente ``concha'' ou
``colher'' e ocorre pela primeira vez num texto português com
Cardim.} Há nos rios de água doce muitos gêneros de conchas
grandes e pequenas; algumas são tão grandes como boas cuias, e servem
de fazer a farinha com elas; outras são pequenas, e servem de colheres;
todas elas são compridas, e de uma cor prateada; nelas se acham algumas pérolas.

\especie{Cágados}\footnote{ \textit{Cágados} é a designação comum a
diversas espécies de répteis da família dos Quelídeos, da ordem dos
quelónios, principalmente dos gêneros \textit{Hydraspis, Platemys} e 
\textit{Hydromedusa.} Vivem em lagoas rasas e terrenos pantanosos.}
 Nos rios se acham muitos cágados, e são tantos em número
que os tapuias engordam em certos tempos somente para os ovos, e andam
a eles como a maravilhoso mantimento.

\especie{Guararigeig}\footnote{ \textit{Guararigeig} ou
\textit{guararieí} é uma variedade de rã, que é a denominação comum a
anfíbios anuros, da família dos Ranídeos, Leptodatilídeos, Elosídeos e
outros. No Brasil existem entre várias espécies a ``rã"-comum''
(\textit{Leptodactylus ocellatus}), a ``rã"-pimenta'' (\textit{L. pentadactylus}), 
a ``rã"-verdadeira'' (\textit{Rana palmipes}) e a
``rã"-malhada"-do"-banhado'' (\textit{Paludicula fuscomaculata}). O termo
tupi ocorre apenas neste texto de Fernão Cardim. Aparece designada por
\textit{Juiguaraigaraí}, em 1587, com Gabriel Soares de Sousa, na
\textit{Notícia do Brasil.}} Não faltam rãs em os rios,
fontes, charcos, lagoas; e são de muitas espécies, principalmente esta
Guararigeig; é cousa espantosa o medo que dela têm os índios
naturais, porque só de a ouvirem, morrem, e por mais que lhes preguem
não têm outro remédio senão deixar"-se morrer, tão grande é a
imaginação, e apreensão que tomam de a ouvir cantar; e qualquer índio
que a ouve morre, porque dizem que deita de si um resplandor como relâmpago.

 Todos estes rios caudais são de tão grandes e espessos arvoredos, que
se navegam muitas léguas por eles sem se ver terra de uma parte nem de
outra; por eles há muitas cousas que contar, que deixo por brevidade. 


\paragraph{Dos animais, árvores, ervas, que vieram de Portugal e se dão no Brasil}

 Este Brasil é já outro Portugal, e não falando no clima que é muito
mais temperado e sadio, sem calmas grandes, nem frios, e donde os
homens vivem muito com poucas doenças, como de cólica, fígado, cabeça,
peitos, sarna, nem outras enfermidades de Portugal; nem falando do mar
que tem muito pescado, e sadio; nem das cousas da terra que Deus cá deu
a esta nação; nem das outras comodidades muitas que os homens têm para
viverem, e passarem a vida, ainda que as comodidades das casas não são
muitas por serem as mais delas de taipa, e palha, ainda que já se vão
fazendo edifícios de pedra e cal, e telha; as comodidades para o
vestido não são muitas, por a terra não dar outro pano mais que de
algodão.\footnote{ Os ameríndios cultivavam nas cercanias dos seus
povoados o barbadense, uma qualidade sul"-americana de \textit{algodão}
(\textit{Gossypium barbadense}) que se desenvolvia muito bem nas terras baixas
e quentes do litoral brasílico. No Nordeste desenvolveu"-se um outro
tipo desta planta, o \textit{G.\,hirsutum.} Era conhecido entre os Tupis
como \textit{amandiyn} que significa ``o que dá novelo'', ou por
\textit{maniim, manoiu.} Estes últimos nomes ocorrem pela primeira vez
num texto português, em 1587, com Gabriel Soares de Sousa, na sua
\textit{Notícia do Brasil.}} E, nesta parte, padecem muito os da terra,
principalmente do Rio de Janeiro até São Vicente, por falta de navios
que tragam mercadorias e panos; porém as mais capitanias são servidas
de todo o gênero de panos e sedas, e andam os homens bem vestidos, e
rasgam muitas sedas e veludos. Porém está já Portugal, como dizia,
pelas comodidades que de lá lhe vêm.

\especie{Cavalos} Nesta província se dá bem a criação de cavalos e há
já muita abundância deles, e formosos ginetes de grande preço que valem
duzentos e trezentos cruzados e mais, e já há correr de patos, de
argolinhas, canas, e outros torneios, e escaramuças, e daqui começam
prover Angola de cavalos, de que lá tem.

\especie{Vacas} Ainda que esta terra tem os pastos fracos; e em Porto
Seguro há uma erva que mata as vacas em a comendo, todavia há já grande
quantidade delas e todo o Brasil está cheio de grandes currais, e há
homem que tem quinhentas ou mil cabeças; e principalmente nos campos de
Piratininga, por ter bons pastos, e que se parecem com os de Portugal,
é uma formosura ver a grande criação que há.

\especie{Porcos} Os porcos se dão cá bem, e começa de haver grande
abundância; é cá a melhor carne de todas, ainda que de galinha, e se dá
aos doentes, e é muito bom gosto.

\especie{Ovelhas} Até o Rio de Janeiro se acham já muitas ovelhas, e
carneiros, e engordam tanto que muitos arrebentam de gordos, nem é cá
tão boa carne como em Portugal.

\especie{Cabras} As cabras ainda são poucas, porém dão"-se bem na
terra, e vão multiplicando muito, e cedo haverá grande multidão.

\especie{Galinhas} As galinhas são infinitas, e maiores que no Reino,
e pela terra ser temperada se criam bem, e os índios as estimam, e as
criam por dentro do sertão trezentas e quatrocentas léguas; não é a
carne delas tão gostosa como no Reino.

\especie{Perus}\footnote{ \textit{Peru} é uma ave galiforme selvagem,
da família dos Fasianídeos, que ocorre do leste da América do Norte até
ao planalto mexicano. Originária das regiões montanhosas da América do
Norte foi domesticado pelos astecas, tendo sido encontrado no México
pelos companheiros de Hernán Cortez, foi por eles criado e chamado de
\textit{gallo"-pavo} ou \textit{gallopabo}, por apresentar algumas
semelhanças com o pavão. Foi levada para a Espanha e daí passou para
Portugal onde passou a ser denominada de \textit{galo do Peru} ou
\textit{galinha do Peru}, como a ela se refere Cardim e depois apenas
peru.} As galinhas de Peru se dão nesta terra, e há grande
abundância, e não há convite onde não entrem.

\especie{Adens}\footnote{ \textit{Adens} é o ganso, uma ave
passeriforme (\textit{Anser domesticus}) originária da Europa e
Ásia.} As gansas se dão bem, e há grande abundância; também
há outro gênero desta terra: são muito maiores, e formosas.

\especie{Cães} Os cães têm multiplicado muito nesta terra, e há"-os de
muitas castas; são cá estimados assim entre os portugueses que os
trouxeram, como entre os índios que os estimam mais que quantas cousas
têm, pelos ajudarem na caça, e serem animais domésticos, e assim os
trazem as mulheres às costas de uma parte para outra, e os criam como
filhos, e lhes dão de mamar ao peito. 

\especie{Árvores} As árvores de espinhos, como laranjeiras, cidreiras,
limoeiras de várias sortes, se dão também nesta terra que quase todo o
ano tem fruto, e há grandes laranjeiras, cidrais, até se darem pelos
matos, e é tanta a abundância destas cousas que delas se não faz caso.
Têm grandes contrárias nas formigas, e com tudo isto há muita
abundância sem nunca serem regadas, e como não falta açúcar se fazem
infinitas conservas, sc.~cidrada, limões, florada etc.

\especie{Figueiras} As figueiras se dão cá bem, e há muitas castas,
como beboras, figos, negrais, berjaçotes e outras muitas castas: e até
no Rio de Janeiro que são terras mais sobre quentes dão duas camadas no ano.

\especie{Marmeleiros} No Rio de Janeiro, e São Vicente, e no campo de
Piratininga se dão muitos marmelos, e dão quatro camadas uma após
outra, e há homem que em poucos marmeleiros colhe dez, e doze mil
marmelos, e aqui se fazem muitas marmeladas, e cedo se escusaram as da
Ilha da Madeira.

\especie{Parreiras} Há muitas castas de uvas como ferrais, boais,
bastarda, verdelho, galego e outras muitas, até o Rio de Janeiro tem
todo o ano uvas se as querem ter, porque se as podam cada mês vão dando
uvas sucessivas. No Rio de Janeiro, e \textit{maxime} em Piratininga se dão
vinhas, e carregam de maneira que se vem ao chão com elas, não dão mais
que uma novidade, já começam de fazer vinhos, ainda que têm trabalho em
o conservar, porque em madeira fura"-lhe a broca logo, e talhas de
barro, não nas têm; porém buscam seus remédios, e vão continuando, e
cedo haverá muitos vinhos.

\especie{Ervas} No Rio de Janeiro e Piratininga há muitas roseiras,
somente de Alexandria,\footnote{ \textit{Roseira de Alexandria}: este
tipo de roseira é o mais antigo que se conhece, a \textit{Rosa
centifolia}, L.} destilam muitas águas, e fazem muito açúcar rosado
para purgas, e para não purgar, porque não têm das outras rosas; cozem
as de Alexandria na água, e botando"-lha fora fazem açúcar rosado muito
bom com que não purgam.

\especie{Legumes} Melões não faltam em muitas capitanias, e são bons e
finos; muitas abóboras que fazem conserva, muitas alfaces, de que
também a fazem couves, pepinos, rabãos, nabos, mostarda, hortelã,
coentros, endros, funchos, ervilhas, gerselim, cebolas, alhos,
borragens, e outros legumes que do Reino se trouxeram, que se dão bem
na terra.

\especie{Trigo} No Rio de Janeiro e Campo de Piratininga se dá bem
trigo, não no usam por não terem atafonas nem moinhos, e também têm
trabalho em o colher, porque pelas muitas águas, e viço da terra, não
vem todo junto, e multiplica tanto que um grão deita setenta, e oitenta
espigas, e umas maduras vão nascendo outras e multiplica quase
\textit{in infinitum}. De menos de uma quarta cevada que um homem semeou no Campo
de Piratininga, colheu sessenta e tantos alqueires, e se os homens se
dessem a esta grangeria, seria a terra muito rica e farta.

\especie{Ervas cheirosas} Há muitos magiricões, cravos amarelos, e
vermelhos se dão bem em Piratininga, e outras ervas cheirosas, como
cebolacecê etc.

 Sobretudo tem este Brasil uma grande comodidade para os homens viverem
que não se dão nela percevejos, nem piolhos, e pulgas há poucas, porém,
entre os índios, e negros da Guiné acham piolhos; porém, não faltam
baratas, traças, vespas, moscas, e mosquitos de tantas castas, e tão
cruéis, e peçonhentos, que mordendo em uma pessoa fica a mão inchada
por três ou quatro dias \textit{maxime} aos Reinóis, que trazem o sangue fresco,
e mimoso do pão e vinho, e mantimentos de Portugal.

\chapter[Do princípio e origem dos índios do Brasil e de seus costumes, 
adoração e cerimônias \medskip]{Do princípio e origem\break dos índios do Brasil\subtitulo{e de seus costumes, 
adoração\break e cerimônias\protect\footnote[*]{ \NoCaseChange{Estes textos do
Padre Fernão Cardim, tal como o anterior sobre o Clima e Terra do
Brasil, foram publicados pela primeira vez no ano de 1625, em Londres,
incluídos na coleção de viagens \textit{Purchas his Pilgrimes}. Vide
``Introdução'' e nota no início do Tratado \textit{Do Clima e Terra do
Brasil e de algumas cousas notáveis que se acham na terra como no mar}, 
desta edição. Os manuscritos em português encontram"-se igualmente em
Évora, na Biblioteca Pública e Arquivo Distrital, no códice \textsc{cxv}/1--33, 
fls. 1--12v.}}}}
\hedramarkboth{Do princípio e origem dos índios}{Fernão Cardim}

Este gentio parece que não tem conhecimento do
princípio do Mundo, do dilúvio parece que tem alguma notícia, mas como
não tem escrituras, nem caracteres, a tal notícia é escura e confusa;
porque dizem que as águas afogaram e mataram todos os homens, e que
somente um escapou em riba de um Janipaba,\footnote{ \textit{Janipaba}: 
trata"-se da \textit{ianipaba} ou \textit{jenipapo} já mencionado por
Cardim no \textit{Tratado} referente ao clima e terra do Brasil, no
capítulo ``Dos óleos de que usam os índios para se untarem''. Cardim
menciona neste texto uma das crenças dos ameríndios sobre a destruição
do Mundo, uma espécie de segundo dilúvio que resultou das disputas
entre os dois filhos de Sumé, tendo sido desencadeado por Tamendonare
que o provocou, batendo fortemente na terra, provavelmente com o pé,
fazendo jorrar uma enorme fonte, cuja água recobriu o Mundo. Desta nova
destruição somente escaparam os dois irmãos com as respectivas
mulheres, refugiando"-se no cume das mais altas montanhas: aquele com a
sua companheira na copa de uma palmeira e Aricoute e a esposa no cimo
de um jenipapeiro. Do primeiro descenderiam os Tamoios e do segundo os
Temiminós, inimigos irredutíveis. Cf. Jorge Couto, \textit{op. cit.}, pp. 109--117.} 
com uma sua irmã que estava prenhe, e que estes dois têm
seu princípio, e que dali começou sua multiplicação.\footnote{ Considerou"-se 
durante anos, em parte devido ao fato de serem ágrafos,
que os ameríndios da floresta tropical não tinham um sistema de
crenças, mas segundo o testemunho dos cronistas e viajantes
quinhentistas e seiscentistas, sabe"-se que tal não é correto. (Vide
nota supra sobre a questão do dilúvio e do nascimento das nações.) Os
estudos efetuados demonstram a existência de uma grande homogeneidade
relativamente ao discurso cosmológico, aos temas míticos e à vida
religiosa dos povos Tupi"-Guarani. Cf. Eduardo Viveiros de Castro,
\textit{Arawetê. Os Deuses Canibais}, Rio de Janeiro, 1986, p. 90, cit.
in Jorge Couto, \textit{op. cit.}, pp. 109--117. Coube sobretudo aos
trabalhos de Alfred Métraux um melhor esclarecimento sobre as questões
da religiosidade das sociedades ameríndias, sendo de destacar a sua
obra \textit{A Religião dos Tupinambás e suas relações com as demais
Tribos Tupi"-Guarani}, trad. port., 2ª ed., São Paulo, 1979.} 

\paragraph{Do conhecimento que têm do Criador}

 Este gentio não tem conhecimento algum de seu Criador, nem de cousa do
Céu, nem se há pena nem glória depois desta vida, e portanto não tem
adoração nenhuma nem cerimônias, ou culto divino, mas sabem que têm
alma e que esta não morre e depois da morte vão a uns campos onde há
muitas figueiras ao longo de um formoso rio, e todas juntas não fazem
outra cousa senão bailar; e têm grande medo do demônio, ao qual chamam
\textit{Curupira},\footnote{ \textit{Curupira} é a designação para o
``espírito mau'' entre os indígenas, um ser fantástico que vivia nas
matas e tinha os dedos dos pés voltados para trás e o calcanhar para a
frente. O termo tupi pode ser traduzido por ``sarnento'', de
\textit{curub} = ``sarna'' + \textit{pir} = ``pele''. Este termo tupi
ocorre pela primeira vez num texto português, em 1560, numa
carta do Pe. José de Anchieta.} \textit{Taguaigba},\footnote{ \textit{Taguaigba} 
ou \textit{taguaiba} aparece na
terminologia tupi como ``diabo''. O termo tupi ocorre pela primeira vez
num texto em português com Cardim.} \textit{Macachera},\footnote{ \textit{Macachera} ou \textit{macaxeira}: tal como os
nomes anteriores, é um termo tupi para designar o ``diabo''. Este também
ocorre pela primeira vez num texto português com Cardim.} \textit{
Anhanga},\footnote{ \textit{Anhanga} ou \textit{Anhangá}, mais uma
das formas de os índios designarem o ``diabo'', gênio do mal. O termo
tupi ocorre também pela primeira vez num texto português com Cardim.
Este termo parece que se pode explicar literalmente por \textit{a"-ñan =}
``encesta a gente'', ``mete a gente em cesto'' ou ``apanha a gente''. Há
ainda alguns autores que consideram \textit{Anhanga} como oposto a
\textit{Tupã}, logo como ``espírito do mal'', já que Tupã designava ``o
espírito do bem'', a quem não precisavam de fazer ofertas, ao passo que
a Anhanga tinham que fazer ofertas para o apaziguar, como alega
Cardim.}  e é tanto o medo que lhe têm, que só de imaginarem
nele morrem, como aconteceu já muitas vezes; não no adoram, nem a
alguma outra criatura, nem têm ídolos de nenhuma sorte, somente dizem
alguns antigos que em alguns caminhos têm certos postos, onde lhe
oferecem algumas cousas pelo medo que têm deles, e por não morrerem.
Algumas vezes lhe aparecem os diabos, ainda que raramente, e entre eles
há poucos endemoniados.

 Usam de alguns feitiços, e feiticeiros, não porque creiam neles, nem os
adorem, mas somente se dão a chupar em suas enfermidades,
parecendo"-lhes que receberão saúde, mas não por lhes parecer que há
neles divindade, e mais o fazem por receber saúde que por outro algum
respeito. Entre eles se alevantaram algumas vezes alguns feiticeiros, a
que chamam \textit{Caraíba},\footnote{ Sobre estes personagens, os
\textit{Caraíbas }, veja"-se a ``Introdução'' desta obra. Para o termo
indígena têm sido dadas algumas explicações etimológicas, como
\textit{Caraíbebé} = ``o santo que voa'', de \textit{cari} = ``santo'' +
\textit{bebé} = ``veloz'', ``rápido'', ``voador''. Este termo tupi
relaciona"-se etimologicamente com o etnônimo \textit{caribe},
designação que os Europeus quinhentistas davam aos indígenas de vários
grupos étnicos das Antilhas, América Central e do extremo norte da
América do Sul. Nesses idiomas dos grupos caribe e aruaque o termo
\textit{caribe} traduzia"-se por ``homem valente'', ``corajoso'' ou ``herói''.
O termo tupi \textit{caraíba} ocorre a primeira vez num texto português
em 1554, numa carta do Pe. Luís da Grã em que surge
para designar os ``cristãos'', tal como acontece mais tarde, em 1584, 
numa carta do Pe. José de Anchieta. Mas exatamente na mesma
época, o Pe. Fernão Cardim utiliza este termo para designar os
``feiticeiros indígenas'', dando ao vocábulo todavia uma conotação
pejorativa, já que entre os indígenas \textit{caraíba} designava o
``guia espiritual'', espécie de pajé que presidia aos seus cultos
religiosos. Cf. António Geraldo da Cunha, \textit{Dicionário Histórico
das Palavras Portuguesas de origem Tupi}, 3º ed., São Paulo,
Melhoramentos, Ed. da Universidade de São Paulo, 1989, pp. 102--103.} 
Santo ou Santidade, e é de ordinário algum índio de ruim vida: este faz 
algumas feitiçarias, e cousas estranhas à natureza, como mostrar que 
ressuscita a algum vivo que se faz de morto, e com esta e outras cousas semelhantes traz após si todo
o sertão enganando"-os dizendo"-lhes que não rocem, nem plantem seus
legumes, e mantimentos, nem cavem, nem trabalhem etc., porque com sua
vinda é chegado o tempo em que as enxadas por si hão"-de cavar, e os
\textit{panicus}\footnote{ \textit{Panicu} ou \textit{panacu} é um
cesto grande, uma espécie de canastra. O termo tupi ocorre muitas vezes
em textos portugueses, a partir de 1557, no \textit{Diálogo sobre a
Conversão do Gentio}, do Padre Manuel da Nóbrega.} ir às roças
e trazer os mantimentos, e com esta falsidade os traz tão embebidos, e
encantados, deixando de olhar por suas vidas, e granjear os mantimentos
que morrendo de pura fome, se vão estes ajuntamentos desfazendo pouco a
pouco, até que a Santidade fica só, ou a matam.

 Não têm nome próprio com que expliquem a Deus, mas dizem que
\textit{Tupã}\footnote{ \textit{Tupã}, termo tupi, que significava
``pai que está no alto'', para designar o raio e o trovão, e assim, por
extensão, Deus. Segundo a tese de vários etnólogos, nomeadamente
Alfred Métraux, \textit{Tupã}, ``pai que está no alto'', era uma figura
secundária na mitologia tupi, correspondendo apenas a um gênio ou
demônio do raio e do trovão, cujas deslocações provocavam tempestades.
Cf. Alfred Métraux, \textit{op. cit.}, pp. 31--39. Outros consideram
que este era uma divindade destruidora, em oposição a \textit{Monan}, 
divindade criadora. Hélène Clastres opina, no entanto, que Tupã não é
nem criador do mundo, nem transformador ou herói cultural, nenhum
fato, gesto ou invenção lhe é expressamente atribuída. Cf. \textit{La
Terre sans Mal}, p. 32. O termo tupi ocorre pela primeira vez num texto
português, em 1549, numa carta do Pe. Manuel da Nóbrega como
uma espécie de divindade dos trovões.} é o que faz os
trovões e relâmpagos, e que este é o que lhes deu as enxadas, e
mantimentos, e por não terem outro nome mais próprio e natural, chamam
a Deus \textit{Tupã.} 

\paragraph{Dos casamentos}

Entre eles há casamentos, porém há muita dúvida se são
verdadeiros, assim por terem muitas mulheres, como pelas deixarem muito
facilmente por qualquer arrufo, ou outra desgraça, que entre eles
aconteça; mas, ou verdadeiros ou não, entre eles se faziam deste 
modo.\footnote{ A questão dos casamentos foi muito debatida pelos
missionários Jesuítas que encontraram sociedades ameríndias que
praticavam a poligamia, ou poliginia, que era o casamento de um homem
com mais de uma mulher. Poligamia e adultério eram duas constantes da
vida dos índios, situação que se complicava depois de tomarem o
batismo, o que terá levado os Padres Jesuítas a encararem o casamento
como mais uma relação de união de um homem e de uma mulher, vivendo em
comunhão com os seus filhos, há vários anos e tendo eles mais de trinta
anos, poderiam ser considerados como vivendo em ``matrimônio''. Cf. Ana
Maria de Azevedo, \textit{op. cit.}, pp. 255--278 e Eugénio dos Santos,
\textit{op. cit.}, pp. 107--118.} Nenhum mancebo se acostumava casar
antes de tomar contrário, e perseverava virgem até que o tomasse e
matasse correndo"-lhe primeiro suas festas por espaço de dois ou três
anos; a mulher da mesma maneira não conhecia homem até lhe não vir sua
regra, depois da qual lhe faziam grandes festas; ao tempo de lhe
entregarem a mulher faziam grandes vinhos, e acabada a festa ficava o
casamento perfeito, dando"-lhe uma rede lavada, e depois de casados
começavam a beber, porque até aí não o consentiam seus pais,
ensinando"-os que bebessem com tento, e fossem considerados e prudentes
em seu falar, para que o vinho lhe não fizesse mal, nem falassem
cousas ruins, e então com uma cuia\footnote{ Vide o \textit{Tratado}
cardiniano sobre o clima e terra do Brasil, cap. \textsc{v}, ``Das Árvores de
Fruto''.} lhe davam os velhos antigos o primeiro vinho, e lhe tinham a
mão na cabeça para que não arrevessassem, porque se arrevessava tinham
para si que não seria valente e vice"-versa. 

\paragraph{Do modo que têm em seus comer e beber}

Este gentio come em todo o tempo, de noite e de dia, e a cada
hora e momento, e como tem que comer não o guardam muito tempo, mas
logo comem tudo o que têm e repartem com seus amigos, de modo que de um
peixe que tenham repartem com todos, e têm por grande honra e primor
serem liberais, e por isso cobram muita fama e honra, e a pior injúria
que lhes podem fazer é terem"-nos por escassos, ou chamarem"-lho, e
quando não têm que comer são muito sofridos com fome e sede.

 Não têm dias em que comam carne e peixe; comem todo gênero de carnes,
ainda de animais imundos, como cobras, sapos, ratos e outros bichos
semelhantes, e também comem todo gênero de frutas, tirando algumas
peçonhentas, e sua sustentação é ordinariamente do que dá a terra sem
a cultivarem, como caças e frutas; porém têm certo gênero de
mantimentos de boa substância, e sadio, e outros muitos legumes de que
abaixo se fará menção. De ordinário não bebem enquanto comem, mas
depois de comer bebem água, ou vinho que fazem de muitos gêneros de
frutas e raízes, como abaixo se dirá, do qual bebem sem regra, nem modo
e até caírem.

 Têm alguns dias particulares em que fazem grandes festas, todas se
resolvem em beber, e duram dois, três dias, em os quais não comem, mas
somente bebem, e para estes beberes serem mais festejados andam alguns
cantando de casa em casa, chamando e convidando quantos acham para
beberem, e revezando"-se continuam estes bailos e música todo o tempo
dos vinhos, em o qual tempo não dormem, mas tudo se vai em beber, e de
bêbados fazem muitos desmanchos, e quebram as cabeças uns aos outros, e
tomam as mulheres alheias etc. Antes de comer nem depois não dão
graças a Deus, nem lavam as mãos antes de comer, e depois de comer as
alimpam aos cabelos, corpo e paus; não têm toalhas, nem mesa, comem
assentados, ou deitados nas redes, ou em cócoras no chão, e a farinha
comem de arremesso, e deixo outras muitas particularidades que têm no
comer e no beber, porque estas são as principais.

\paragraph{Do modo que têm em dormir} 

Todo este gentio tem por cama umas redes de algodão, e ficam
nelas dormindo no ar; estas fazem lavradas, e como no ar, e não tem
outros cobertores nem roupa, sempre no Verão e Inverno tem fogo
debaixo: não madrugam muito, agasalham"-se com cedo, e pelas madrugadas
há um principal em suas \textit{ocas}\footnote{ \textit{Oca} é o
termo tupi para designar a habitação comunitária ameríndia. O termo
tupi parece ser originário do verbo \textit{og} = ``cobrir'', ``tapar'',
que faz no supino \textit{oca} = ``para tapar'' e no infinito
\textit{oga}, e nessas duas formas serve de substantivo = ``casa''. 
Este termo tupi ocorre a primeira vez num texto português
precisamente com Fernão Cardim, nesta obra onde faz uma descrição
pormenorizada da mesma. Mais à frente, em capítulo próprio intitulado
de ``As Casas'', o Pe. Fernão Cardim descreve estes locais, que
eram grandes habitações onde viviam cerca de duzentas e mais pessoas.} 
que deitado na rede por espaço de meia hora lhes prega, e admoesta que
vão trabalhar como fizeram seus antepassados, e distribui"-lhes o tempo,
dizendo"-lhes as cousas que hão"-de fazer, e depois de alevantado
continua a pregação, correndo a povoação toda. Tomaram este modo de um
pássaro que se parece com os falcões o qual canta de madrugada e lhe
chamam rei, senhor dos outros pássaros, e dizem eles que assim como
aquele pássaro canta de madrugada para ser ouvido dos outros, assim
convém que os principais façam aquelas falas e pregações de madrugada
para serem ouvidos dos seus.

\paragraph{Do modo que têm em se vestir}

Todos andam nus assim homens como mulheres, e não têm gênero
nenhum de vestido e por nenhum caso \textit{verecundant},\footnote{ Expressão latina 
que significa ``envergonham"-se''.} antes parece
que estão no estado de inocência nesta parte, pela grande honestidade
e modéstia que entre si guardam, e quando algum homem fala com mulher
vira"-lhe as costas. Porém, para saírem galantes, usam de várias
invenções, tingindo seus corpos com certo sumo de uma árvore\footnote{ Os ameríndios 
utilizavam usualmente o sumo de algumas plantas que
proporcionavam tintas, como o \textit{jenipapo}, inicialmente azul
escuro mas que enegrecia com a oxidação. Antes de pintarem os corpos os
índios depilavam todo o corpo, utilizando uma pedra muito afiada que
parecia uma navalha, ou as unhas. Depois do contato com os Europeus
passaram a utilizar uma pinça.} com que ficam pretos, dando muitos
riscos pelo corpo, braços etc., a modo de imperiais. 

 Também se empenam, fazendo diademas e braceletes, e outras invenções
muito lustrosas, e fazem muito caso de todo o gênero de penas finas.
Não deixam criar cabelo nas partes de seu corpo, porque todos os
arrancam, somente os da cabeça deixam, os quais tosquiam de muitas
maneiras, porque uns o trazem comprido com uma meia lua rapada por
diante, que dizem tomaram este modo de S.\,Tomé,\footnote{ Este termo
tupi \textit{Maire"-Monan, Mair"-Zumane, Sumé} ou \textit{Pai Zomé} é um
dos mitos históricos americanos, sendo este uma entidade mitológica, um
herói, ``grande pajé e caraíba'', que ensinou aos indígenas o cultivo da
terra, a agricultura de coivara e que instituiu a organização social.
Mais tarde, os missionários alteraram este termo para \textit{Tomé}, a
fim de identificar esta personalidade mítica com o apóstolo São Tomé,
que teria atingido o Novo Mundo para divulgar a mensagem cristã.
Segundo algumas interpretações, Monan, Maíra e Sumé representavam a
mesma personagem. Cf. Alfred Métraux, \textit{op. cit.}, p. 15 e
Jorge Couto, \textit{op. cit.}, p. 110.} e parece que tiveram dele
alguma notícia, ainda que confusa. Outros fazem certo gênero de coroas
e círculos que parecem frades: as mulheres têm cabelos compridos e de
ordinário pretos, e de uns e outros é o cabelo corredio; quando andam
anojados deixam crescer o cabelo, e as mulheres quando andam de dó,
cortam os cabelos, e também quando os maridos vão longe, e nisto
mostram terem"-lhe amor e guardarem"-lhe lealdade; é tanta a variedade
que têm em se tosquiarem, que pela cabeça se conhecem as nações.

 Agora já andam alguns vestidos, assim homens como mulheres, mas
estimam"-no tão pouco que o não trazem por honestidade, mas por
cerimônia, e porque lho mandam trazer, como se vê bem, pois alguns
saem de quando em quando com umas jornes\footnote{ A \textit{jórnea} 
ou \textit{jorne} (fr. \textit{journade}) era um tipo
de pelote da primeira metade do século \textsc{xv}, que se usava solta e
ampla\textit{.} Aparentemente a jórnea era uma veste própria para
caçadas e viagens ou servia para a gente de inferior condição. Cf.
A.H. de Oliveira Marques, \textit{A Sociedade Medieval Portuguesa}, 2ª
ed., Lisboa, Liv. Sá da Costa Ed., 1971, pp. 23--62.} que lhes dão pelo
umbigo sem mais nada, e outros somente com uma carapuça na cabeça, e o
mais vestido deixam em casa: as mulheres fazem muito caso de fitas e pentes.

\paragraph{Das casas}

Usam estes índios de umas \textit{ocas} ou casas de madeira
cobertas de folha, e são de comprimento algumas de duzentos e trezentos
palmos, e têm duas e três portas muito pequenas e baixas; mostram sua
valentia em buscarem madeira e esteios muito grossos e de dura, e há
casa que tem cinquenta, sessenta ou setenta lanços de 25 ou 30 palmos
de comprimento e outros tantos de largo.
 
 Nesta casa mora um principal,\footnote{ O \textit{Principal} a que o
Pe. Fernão Cardim alude ao longo deste texto por diversas vezes, parece
ser um dos membros do grupo tribal que detinha mais poderes dentro da
aldeia e cuja principal missão seria a de orientar a vida comunitária e
em período de guerra dirigir os homens, os quais lhe obedeciam por
respeito, mais do que pela força. A generalidade dos grupos ameríndios
da floresta tropical, incluindo os Tupi"-Guarani, adotou como forma de
organização política predominante o grupo local (correspondente a uma
taba), que se situava numa posição intermédia entre a menor unidade
vicinal (a oca) e o agrupamento territorial mais abrangente (o grupo
tribal). Cf. Florestan Fernandes, \textit{A Organização Social dos
Tupinambás}, 2ª ed., São Paulo/Brasília, 1989, p. 55, cit. in Jorge
Couto, \textit{op. cit.}, pp. 95--97.} ou mais, a que todos obedecem, e
são de ordinário parentes; e em cada lanço destes pousa um casal com
seus filhos e família, sem haver repartimento entre uns e outros, e
entrar em uma destas casas é ver um labirinto, porque cada lanço tem
seu fogo e suas redes armadas, e alfaias, de modo que entrando nela se
vê tudo quanto tem, e casa há que tem duzentas e mais pessoas.

\paragraph[Da criação dos filhos]{Da criação dos filhos\protect\footnote{ Tema que interessou muito o
autor que o aborda várias vezes ao longo dos seus textos, o que é
natural já que na Europa Quinhentista a criança era criada à margem dos
pais, usualmente por amas, que as amamentavam, não existindo laços
afetivos entre as mães e os recém"-nascidos, e mesmo mais tarde na
medida em que as crianças viviam no ``mundo dos adultos'' até ao Século
das Luzes. Cf. \textit{História da Vida Privada}, trad. port., dir.
Philippe Ariès, Lisboa, Ed. Afrontamento, 1990.}}

 As mulheres parindo, e parem no chão, não levantam a criança, mas
levanta"-a o pai, ou alguma pessoa que tomam por seu compadre, e na
amizade ficam como os compadres entre os Cristãos; o pai lhe corta a
vide com os dentes, ou com duas pedras, dando com uma na outra, e logo
se põe a jejuar até que lhe cai o umbigo, que é de ordinário até os
oito dias, e até que não lhe caia não deixam o jejum, e em lhe caindo,
se é macho lhe faz um arco com flechas, e lho ata no punho de rede, e
no outro punho muitos molhos de ervas, que são os contrários que seu
filho há de matar e comer, e acabadas estas cerimônias fazem vinhos com
que alegram todos.\footnote{ Os ritos de gestação e de nascimento eram
muito considerados pelas sociedades ameríndias. Assim, os ritos de
gestação impunham um longo período de proibição de relações sexuais com
a mulher desde o momento em que era detectada a gravidez até que a
criança andasse. A \textit{couvade} era uma complexa instituição que
implicava a rigorosa observância de um período de resguardo e
abstinência por parte do pai, quando se tratava de um recém"-nascido do
sexo masculino, a execução, por parte do progenitor, de vários atos de
significado mágico, competindo"-lhe ainda dar ao filho o nome de um
antepassado. Tratando"-se de uma menina, o progenitor era substituído
nesta função pela mulher ou por uma irmã desta. Cf. Florestan
Fernandes, \textit{op. cit.}, pp. 145--152, cit. in Jorge Couto, 
\textit{op. cit.}, p. 94.} As mulheres quando parem logo se vão lavar
aos rios, e dão de mamar à criança de ordinário ano e meio, sem lhe
darem de comer outra cousa; amam os filhos extraordinariamente, e
trazem"-nos metidos nuns pedaços de redes que chamam 
\textit{typoya}\footnote{ \textit{Typoia, tipoia} ou \textit{tupya}: originariamente
este termo tupi, que ocorre a primeira vez num texto português com
Cardim, significava uma espécie de rede que as índias utilizavam para
transportar às costas os seus filhos. Mais tarde, passou a designar um
vestido sem mangas utilizado pelas índias já catequizadas,
particularmente nas cerimônias religiosas; uma rede de dormir, ou ainda
uma pequena rede onde se faziam transportar homens ou mulheres, uma
espécie de palanquim.} e os levam às roças e a todo o gênero
de serviço, às costas, por frios e calmas, e trazem"-nos como ciganas
escanchados no quadril, e não lhes dão nenhum gênero de castigo. Para
lhes não chamarem os filhos\footnote{ No manuscrito inglês publicado
por Samuel Purchas aparece a expressão: ``Para que os seus filhos não
chorem\ldots{}''. Tradução da autora.} têm muitos agouros, porque lhe põem
algodão sobre a cabeça, penas de pássaros e paus, deitam"-nos sobre as
palmas das mãos, e roçam"-nos por elas para que cresçam. Estimam mais
fazerem bem aos filhos que a si próprios, e agora estimam muito e amam
os padres, porque lhos criam e ensinam a ler, escrever e contar, cantar
e tanger, cousas que eles muito estimam.

\paragraph[Do costume que têm em agasalhar os hóspedes]{Do costume que têm em 
agasalhar os hóspedes\protect\footnote{ Fernão Cardim descreve o hábito da
saudação lacrimosa praticada pelos índios quando recebiam algum
hóspede e que é descrita pela grande maioria dos cronistas e viajantes
que com eles entraram em contato. Jean de Léry apresenta mesmo na
sua obra \textit{Viagem à Terra do Brasil} algumas gravuras que
coincidem com a descrição cardiniana. Estas saudações lacrimosas eram
também praticadas durante os rituais funerários. Cf. Alfred Métraux,
\textit{La Religion des Tupinambá}, Paris, Leroux, 1928, pp. 180--188.}}

Entrando"-lhe algum hóspede pela casa a honra que lhe fazem é
chorarem"-no: entrando, pois, logo o hóspede na casa o assentam na rede,
e depois de assentado, sem lhe falarem, a mulher e filhas e mais
amigas se assentam ao redor, com os cabelos baixos, tocando com a mão
na mesma pessoa, e começam a chorar em altas vozes, com grande
abundância de lágrimas, e ali contam em prosas trovadas quantas cousas
têm acontecido desde que se não viram até aquela hora, e outras muitas
que imaginam, e trabalhos que o hóspede padeceu pelo caminho, e tudo o
mais que pode provocar a lástima e choro. O hóspede neste tempo não
fala palavra, mas depois de chorarem por bom espaço de tempo limpam as
lágrimas, e ficam tão quietas, modestas, serenas e alegres que parece
nunca choraram, e logo se saúdam, e dão o seu 
\textit{Ereiupe},\footnote{ A expressão tupi \textit{Ereiupe}, que o Pe. Fernão
Cardim, tal como a maioria dos cronistas que deram conhecimento dos
hábitos dos índios do Brasil, apresenta várias vezes ao longo dos seus
textos, significa ``Vieste?'', que se decompõe em \textit{erê} = ``tu'' +
\textit{júr =} do verbo \textit{aju} = ``vieste'' + \textit{pe} =
partícula interrogativa que forma a dicção ``Tu vieste?''. Era a forma
de saudação dos povos Tupis.} e lhe trazem de comer etc.; e
depois destas cerimônias contam os hóspedes ao que vêm. Também os
homens se choram uns aos outros, mas é em casos alguns graves, como
mortes, desastres de guerras etc.; têm por grande honra agasalharem a
todos e darem"-lhe todo o necessário para sua sustentação, e algumas
peças, como arcos, flechas, pássaros, penas e outras cousas, conforme a
sua pobreza, sem algum gênero de estipêndio.

\paragraph{Do costume que têm em beber fumo}

Costumam estes gentios beber fumo de \textit{petigma},\footnote{ \textit{Petigma, 
petume, apty, petym, betum} ou \textit{petum} são as várias designações tupis para o tabaco, o fumo,
considerado como uma erva santa. Vide nota supra no \textit{Tratado do
Clima e Terra do Brasil}\ldots{}, cap. \textsc{xi}, ``Das ervas que servem
para mezinhas''.} por outro nome erva santa; esta
secam e fazem de uma folha de palma uma \textit{canguera},\footnote{ \textit{Idem, ibidem.}} 
que fica como canudo de cana cheio desta erva, e pondo"-lhe o fogo na ponta metem o mais grosso na boca, e
assim estão chupando e bebendo aquele fumo, e o têm por grande mimo e
regalo, e deitados em suas redes gastam em tomar estas fumaças parte
dos dias e das noites. A alguns faz muito mal, e os atordoa e embebeda;
a outros faz bem e lhes faz deitar muitas reimas pela boca. As mulheres
também o bebem, mas são as velhas e enfermas, porque é ele muito
medicinal, principalmente para os doentes de asma, cabeça ou estômago,
e daqui vem grande parte dos portugueses beberem este fumo, e o têm por
vício, ou por preguiça, e imitando os índios gastam nisso dias e noites.

\paragraph{Do modo que têm em fazer suas roçarias e como pagam uns aos outros}

Esta nação não tem dinheiro com que possam satisfazer aos
serviços que lhes fazem, mas vivem \textit{comutatione rerum}\footnote{ Expressão latina que significa ``da troca das coisas''.} e
principalmente a troco de vinho fazem quanto querem, assim quando
hão"-de fazer algumas cousas, fazem vinho e avisando os vizinhos, e
apelidando toda a povoação lhes rogam os queiram ajudar em suas roças,
o que fazem de boa vontade, e trabalhando até as 10 horas tornam para
suas casas a beber os vinhos, e se aquele dia se não acabam as
roçarias, fazem outros vinhos e vão outro dia até as 10 horas acabar
esse serviço; e deste modo usam os brancos prudentes, e que sabem a
arte e maneira dos índios, e quanto fazem por vinho, por onde lhes
mandam fazer vinhos, e os chamam às suas roças e canaviais, e com isto lhes pagam.

 Também usam por ordinário, por troco de algumas cousas, de contas
brancas que se fazem de búzios, e a troco de alguns ramais dão até as
mulheres, e este é o resgate ordinário de que usam os brancos para lhes
comprarem os escravos e escravas que têm para comer.

\paragraph{Das jóias e metaras}

Usam estes índios ordinariamente, principalmente nas festas
que fazem, de colares de búzios, de diademas de penas e de umas
\textit{metaras},\footnote{ \textit{Metaras} eram as pedras que os
índios colocavam nos lábios como adorno. Utilizavam a resina de
jatobá ou a madeira de gameleiras como a ubiragara, a apeíba e a
embaúba na confecção de adornos, nomeadamente do \textit{botoque}, que
era uma rodela de madeira introduzida nos furos artificiais dos lóbulos
da orelha ou no lábio inferior. Fabricavam ainda adornos com resina,
osso, e sobretudo com pedras coloridas, como o quartzo, de preferência
verde, com os quais produziam o \textit{tembetá}, ``pedra do beiço'', que
era um adorno labial em forma de T, usado exclusivamente pelos
homens. O termo tupi ocorre pela primeira vez num texto
português com Cardim e significa ``adorno'', ``enfeite'', o que condiz com
a descrição cardiniana. Cf. André Prous, \textit{Arqueologia
Brasileira}, Brasília, 1992, p. 244, cit. in Jorge Couto, \textit{op. cit.}, 
p. 88.} pedras que metem no beiço de baixo, verdes, brancas,
azuis, muitas finas e que parecem esmeraldas ou cristal, são redondas e
algumas tão compridas que lhe dão pelos peitos, e ordinário é em os
grandes principais terem um palmo e mais de comprimento. Também usam de
manilhas brancas dos mesmos búzios, e nas orelhas metem umas pedras
brancas de comprimento de um palmo e mais, e estes outros semelhantes
são os arreios com que se vestem em suas festas, quer sejam em matanças
dos contrários, quer de vinhos, e estas são as riquezas que mais
estimam que quanto têm.

\paragraph{Do tratamento que fazem às mulheres e como as \mbox{escudeiram}}

Costumam estes índios tratar bem às mulheres, nem lhes dão
nunca, nem pelejam com elas, tirando em tempo de vinhos, porque então
de ordinário se vingam delas, dando por desculpa depois o vinho que
beberam e logo ficam amigos como dantes, e não duram muito os ódios
entre eles, sempre andam juntos e quando vão fora a mulher vai detrás
e o marido diante para que se acontecer alguma cilada não caia a mulher
nela, e tenha tempo para fugir enquanto o marido peleja com o
contrário etc., mas à tornada da roça ou qualquer outra parte vem a
mulher diante, e o marido detrás, porque como tenha já tudo seguro, se
acontecer algum desastre possa a mulher que vai diante fugir para casa,
e o marido ficar com os contrários, ou qualquer outra cousa. Porém, em
terra segura ou dentro na povoação sempre a mulher vai diante, e o
marido detrás, porque são ciosos e querem sempre ver a mulher.

\paragraph{Dos seus bailos e cantos}

Ainda que são melancólicos, têm seus jogos, principalmente os
meninos, muito vários e graciosos, em os quais arremedam muitos gêneros
de pássaros, e com tanta festa e ordem que não há mais que pedir, e os
meninos são alegres e dados a folgar e folgam com muita quietação e
amizade, que entre eles não se ouvem nomes ruins, nem pulhas, nem
chamarem nomes aos pais e mães, e raramente quando jogam se
desconcertam, nem desavêm por causa alguma, e raramente dão uns nos
outros, nem pelejam; de pequeninos os ensinam os pais a bailar e cantar
e os seus bailos não são diferenças de mudança, mas é um contínuo bater
de pés estando quedos, ou andando ao redor e meneando o corpo e a
cabeça, e tudo fazem por tal compasso, com tanta serenidade, ao som de
um cascavel feito ao modo dos que usam os meninos em Espanha, com
muitas pedrinhas dentro ou umas certas sementes de que também fazem
muito boas contas, e assim bailam cantando juntamente, porque não fazem
uma cousa sem a outra, e têm tal compasso e ordem, que às vezes cem
homens bailando e cantando em carreira, enfiando uns detrás dos outros,
acabam todos juntamente uma pancada, como se estivessem todos em um
lugar; são muito estimados entre eles os cantores, assim homens como
mulheres, em tanto que se tomam um contrário bom cantor e inventor de
trovas, por isso lhe dão a vida e não no comem nem aos filhos. As
mulheres bailam juntamente com os homens, e fazem com os braços e corpo
grandes gatimanhas e momos, principalmente quando bailam sós. Guardam
entre si diferenças das vozes em sua consonância, e de ordinário as
mulheres levam os tiples, contraltos e tenores. 

\paragraph[Dos seus enterramentos]{Dos seus enterramentos\protect\footnote{ As comunidades ameríndias
preocupavam"-se com os ritos funerários, que se destinavam, por um lado,
a auxiliar o espírito do morto a alcançar o Guajupiá, que era um
paraíso situado para além das montanhas, onde cresciam bosques de
sapucaia, aí se encontrando com os espíritos dos seus antepassados, no
meio de grande abundância, divertindo"-se incessantemente, e, por outro
lado, a proteger a comunidade do seu espectro, uma vez que o morto era
considerado como um inimigo. Cf. Florestan Fernandes, \textit{op.
cit.}, pp. 161--163, cit. in Jorge Couto, \textit{op. cit.}, pp. 112--113.}}

 São muito maviosos e principalmente em chorar os mortos, e logo como
algum morre os parentes se lançam sobre ele na rede e tão depressa que
às vezes os afogam antes de morrer, parecendo"-lhes que está morto, e os
que se não podem deitar com o morto na rede se deitam no chão dando
grandes baques, que parece milagre não acabarem com o mesmo morto, e
destes baques e choros ficam tão cortados que às vezes morrem. Quando
choram dizem muitas lástimas e mágoas, e se morre a primeira noite,
toda ela em peso choram em alta voz, que é espanto não cansarem.

 Para estas mortes e choros chamam os vizinhos e parentes, e se é
principal, ajunta"-se toda a aldeia a chorar, e nisto têm também seus
pontos de honra, e aos que não choram lançam pragas, dizendo que não
hão"-de ser chorados: depois de morto o lavam, e pintam muito galante,
como pintam os contrários, e depois o cobrem de fio de algodão que não
lhe parece nada, e lhe metem uma \textit{cuya}\footnote{ Vide nota
supra, \textit{cuia.}} no rosto, e assentando o metem em um
pote que para isso têm debaixo da terra, e o cobrem de terra,
fazendo"-lhe uma casa, aonde todos os dias lhe levam de comer, porque
dizem que como cansa de bailar, vem ali comer, e assim os vão chorar
por algum tempo todos os dias seus parentes, e com ele metem todas as
suas jóias e \textit{metaras}, para que as não veja ninguém, nem se
lastime; mas o defunto tinha alguma peça, como espada etc., que lhe
haviam dado, torna a ficar do que lha deu, e a torna onde quer que a
ache, porque dizem que como um morre perde todo o direito do que lhe tinham dado.

 Depois de enterrado o defunto os parentes estão em contínuo pranto de
noite e de dia, começando uns, e acabando outros; não comem senão de
noite, armam as redes junto dos telhados, e as mulheres ao segundo dia
cortam os cabelos, e dura este pranto toda uma lua, a qual acabada
fazem grandes vinhos para tirarem o dó, e os machos se tosquiam, e as
mulheres se enfeitam tingindo"-se de preto, e estas cerimônias e outras
acabadas, começam a comunicar uns com os outros, assim homens como as
mulheres; depois de lhes morrerem seus companheiros, algumas vezes, não
tornam a casar, nem entram em festas de vinhos, nem se tingem de preto,
porém isto é raro entre eles, por serem muito dados a mulheres, e não
podem viver sem elas.

\paragraph{Das ferramentas de que usam}

 Antes de terem conhecimento dos portugueses usavam de ferramentas e
instrumentos de pedra, osso, pau, canas, dentes de animal etc., e com
estes derrubavam grandes matos com cunhas de pedra, ajudando"-se do
fogo;\footnote{ Os grupos tribais da floresta tropical praticavam a
horticultura de raízes ou \textit{agricultura de} \textit{coivara}, 
``ramos secos que ficavam nas terras depois de roçadas'', complexo
cultural caracterizado pela utilização dos meios vegetativos de
reprodução, ou seja, pelo cultivo através de mudas e não por semeadura.
Até porque, atendendo à fraca potencialidade agrícola da maioria das
regiões tropicais úmidas, geralmente pouco férteis e com elevado teor
de alumínio, a par da falta de fertilizantes de origem animal, as
populações tiveram que desenvolver um modelo agrícola adaptado às
características ecológicas do seu \textit{habitat} e baseado na
exploração temporária de uma parcela da mata. Cf. Jorge Couto, 
\textit{op. cit.}, pp. 65--70. O termo tupi \textit{coivara} ocorre pela
primeira vez num texto português em 1607, com o Pe. Jerónimo
Rodrigues, na \textit{Relação. Missão dos Carijós. Relação do P.
Jerónimo Rodrigues}, pub. por Serafim Leite, in \textit{Novas Cartas
Jesuíticas}, 1940, pp. 196--246, em que escreve: ``E como as
árvores são pequenas e pau mole, facilmente fazem sua roça, a qual,
acabante de a queimarem, logo plantam, sem fazerem coibara nem fazerem
covas para a mandiba\ldots{}''.} assim mesmo cavavam a terra com paus agudos
e faziam suas \textit{metaras}, contas de búzios, arcos e flechas tão
bem feitos como agora fazem, tendo instrumentos de ferro, porém
gastavam muito tempo a fazer qualquer cousa, pelo que estimam muito o
ferro pela facilidade que sentem em fazer suas cousas com ele, e esta é
a razão por que folgam com a comunicação dos brancos.

\paragraph{Das armas de que usam}

As armas deste gentio o ordinário são arcos e flechas e deles
se honram muito, e os que fazem de boas madeiras, e muito galantes,
tecidos com palma de várias cores, e lhes tingem as cordas de verde ou
vermelho, e as flechas fazem muito galantes, buscando para elas as mais
formosas penas que acham; fazem estas flechas de várias canas, e na
ponta lhes metem dentes de animais ou umas certas canas muito duras e
cruéis, ou uns paus agudos com muitas farpas, e às vezes as ervas com peçonha. 

 Estas flechas ao parecer, parece cousa de zombaria, porém é arma cruel;
passam umas couraças de algodão, e dando em qualquer pau o abrem pelo
meio, e acontece passarem um homem de parte a parte, e ir pregar no
chão; exercitam"-se de muito pequenos nestas armas, e são grandes
frecheiros e tão certeiros que lhes não escapa passarinho por pequeno
que seja, nem bicho do mato, e não tem mais que quererem meter uma
flecha por um olho de um pássaro, ou de um homem, ou darem em qualquer
outra cousa, por pequena que seja, que o não façam muito ao seu alvo, e
por isso são muito temidos, e tão intrépidos e ferozes que mete
espanto. São como bichos do mato, porque entram pelo sertão a caçar
despidos e descalços sem medo nem temor algum.

 Veem sobre maneira, porque à légua enxergam qualquer cousa, e da mesma
maneira ouvem; atinam muito; regendo"-se pelo sol, vão a todas as partes
que querem, duzentas e trezentas léguas, por matos espessos sem errar
ponto, andam muito, e sempre, de galope, e principalmente com cargas,
nenhum a cavalo os pode alcançar; são grandes pescadores e nadadores,
nem temem mar, nem ondas, e aturam um dia e noite nadando, e o mesmo
fazem remando e às vezes sem comer.

 Também usam por armas de espadas de pau, e os cabos delas tecem de
palma de várias cores e os empenam com penas de várias cores,
principalmente em suas festas e matanças: estas espadas são cruéis,
porque não dão ferida, mas pisam e quebram a cabeça a um homem sem
haver remédio de cura.

\paragraph{Do modo que este gentio tem acerca de matar e comer carne humana}

De todas as honras e gostos da vida, nenhum é tamanho para
este gentio como matar e tomar nomes nas cabeças de seus 
contrários,\footnote{ A antropofagia, ou seja, o hábito de comer carne humana sob
várias modalidades, verificou"-se em quase todos os povos ameríndios,
com maior destaque entre os Tupis, designadamente Potiguares, Caetés,
Tupinambás, Tupiniquins e Tamoios. Os ritos antropofágicos, que eram
centrais na cultura tupi, obedeciam a regras comuns à generalidade dos
grupos tribais do litoral. Revestia"-se de caráter exclusivamente
ritual, ainda que seja encarada de outras formas pela historiografia
mais recente. As notícias fornecidas pelos textos quinhentistas e
seiscentistas relatam a importância na organização social indígena,
como sendo um fator indispensável aos ritos de nominação e iniciação.
As sociedades indígenas eram estruturadas em função da guerra, os
grupos tribais desenvolveram, por isso, uma escala de estratificação
social em que o valor e importância baseavam"-se fundamentalmente na
capacidade de perseguir e matar o maior número de inimigos. Mesmo para
aquele que morria e que era comido, era uma honra perecer como um
guerreiro, era uma passagem para o Além de uma forma gloriosa,
preferível à morte por doença, à morte natural. Veja"-se a ``Introdução''
a esta obra. Cf. Eduardo Viveiros de Castro, \textit{Arawetê. Os
Deuses Canibais}, Rio de Janeiro, 1986, pp. 42--44; Florestan Fernandes,
\textit{A Função Social da Guerra na Sociedade Tupinambá}, pp. 316--349;
Alfred Métraux, \textit{op. cit}, pp. 114--147 e Mário Maestri, ``Considerações 
sobre a Antropofagia Cerimonial e Alimentar Tupinambá'',
in \textit{Anais da \textsc{x} Reunião da Sociedade Brasileira de Pesquisa
Histórica}, (Curitiba), \textsc{x} (1991), p. 118 e ainda do mesmo autor,
\textit{A Terra dos Males sem Fim. A Agonia Tupinambá no Litoral
Brasileiro, Século \textsc{xvi}}, Porto Alegre"-Bruxelas, 1190--1991, pp. 44--55 e
Jorge Couto, \textit{op. cit.}, pp. 101--109.} nem entre eles há festas
que cheguem às que fazem na morte dos que matam com grandes cerimônias,
as quais fazem desta maneira. Os que tomados na guerra vivos são
destinados a matar, vêm logo de lá com um sinal, que é uma cordinha
delgada ao pescoço, e se é homem que pode fazer fugir traz uma mão
atada ao pescoço debaixo da barba, e antes de entrar nas povoações que
há pelo caminho os enfeitam, depenando"-lhes as pestanas e sobrancelhas
e barbas, tosquiando"-os ao seu modo, e empenando"-os com penas amarelas
tão bem assentadas que lhes não aparece cabelo: as quais os fazem tão
lustrosos como aos Espanhóis os seus vestidos ricos, e assim vão
mostrando sua vitória por onde quer que passam. Chegando à sua terra,
o saem a receber as mulheres gritando e juntamente dando palmadas na
boca, que é recebimento comum entre eles, e sem mais outra vexação ou
prisão, salvo que lhes tecem no pescoço um colar redondo como corda de
boa grossura, tão dura como pau, e neste colar começam de urdir grande
número de braças de corda de comprimento de cabelos de mulher,
arrematada em cima com certa volta, e solta em baixo, e assim vai toda
de orelha a orelha por detrás das costas e ficam com esta coleira uma
horrenda cousa; e se é fronteiro e pode fugir, lhe põem em lugar de
grilhões por baixo dos giolhos uma pea de fio de tecido muito apertada,
a qual para qualquer faca fica fraca, se não fossem as guardas que
nenhum momento se apartam dele, quer vá pelas casas, quer para o mato,
ou ande pelo terreiro, que para tudo tem liberdade, e comumente a
guarda é uma que lhe dão por mulher, e também para lhe fazer de comer,
o qual se seus senhores lhe não dão de comer, como é costume, toma um
arco e flecha e atira à primeira galinha ou pato que vê, de quem quer
que seja, e ninguém lhe vai à mão, e assim vai engordando, sem por isso
perder o sono, nem o rir e folgar como os outros, e alguns andam tão
contentes com haverem de ser comidos, que por nenhuma via consentiram
ser resgatados para servir, porque dizem que é triste cousa de morrer,
e ser fedorento e comido de bichos. Estas mulheres são comumente nesta
guarda fiéis, porque lhes fica em honra, e por isso são muitas vezes
moças e filhas de príncipe, \textit{maxime} se seus irmãos hão"-de ser os
matadores, porque as que não têm estas obrigações muitas vezes se
afeiçoam a eles de maneira que somente lhes dão azo para fugirem, mas
também se vão com eles; nem elas correm menos riscos se as tornam que
de levarem umas poucas de pancadas, e às vezes são comidas dos mesmos a
quem deram a vida.

 Determinado o tempo que há de morrer, começam as mulheres a fazer
louça, a saber: panelas, alguidares, potes para os vinhos, tão grandes
que cada um levará uma pipa; isto prestes, assim os principais como os
outros mandam seus mensageiros a convidar outros de diversas partes
para tal lua, até dez, doze léguas e mais, para o qual ninguém se
escusa. Os hóspedes vêm em magotes com mulheres e filhos, e todos
entram no lugar com danças e bailos, e em todo o tempo em que se junta
a gente, há vinho para os hóspedes, porque sem ele todo o mais
agasalhado não presta; a gente junta, começam as festas alguns dias
antes, conforme ao número, e certas cerimônias que precedem, e cada uma
gasta um dia.

 Primeiramente têm eles para isto umas cordas de algodão de arrazoada
grossura, não torcidas, se não tecidas de um certo lavor galante; é
cousa entre eles de muito preço, e não nas têm senão alguns principais,
e segundo elas são primas, bem feitas, e eles vagarosos, é de crer que
nem um ano se fazem: estas estão sempre muito guardadas, e levam"-se ao
terreiro com grande festa e alvoroço dentro de uns alguidares, onde
lhes dá um mestre disto dois nós, por dentro dos quais com força corre
uma das pontas de maneira que lhes fica bem no meio um laço; estes nós
são galantes e artificiosos, que poucos se acham que os saibam fazer,
porque têm algumas dez voltas e cinco vão por cima das outras cinco,
como se um atravessasse os dedos da mão direita por cima dos da
esquerda, e depois atingem com um polme de um barro como cal e
deixam"-nas enxugar.

 O segundo dia trazem muito feixes de canas bravas de comprimento de 
lanças e mais, e à noite põem"-nos em roda em pé, com as pontas para
cima, encostados uns aos outros, e pondo"-lhes ao fogo se faz uma
formosura e alta fogueira, ao redor da qual andam bailando homens e
mulheres com maços de flechas ao ombro, mas andam muito depressa,
porque o morto que há"-de ser, que os vê melhor do que é visto por causa
do fogo, atira com quanto acha, e quem leva, leva, e como são muitos,
poucas vezes erra.

 Ao terceiro dia fazem uma dança de homens e mulheres, todos com gaitas
de canas e batem todos à uma no chão ora com um pé, ora com outro, sem
discreparem, juntamente e ao mesmo compasso assopram os canudos, e não
há outro cantar nem falar, e como são muitos e as canas umas mais
grossas, outras menos, além de atroarem os matos, fazem uma harmonia
que parece música do inferno, mas eles aturam nelas como se fossem as
mais suaves do mundo; e estas são suas festas, afora outras que
entremetem com muitas graças e adivinhações.

 Ao quarto dia, em rompendo a alva, levam o contrário a lavar a um rio,
e vão"-se detendo para que, quando tornarem, seja já dia claro, e
entrando pela aldeia, o preso vai já com olho sobre o ombro, porque não
sabe de que casa ou porta lhe há"-de sair um valente que o há"-de ferrar
por detrás, porque, como toda sua bem"-aventurança consiste em morrer
como valente, e a cerimônia que se segue é já das mais propínquas à
morte, assim como o que há"-de aferrar mostra suas forças em só ele o
subjugar sem ajuda de outrem, assim ele quer mostrar ânimo e forças em
lhe resistir; e às vezes o faz de maneira que, afastando"-se o primeiro
como cansado em luta, lhe sucede outro que se tem por mais valente
homem, os quais às vezes ficam bem enxovalhados, e mais o ficariam, se
já a este tempo o cativo não tivesse a pea ou grilhões. Acabada esta
luta ele em pé, bufando de birra e cansaço com o outro que o tem
aferrado, sai com coro de ninfas que trazem um grande alguidar novo
pintado, e nele as cordas enroladas e bem alvas, e posto este presente
aos pés do cativo, começa uma velha como versada e mestra do coro a
entoar uma cantiga que as outras ajudam, cuja letra é conforme a
cerimônia, enquanto elas cantam os homens tomam as cordas, e metido o
laço no pescoço lhe dão um nó simples junto dos outros grandes, para
que se não possa mais alargar, e feita de cada ponta uma roda de dobras
as metem no braço à mulher que sempre anda detrás dele com este peso e
se o peso é muito pelas cordas serem grossas e compridas, dão"-lhe outra
que traga uma das rodas, e se ele dantes era temeroso com a coleira,
mais o fica com aqueles dois nós tão grandes no pescoço da banda
detrás, e por isso diz um dos pés de cantiga: \textit{nós somos aquelas
que fazemos estirar o pescoço ao pássaro}, posto que depois de outras
cerimônias lhe dizem noutro pé: \textit{Si tu foras papagaio, voando nos fugiras.} 

 A este tempo estão os potes de vinho postos em carreira pelo
meio de uma casa grande, e como a casa não tem repartimentos, ainda
que seja de 20 ou 30 braças de comprido, está atulhada de gente, e 
tanto que começam a beber é um lavarinto ou inferno vê"-los e ouvi"-los,
porque os que bailam e cantam aturam com grandíssimo fervor quantos
dias e noites os vinhos duram: porque, como esta é a própria festa das
matanças, há no beber dos vinhos muitas particularidades que duram
muito, e a cada passo urinam, e assim aturam sempre, e de noite e dia
cantam e bailam, bebem e falam cantando em magotes por toda a casa, de
guerras e sortes que fizeram, e como cada um quer que lhe ouçam a sua
história, todos falam a quem mais alto, afora outros estrondos, sem
nunca se calarem, nem por espaço de um quarto de hora.

 Aquela manhã que começam a beber enfeitam o cativo por um modo
particular que para isso têm a saber: depois de limpo o rosto, e quanta
penugem nele há, o untam com um leite de certa árvore que pega muito,
e sobre ele põem um certo pó de umas cascas de ovo verde de certa ave
do mato, e sobre isto o pintam de preto com pinturas galantes, e
untando também o corpo até a ponta do pé o enchem todo de pena, que
para isto têm já picada e tinta de vermelho, a qual o faz parecer a
metade mais grosso, e a cousa do rosto o faz parecer tanto maior e
luzente, e os olhos mais pequenos, que fica uma horrenda visão, e da
mesma maneira que eles têm pintado o rosto, o está também a espada, a
qual é de pau ao modo de uma palmatória, senão que a cabeça não é tão
redonda, mas quase triangular, e as bordas acabam quase em gume, e
haste, que será de 7 ou 8 palmos, não é tão roliça, terá junto da
cabeça 4 dedos de largura e vem cada vez estreitando até o cabo, onde
tem uns pendentes ou campainhas de pena de diversas cores, é cousa
galante e de preço entre eles, eles lhe chamam
\textit{Ingapenambin},\footnote{ \textit{Ingapenambin} é uma espada
de pau. O termo tupi é formado de \textit{ingape} = ``espada de pau'' +
\textit{nambi} = ``orelha'', daí a expressão de Cardim para designar esta
arma ``orelhas da espada'', que é onde ocorre pela primeira vez este
termo tupi.} orelhas da espada. 

 O derradeiro dia dos vinhos fazem no meio do terreiro uma choça de
palmas ou tantas quantos são os que hão de morrer, e naquela se
agasalha, e sem nunca mais entrar em casa, e em todo o dia e noite é
bem servido de festas mais que de comer, porque lhe dão outro conduto
senão uma fruta que tem sabor de nozes, para que ao outro dia não tenha
muito sangue.

Ao quinto dia pela manhã, ali às sete horas pouco mais ou
menos, a companheira o deixa, e se vai para casa muito saudosa e
dizendo por despedida algumas lástimas pelo menos fingidas; então lhe
tiram a peia e lhe passam as cordas do pescoço à cinta, e posto em pé à
porta do que o há"-de matar, sai o matador em uma dança, feito alvo como
uma pomba com barro branco, e uma a que chamam capa de pena, que se ata
pelos peitos, e ficam"-lhe as abas para cima como asas de Anjo, e nesta
dança dá uma volta pelo terreiro e vem fazendo uns esgares estranhos
com olhos e corpo, e com as mãos arremeda o minhoto que desce à carne,
e com estas diabruras chega ao triste, o qual tem as cordas estiradas
para as ilhargas e de cada parte um que o tem, e o cativo, se acha com
que atirar, o faz de boa vontade, e muitas vezes lhe dão com o que,
porque lhe saem muito valentes, e tão ligeiros em furtar o corpo que
os não pode acertar.

Acabado isto, vem um honrado, padrinho do novo cavaleiro que há"-de
ser, e tomada a espada lha passa muitas vezes por entre as pernas,
metendo"-a ora por uma parte, ora por outra, da própria maneira que os
cachorrinhos dos sanfonineiros, lhe passam por entre as pernas, e
depois tomando"-a pelo meio com ambas as mãos aponta com uma estocada
aos olhos do morto, e isto feito lhe vira a cabeça para cima da maneira
que dela hão"-de usar, e a mete nas mãos do matador, já como apta e
idônea com aquelas bênçãos para fazer o seu ofício para o qual se põe
algum tanto ao lado esquerdo, de tal jeito que com o gume da espada
lhe acerta no toutiço, porque não tira a outra parte, e é tanta a
bruteza destes que, por não temerem outro mal senão aquele presente tão
inteiros estão como se não fosse nada, assim para falar, como para
exercitar as forças, porque depois de se despedirem da vida com dizer
\textit{que muito embora morra, pois muitos tem mortos, e que além
disso cá ficam seus irmãos e parentes para o vingarem}, e nisto
aparelha"-se um para furtar o corpo, que é toda a honra de sua morte. 

E, são nisto tão ligeiros que muitas vezes é alto dia sem o poderem
matar, porque em vindo a espada pelo ar, ora desvia a cabeça, ora lhe
furta o corpo, e são nisto tão terríveis que se os que têm as pontas
das cordas o apertam, como fazem quando o motor é frouxo, eles tão rijo
que os trazem a si e os fazem afrouxar em que lhes pese, tendo um olho
neles e outro na espada, sem nunca estarem quedos, e como matador os
não pode enganar ameaçando sem dar, sob pena de lhe darem uma apupada,
e eles lhe adivinham o golpe de maneira que, por mais baixo que venha,
num assopro se abatem e fazem tão rasos que é cousa estranha, e não é
menos tomarem a espada aparando"-lhe o braço por tal arte que se lhe
fazerem nada correm com ela juntamente para baixo e a metem debaixo do
sovaco tirando pelo matador, ao qual, se então não acudissem, o outro
despacharia, porque têm eles neste ato tantos agouros que para matar
um menino de cinco anos tão enfeitados como para matar algum gigante, e
com estas ajudas ou afouteza tantas vezes dá, até que acerta algumas e
esta basta, porque tanto que ele cai lhe dá tantas que lhe quebra a
cabeça, posto que já se viu um que a tinha tão dura, que nunca lha
puderam quebrar, porque como a trazem sempre descoberta, têm as cabeças
tão duras que as nossas em comparação delas ficam como de cabaças, e
quando querem injuriar algum branco lhe chamam cabeça mole.

 Se este que mataram cai de costas, e não de bruços, têm"-no por grande
agouro e prognóstico que o matador há"-de morrer, e ainda que caia de
bruços têm muitas cerimônias, as quais se se não guardam têm para si
que o matador não pode viver; e são muitas delas tão penosas que se
alguém por amor de Deus sofresse os seus trabalhos não ganharia pouco,
como abaixo se dirá.

 Morto o triste, levam"-no a uma fogueira que para isto está prestes, e
chegando a ela, em lhe tocando com a mão dá uma pelinha pouco mais
grossa que véu de cebola, até que todo fica mais limpo e alvo que nem
um leitão pelado, e então se entrega ao carniceiro ou magarefe, o qual
lhe faz um buraco abaixo do estômago, segundo seu estilo, por onde os
meninos primeiro metem a mão e tiram as tripas, até que o magarefe
corta por onde quer, e o que lhe fica na mão é o quinhão de cada um, e
o mais se reparte pela comunidade, salvo algumas partes principais que
por honra, se dão aos hóspedes mais honrados, as quais eles levam muito
assadas, de maneira que não se corrompam, e sobre elas depois em suas
terras fazem festas e vinhos de novo. 

\paragraph{Das cerimônias que se fazem ao novo cavaleiro}

Acabando o matador de fazer seu ofício, lhe fazem a ele outro
desta maneira: tirada a capa de pena, e deixada a espada, se vai para
casa, à porta da qual o está esperando o mesmo padrinho que foi com um
arco de tirar a mão, a saber, as pontas uma no lumiar de baixo e a
outra em cima, e tirando pela corda como quem quer atirar, o matador
passa por dentro tão sutilmente que não toca em nada, e em ele
passando, o outro alarga a corda com um sinal de pesar, porque errou o
a que atirava, como que aquilo tem virtude para depois da guerra o
fazer ligeiro, e os inimigos o errarem; como é dentro começa de ir
correndo por todas as casas, e as irmãs e primas da mesma maneira
diante dele dizendo: ``meu irmão se chama \textit{N}'', repetindo por
toda a aldeia, e se o Cavaleiro tem alguma cousa boa, quem primeiro
anda lha toma até ficar sem nada. 

 Isto acabado tem pelo chão lançados certos paus de pilão, sobre os
quais ele está em pé aquele dia com tanto silêncio, como que dera o
pasmo nele, e levando"-lha ali a apresentar a cabeça do morto, tiram"-lhe
um olho, e com as raízes ou nervos dele lhe untam os pulsos, e cortada
a boca inteira lha metem no braço como manilha, depois se deita na sua
rede como doente, e na verdade ele o está de medo, que se não cumprir
perfeitamente todas as cerimônias, o há"-de matar a alma do morto. Dali
a certos dias lhe dão o hábito, não no peito do pelote, que ele não
tem, senão na própria pele, sarrafaçando"-o por todo o corpo com um
dente de cutia\footnote{ \textit{Cutia} ou \textit{acutia} é o nome
comum a duas espécies de mamífero roedor já descrito por Cardim no
\textit{Tratado} referente ao clima e terra, no capítulo \textsc{i} ``Dos
Animais''.} que se parece com dente de coelho, o qual, assim por sua
pouca sutileza, como por eles terem a pele dura, parece que rasgam
algum pergaminho, e se eles são animosos não lhe dão as riscas
direitas, senão cruzadas, de maneira que ficam uns lavores muito
primos, e alguns gemem e gritam com as dores.

 Acabado isto, tem carvão moído e sumo de erva moura com que eles
esfregam as riscas ao través, fazendo"-as arreganhar e inchar, que é
ainda maior tormento, e em quanto lhe saram as feridas que duram alguns
dias, está ele deitado na rede sem falar nem pedir nada, e para não
quebrar o silêncio tem a par de si água e farinha e certa fruta como
amêndoas, que chamam \textit{mendobis},\footnote{ O nome
\textit{mendobis} ocorre no texto de Purchas \textit{amenduins} e 
trata"-se do \textit{amendoim} que é o nome comum a diversas plantas da
família das Leguminosas, nomeadamente o \textit{Arachis hypogaea}, que
é originária do Brasil e que possui mais de 30 espécies. O termo tupi
ocorre pela primeira vez num texto português com Cardim. A semente do
amendoim, consumida crua ou torrada, é rica em proteínas e substâncias
oleaginosas, possuindo um teor de gordura extremamente elevado,
constituindo um alimento muito energético e rico em proteínas que
minorava as carências proteicas e corrigia uma alimentação
excessivamente baseada em glúcidos, como a mandioca, batata"-doce e
cará. Cf. Carl O. Sauer, ``As Plantas Cultivadas na América do Sul
Tropical'', in \textit{Suma Etnológica Brasileira}, 
1. \textit{Etnobiologia}, coord. de Berta G. Ribeiro, 2ª ed., Petrópolis,
1987, pp. 64--65, cit. in Jorge Couto, \textit{op. cit.}, p. 69.} 
porque não prova peixe nem carne aqueles dias.

 Depois de sarar, passados muitos dias ou meses, se fazem grandes vinhos
para ele tirar o dó e fazer o cabelo, que até ali não fez, e então, se
tinge de preto, e dali por diante fica habilitado para matar sem
fazerem a ele cerimônia que seja trabalhosa, e ele se mostra também
nisso honrado ou ufano, e com certo desdém, como quem tem honra, e não
ganha de novo, e assim não faz mais que dar ao outro um par de
pancadas, ainda que a cabeça fique inteira e ele bulindo, vai"-se para
casa, e a este acodem logo a lhe cortar a cabeça, e as mães com os
meninos ao colo lhe dão os parabéns, estreiam"-os para a guerra
tingindo"-lhes os braços com aquele sangue.

 Estas são as façanhas, honras, valentias, em que estes gentios tomam
nomes de que se prezem muito, e ficam dali por diante \textit{Abaétés,
Murubixaba, Moçacara},\footnote{ Qualquer um destes nomes
\textit{Abaétés, Murubixaba, Morubixaba} e \textit{Moçacara}
designava categoria e posição entre os membros dos grupos tribais.
Eram os homens bons, ilustres e honrados, entre os indígenas. Qualquer
um destes nomes ocorre pela primeira vez num texto português com Fernão
Cardim. No caso concreto de \textit{Murubixaba} ou \textit{Morubixaba}
significa sobretudo ``chefe'', mas que como já verificamos em nota
supra, estes líderes desempenhavam as suas funções com base na
persuasão, não podendo recorrer à ameaça do uso da força. Cf.~Jorge
Couto, \textit{op. cit.}, pp. 95--97.} que são títulos e nomes
de cavaleiros: e estas são as infelizes festas, em que estes tristes
antes de terem conhecimento de seu Criador põem sua felicidade e glória.

\paragraph[Da diversidade de nações e línguas]{Da diversidade de nações e línguas\protect\footnote{ O
Padre Fernão Cardim, neste seu \textit{Tratado} sobre os índios, faz
uma exaustiva enumeração de várias nações ameríndias, cerca de cento e
quatro, a grande maioria delas não mencionadas nos textos quinhentistas
e seiscentistas, nem posteriormente. Na opinião de Baptista Caetano,
seria um trabalho muito difícil e, em alguns casos, quase impossível de
procurar encontrar uma explicação para todos os nomes apresentados por
Cardim, já que muitos poderão não pertencer ao \textit{abanheenga}
(língua geral) e outros poderão até ser de mera inventiva de algum
narrador, ou mesmo, segundo o Visconde de Porto Seguro, na
\textit{História Geral do Brasil}, tomo \textsc{i}, 1854, p. 101, ``[\ldots{}] meras
alcunhas, com que se designavam as cabildas umas às outras''. Cf.
notas de Baptista Caetano, in Fernão Cardim, \textit{op. cit.}, p. 110.
O nosso autor procura agrupá"-los em dois grandes grupos: os
\textit{Tupis} e os \textit{Tapuias}, com base no tronco linguístico e no
próprio etnocentrismo dos Tupis que, no decurso do processo de luta e
domínio do litoral, venceram as populações que aí habitavam,
considerando"-as como ``selvagens'', no sentido de ``inimigos bárbaros''. 
Esta situação acabou por se transpor para os portugueses, como é
evidente nos textos cardinianos e outros contemporâneos. Os
\textit{Tupis} usavam uma língua que posteriormente se torna geral,
chamada ``língua geral da costa'' e os \textit{Tapuias}, que eram
populações \textit{Jês}, uma língua completamente diferente, de grande
dificuldade de compreensão.}}

 Em toda esta província há muitas nações de diferentes línguas, porém
uma é principal\footnote{ Os ameríndios que se fixaram no espaço
brasílico e nas imediações das suas atuais fronteiras são agrupados,
de acordo com critérios linguísticos, em dois troncos (Macro"-Tupi e
Macro"-Jê); grandes famílias (Caribe, Aruaque e Arauá); famílias menores
situadas a norte do Amazonas (Tucano, Macu e Ianomâmi) e famílias
menores estabelecidas a sul do mesmo rio (Guaicuru, Nambiquara,
Txapacura, Pano, Mura e Catuquina), bem como grupos isolados (Aicaná,
Aricapú, Auaquê, Irántche, Jabuti, Canoê, Coiá, Trumái e outras). Cf.
Aryon Dall'Igna Rodrigues, \textit{Línguas Brasileiras. Para o
Conhecimento das Línguas Indígenas}, São Paulo, 1987, pp. 41--98, cit.
in Jorge Couto, \textit{op. cit.}, pp. 51--56.} que compreende algumas
dez nações de índios: estes vivem na costa do mar, e em uma grande
corda do sertão, porém são todos estes de uma só língua ainda que em
algumas palavras discrepam e esta é a que entendem os portugueses; é
fácil, e elegante, e suave, e copiosa, a dificuldade dela está em ter
muitas composições; porém dos portugueses, quase todos os que vêm do
Reino e estão cá de assento e comunicação com os índios a sabem em
breve tempo, e os filhos dos portugueses, assim homens como mulheres,
principalmente na Capitania de São Vicente, e com estas dez nações de
índios têm os Padres comunicações por lhes saberem a língua, e serem
mais domésticos e bem inclinados: estes foram e são amigos antigos dos
portugueses, com cuja ajuda e armas, conquistaram esta terra, pelejando
contra seus próprios parentes, e outras diversas nações bárbaras e eram
tantos os desta casta que parecia impossível extinguir, porém os
portugueses lhes têm dado tal pressa que quase todos são mortos e lhes
têm tal medo, que despovoam a costa e fogem pelo sertão a dentro até
trezentos a quatrocentas léguas.

 Os primeiros desta língua se chamam \textit{Potiguaras},\footnote{ \textit{Potiguaras}, 
\textit{Potiguares, Potyguara} ou \textit{Potygoar} 
significava etimologicamente ``comedor de camarão'', já
que \textit{poti} = ``camarão'' e \textit{goar} = ``comedor''. Era um grupo
que vivia na zona costeira localizada entre o rio Jaguaribe e o
Paraíba. Eram inimigos dos portugueses, tal como Cardim refere,
atacavam"-lhes as roças e os engenhos, queimando"-os e matando"-os, porque
eram muito guerreiros. Invadiram a região dos Caetés e tomaram"-lhes o
território.} senhores da Paraíba, 30 léguas de Pernambuco,
senhores do melhor pau do Brasil e grandes amigos dos Franceses, e com
eles contrataram até agora, casando com eles suas filhas; mas agora na
era de 84 foi a Paraíba tomada por Diogo Flores,\footnote{ General 
de Filipe \textsc{ii} que expulsou os Franceses da região da Paraíba. Aí
edificou um forte dotado de uma guarnição de 100 soldados, além dos
próprios Portugueses, que tinham como capitão Frutuoso Barbosa.} 
General de Sua Majestade, botando os Franceses fora, e deixou um forte
com cem soldados, afora os portugueses, que também têm seu Capitão e
Governador Frutuoso Barbosa,\footnote{ Frutuoso Barbosa foi um rico
comerciante instalado em Pernambuco e conhecedor das potencialidades
agrícolas das terras paraibanas, obteve em 25 de janeiro de 1579 uma
mercê do cardeal"-rei D. Henrique, de capitão"-mor da conquista da
Paraíba, pelo prazo de dez anos, sob a condição de conquistá"-la e
povoá"-la. Esta provisão foi confirmada, mais tarde, já na Monarquia
Dual, por Filipe \textsc{ii}. Depois de diversas tentativas e duras batalhas,
agravadas pelos temporais e pela aguerrida resistência de Franceses e
Potiguares e já arruinado, tornou"-se governador dessa capitania, na
qual se manteve de 1585 a 1588. Cf. Joaquim Veríssimo Serrão,
\textit{Do Brasil Filipino ao Brasil de 1640}, São Paulo, Companhia
Editora Nacional, 1968, pp. 28--30; J. F. de Almeida Prado, \textit{A
Conquista da Paraíba} (\textit{Séculos \textsc{xvi} a \textsc{xviii}}), São Paulo, Companhia
Editora Nacional, 1964; Jorge Couto, ``As Tentativas Portuguesas de
Colonização do Maranhão e o Projeto da França Equinocial'', in
\textit{A União Ibérica e o Mundo Atlântico}, coord. de Maria da Graça
M. Ventura, Lisboa, Colibri, 1997, pp. 196--197.} que com a principal
gente de Pernambuco levou exército por terra com que venceu os
inimigos, porque do mar os da armada não pelejaram. 

 Perto destes vivia grande multidão de gentio que chamam
\textit{Viatã},\footnote{ \textit{Viatã}: este povo era vizinho dos
Potiguares e segundo o testemunho de Cardim foram extintos devido às 
guerras. O termo tupi apenas ocorre como ``farinha dura'' ou ``farinha
muito torrada''.} destes já não há nenhuns, porque sendo eles
amigos dos \textit{Potiguaras} e parentes, os portugueses os fizeram
entre si inimigos, dando"-lhos a comer, para que desta maneira lhes
pudesse fazer guerra e tê"-los por escravos, e finalmente, tendo uma
grande fome, os portugueses em vez de lhes acudir, os cativaram e
mandaram barcos cheios a vender a outras capitanias: ajuntou"-se a isto
um clérigo português mágico, que com seus enganos os acarretou todos 
a Pernambuco, e assim se acabou esta nação, e ficando os portugueses
sem vizinhos que os defendessem dos \textit{Potiguaras}, os quais até
agora que foram desbaratados, perseguiram os portugueses dando"-lhes de
súbito nas roças, fazendas, e engenhos, queimando"-lhos, e matando muita
gente portuguesa, por serem muito guerreiros; mas já pela bondade de
Deus estão livres deste sobroço. 

 Outros há a que chamam \textit{Tupinabas:}\footnote{ \textit{Tupinabas}
ou \textit{Tupinambás } foi a denominação dada pelos autores dos
séculos \textsc{xvi} e \textsc{xvii} a diversos povos indígenas de língua do tronco
Tupi. O nome queria designar ``descendentes dos Tupis'', habitavam na
região do Rio Real (Sergipe) até junto de Ilhéus, eram considerados
inimigos dos do Camamu e Tinharé. Os estudiosos atuais colocam"-nos na
região costeira entre o rio Parnaíba e o rio Pará, costas do Maranhão e
no litoral desde a margem direita do São Francisco até à zona norte de
Ilhéus, depois de terem derrotado os anteriores habitantes. Estavam
organizados em dois grupos inimigos, o que provocava grandes combates
entre eles. Por outro lado, os moradores da região onde veio a ser
edificada a vila do Pereira e, posteriormente, a cidade do Salvador
eram inimigos dos habitantes das ilhas de Itaparica e Tinharé e da
costa norte de Ilhéus, o que provocava acesos combates entre aqueles
bandos. Cf. Jorge Couto, \textit{A Construção do
Brasil. Ameríndios, Portugueses e Africanos, do início do povoamento a
finais do Quinhentos}, Lisboa, Edições Cosmos, 1995, pp. 56--60.} 
estes habitam do Rio Real até junto dos Ilhéus; estes
entre si eram também contrários, os da Bahia com os do Camamu e Tinharê.

 Por uma corda do Rio de São Francisco vivia outra nação a que chamavam
\textit{Caaété},\footnote{ \textit{Caaété} ou \textit{Caéte} 
literalmente significava ``mato verdadeiro ou real'', mas também é
possível que seja \textit{acaêtê} que significava ``cabeça dura''.
Habitavam entre o Paraíba e a margem norte do rio São Francisco
(Alagoas), de onde fugiram para o interior, acoitando"-se nas serras.
Eram conhecidos pela sua violência e pelas práticas de antropofagia.} 
e também havia contrários entre estes e os de Pernambuco.

 Dos Ilhéus, Porto Seguro até Espírito Santo habitava outra nação, que
chamavam \textit{Tupinaquim};\footnote{ \textit{Tupinaquim} ou
\textit{Tupiniquins} eram descendentes dos Tupis, por isso o seu nome
significa ``colaterais dos Tupis''. Habitavam as zonas do litoral desde o
estuário do Camamu, a norte de Ilhéus, até ao do Cricaré ou São Mateus,
Espírito Santo ou, segundo alguns autores, até o Rio de Janeiro. No
início da colonização aliaram"-se aos portugueses na luta contra os
Franceses. Fizeram também guerra aos Tupinambás, seus inimigos
tradicionais e debateram"-se com as investidas dos \textit{Aimorés}, 
pertencentes à família Botocudo (Macro"-Jê), que lhes disputavam o
território.} estes procederam dos de Pernambuco e se
espalharam por uma corda do sertão, multiplicando grandemente, mas já
são poucos; estes foram sempre muito inimigos das cousas de Deus,
endurecidos em seus erros, porque eram vingativos e queriam vingar"-se
comendo seus contrários e por serem amigos de muitas mulheres. Já
destes há muitos cristãos e são firmes na fé.

 Há outra nação parente destes, que corre do sertão de São Vicente até
Pernambuco, a que chamam \textit{Tupiguae:}\footnote{ \textit{Tupiguae}: 
possivelmente designava ``grupo tribal'' ou ``os de casa'', ``os
caseiros'', ``os domésticos''.} estes eram sem número, vão"-se
acabando, porque os portugueses os vão buscar para se servirem deles, e
os que lhes escapam fogem para muito longe, por não serem escravos. Há
outra nação vizinha a estes, que chamam \textit{Apigapigtanga} e 
\textit{Muriapigtanga.}\footnote{ Os nomes destes povos,
\textit{Apigapistanga} e \textit{Muriapigtanga}, são difíceis de
definir etimologicamente. Segundo alguns especialistas, seriam
possivelmente ``cabilda de selvagens'', de inimigos de Tapuias.} 
Também há outra nação contrária aos \textit{Tupinaquins}, que chamam
\textit{Guaracaio} ou \textit{Itati.}\footnote{ Por este nome
\textit{Guaracaio} ou \textit{Itati}, deverão ser povos originários ou
desgarrados do Peru, na medida em que não constam em nenhum outro autor
e o termo \textit{guaraio} explica"-se pelo quíchua
\textit{huaraca}, como ``funda'', ou o verbo ``atirar com funda'' + o
sufixo \textit{yoc} = \textit{huara"-cayoc} para designar ``aquele que
atira com funda''. Semelhante designação podemos encontrar para o outro
nome apresentado por Cardim, \textit{Itati}, que ocorre em abanheenga,
\textit{Itaitig} como ``atirar pedras'', que se tomarmos como substantivo
fica ``o atirador de pedras''.} 

 Outra nação mora no Espírito Santo a que chamam 
\textit{Tegmegminó}:\footnote{ \textit{Tegmegminó} ou \textit{Temiminó}: o nome significa
``netos do homem''. São os índios vizinhos dos Tamoios, que habitavam as
terras de Ubatuba a São Vicente, designadamente na ilha de Paranapuã ou
dos Maracajás, atual ilha do Governador, na baía da Guanabara, não
conseguindo resistir às investidas dos seus implacáveis
inimigos.} eram contrários dos \textit{Tupinaquins}, mas já
são poucos. Outra nação que se chama \textit{Tamuya},\footnote{ \textit{Tamuya} são usualmente denominados \textit{Tamoios} ou
\textit{Tamoyos}. Eram os habitantes do Rio de Janeiro, do cabo de São
Tomé a Angra dos Reis, com quem os portugueses contataram quando
atingiram as terras brasileiras. O seu nome designava ``avô'', ``avós''.}
moradores do Rio de Janeiro, estes destruíram os portugueses
quando povoaram o Rio, e deles há muito poucos, e alguns que há no
sertão se chamam \textit{Ararape.}\footnote{ \textit{Ararape}: nome
de grupo tribal de difícil identificação.} 

 Outra nação se chama \textit{Carijó}:\footnote{ \textit{Carijós} 
habitavam a sul de São Vicente e eram inimigos dos Tupiniquins de São
Vicente, que se estendiam ao longo do litoral até à lagoa dos Patos,
numa extensão de oitenta léguas e no interior do sertão até ao
Paraguai. Era a região dos Guaranis, que assimilaram ou ``guarinizaram''
os povos autóctones, como os Carijós, além dos Tapes, Patos e Arachãs.} 
habitam além de São Vicente como oitenta léguas, contrários dos
\textit{Tupinaquins} de São Vicente; destes há infinidade e correm pela
costa do mar e sertão até o Paraguai, que habitam os Castelhanos. 

 Todas estas nações acima ditas, ainda que diferentes, e muitas delas
contrárias umas das outras, têm a mesma língua, e nestes se faz a
conversão, e tem grande respeito aos Padres da Companhia e no sertão
suspiram por eles, e lhes chamam \textit{Abarê}\footnote{ \textit{Abaré} 
é o termo tupi que designava ``o padre católico ou
cristão'', sobretudo aos jesuítas. Etimologicamente significa ``o homem''
ou ``pessoa por excelência'', ou simplesmente ``o ilustre'', ``o eminente'',
formado de \textit{baá} = ``homem'' + \textit{ré} = ``diverso''. ``Homem
diferente'', porque se veste de negro (roupeta) e não mantinha relações
com mulheres. O termo tupi ocorre pela primeira vez num texto português
em 1552, numa carta de Leonardo Nunes.} e\textit{
Pai}, desejando a suas terras convertê"-los, e é tanto este crédito que
alguns portugueses de ruim consciência se fingem Padres, vestindo"-se em
roupetas, abrindo coroas na cabeça, e dizendo que são Abarês e que os
vão buscar para as igrejas dos seus pais, que são os nossos, os trazem
enganados, e em chegando ao mar os repartem entre si, vendem e ferram,
fazendo primeiro neles lá no sertão grande mortandade, roubos e saltos,
tomando"-lhes as filhas e mulheres etc., e se não foram estes e
semelhantes estorvos já todos os desta língua foram convertidos à nossa santa fé.

 Há outras nações contrárias e inimigas destas, de diferentes línguas,
que em nome geral se chamam de \textit{Tapuya},\footnote{ \textit{Tapuyas} 
ou \textit{Tapuias} era a designação atribuída aos
membros de todas as outras famílias linguísticas, sem ser Tupi"-Guarani
e sobretudo Jês, que ainda não tinham atingido o estádio
civilizacional desses povos. Uma missiva, datada de 1555, da autoria
de um dos missionários que chegou à Terra de Santa Cruz integrado no
primeiro grupo de Jesuítas (1549) fornece"-nos um exemplo paradigmático
da visão quinhentista lusitana dos Tapuias, ou bárbaros de ``língua
travada'', considerados como ``[\ldots{}] geração de índios bestial e feroz;
porque andam pelos bosques como manadas de veados, nus, com os cabelos
compridos como mulheres; a sua fala\ldots{} mui bárbara e eles mui
carniceiros e trazem flechas ervadas e dão cabo de um homem num
momento''. Cf. \textit{Cartas Jesuíticas, \textsc{ii}. Cartas Avulsas
1550--1568}, Rio de Janeiro, 1931, p. 148, cit. in Jorge Couto,
\textit{op. cit.}, 60.} e também entre si são contrárias;
primeiramente no sertão vizinho aos Tupinaquins habitam os
\textit{Guaimurês},\footnote{ \textit{Guaimurês} habitavam o
interior do sertão baiano. Há autores recentes que consideram que este
povo eram os \textit{Aimorés} ou \textit{Aimbarés}, também conhecidos
como os primeiros pelos ``senhores dos matos selvagens'', pela sua
capacidade de dominar e habitar essas regiões, nomeadamente Minas
Gerais, Bahia e Espírito Santo. Usavam botoques e eram mais altos e
claros que os tupinambás. Eram muito aguerridos e entravam em conflito
com os colonos e com outras nações e grupos tribais.} e tomam
algumas oitenta léguas de costa, e para o sertão quanto querem, são
senhores dos matos selvagens, muito encorpados, e pela continuação e
costume de andarem pelos matos bravos tem os couros muito rijos, e para
este efeito açoutam os meninos em pequenos com uns cardos para se
acostumarem a andar pelos matos bravos; não têm roças, vivem de rapina
e pela ponta da flecha, comem a mandioca crua sem lhes fazer mal, e
correm muito e aos brancos não dão senão de salto, usam de uns arcos
muito grandes, trazem uns paus feitiços muito grossos, para que em
chegando logo quebrem as cabeças. Quando vêm à peleja estão escondidos
debaixo de folhas, e dali fazem a sua e são mui temidos, e não há poder
no mundo que os possa vencer; são muito covardes em campo, e não ousam
sair, nem passam água, nem usam de embarcações, nem são dados a pescar;
toda a sua vivenda é do mato; são cruéis como leões; quando tomam
alguns contrários cortam"-lhe a carne com uma cana de que fazem as
flechas, e os esfolam, que lhes não deixam mais que os ossos e tripas:
se tomam alguma criança e os perseguem, para que lha não tomem viva lhe
dão com a cabeça em um pau, desentranham as mulheres prenhes para lhes
comerem os filhos assados. Estes dão muito trabalho em Porto Seguro,
Ilhéus e Camamu, e estas terras se vão despovoando por sua causa; não
se lhes pode entender a língua. 

 Além destes, e para o sertão e campos de Caátinga vivem muitas nações
Tapuyas, que chamam \textit{Tucanuço},\footnote{ \textit{Tucanuço} ou
\textit{Tacanunu}: não encontramos referência a este povo, nem em
obras contemporâneas, nem posteriores, a não ser o nome mais próximo
que é \textit{Tucano}, mas que era um grupo tribal que habitava o norte
do Amazonas.} estes vivem no sertão do Rio Grande pelo direito
de Porto Seguro; têm outra língua, vivem no sertão antes que cheguem ao
Aquitigpe e chamam"-se \textit{Nacai.}\footnote{ \textit{Nacai}: 
nome de grupo tribal de difícil identificação.} Outros há que
chamam \textit{Oquigtajuba.}\footnote{ \textit{Oquigtajuba}: o mesmo
sucedeu com este grupo.} Há outra nação que chamam
\textit{Pahi}:\footnote{ \textit{Pahi}, \textit{Paayagua, Paiconeca,
Payana} ou \textit{Paipocoa}: com estas designações de Fernão Cardim
não aparece em nenhum autor.} estes se vestem de
pano de algodão muito tapado e grosso como rede, com este se cobrem
como com saio, não tem mangas; têm diferente língua. No Ari há outros
que também vivem no campo indo para o Aquitigpe. Há outros que chamam
\textit{Parahió},\footnote{ \textit{Parahió}: tal como os anteriores
povos, não encontramos referência em outros autores.} é muita
gente e de diferente língua. 

 Outros que chamam \textit{Nhandeju},\footnote{ \textit{Nhandeju}: 
mais um grupo tribal apenas mencionado por Fernão Cardim.} 
também de diferente língua. Há outros que chamam 
\textit{Macutu}.\footnote{ \textit{Macutu}: idêntica situação para este
grupo.} Outros \textit{Napara};\footnote{ \textit{Napara}: o
mesmo sucede em relação a este grupo.} estes têm roças. Outros
que chamam \textit{Cuxaré};\footnote{ \textit{Cuxaré}: este grupo
tribal aparece possivelmente em analogia com outros grupos de nações
diversas, como, por exemplo, \textit{Cuxari, Cuzari, Cossari} do
Amazonas, entre outros nomes, que significavam ``os longínquos'' ou ``os
campeiros'', o que se identifica com o que Cardim afirma sobre estes
povos que viviam no meio do sertão.} estes vivem no meio do
campo do sertão. Outros que vivem para a parte do sertão da Bahia, que
chamam \textit{Guayaná},\footnote{ \textit{Guayaná} ou \textit{Guianás}: 
O Visconde de Porto Seguro procurou na sua \textit{História Geral
do Brasil}, de 1854, explicar este nome como sendo ``gente estimada'', de
\textit{guaya} = ``gente'' + \textit{na} = ``estimado'', o que não tem a
aceitação de etimologistas mais recentes, como Baptista Caetano.
Segundo teorias mais recentes, este grupo tribal devia ter pertencido ao
primeiro grupo da família Jê que migrou, há cerca de 3000 anos e que se
terá fixado na região meridional e serão os antepassados dos atuais
\textit{Caiangangues} ou \textit{Coroados.} Cf. Jorge Couto,
\textit{op. cit.}, p. 52.} têm língua por si. Outros
pelo mesmo sertão, que chamam \textit{Taicuyu}\footnote{ \textit{Taicuyu}: 
mais um grupo tribal para o qual não aparecem outras
referências posteriores.} vivem em casas, têm outra língua.
Outros no mesmo sertão, que chamam \textit{Cariri},\footnote{ \textit{Cariri}: 
outro grupo de língua diferente que só aparece
referido por Fernão Cardim. Pode ser identificado com \textit{kariri}, 
que significa em tupi ``taciturno'', ``pacífico'' ou ainda ``silencioso'',
o que se adequa com o que Cardim menciona de serem amigos dos
portugueses. Eram pertencentes ao tronco Macro"-Jê, que depois de terem
sido expulsos do litoral pelos Tupis, fixaram"-se nos sertões
nordestinos (serras da Borborema, dos Cariris Velhos, dos Cariris Novos
e vales do Acaraju, do Jaguaribe, do Açu, do Apodi e do baixo São
Francisco). Cf. Estevão Pinto, \textit{Os Indígenas do Nordeste}, vol.
\textsc{i}, São Paulo, 1935, pp. 115--117, cit. in Jorge Couto, \textit{op. cit.}, 
pp. 52--59.} têm língua diferente: estas três nações e seus
vizinhos são amigos dos portugueses. Outros que chamam \textit{Pigru},\footnote{ \textit{Pigru}: 
mais um grupo de difícil identificação
até porque sendo de língua diferente, pode ter tantas explicações que
se torna quase impossível decifrar.} vivem em casas. Outros
que chamam \textit{Obacoatiara},\footnote{ \textit{Obacoatiara}: 
idêntica situação, ainda que em abanheenga signifique ``cara pintada'',
o que não tem muito a ver com a descrição que Cardim faz deste povo.}
estes vivem em ilhas no Rio de São Francisco, têm casas como
cafuas debaixo do chão; estes quando os contrários vêm contra eles
botam"-se à água, e de mergulho escapam, e estão muito debaixo de água,
têm flechas grandes como chuços, sem arcos, e com elas pelejam; são
muito valentes, comem gente, têm diferente língua. Outros que vivem
muito pelo sertão a dentro, que chamam \textit{Anhehim},\footnote{ \textit{Anhehim}: 
nome de grupo tribal de difícil
identificação.} têm outra língua. Outros que vivem em casas,
que chamam \textit{Aracuaiati},\footnote{ \textit{Aracuaiati}: tal
como para muitos outros povos de difícil identificação, podemos
procurar explicar o seu nome através de várias línguas, o que se torna
muito pouco credível. Há mesmo etimologistas que encontram semelhanças
entre este nome e \textit{Araguaya}, que é o nome do grande rio de
Goiás.} têm outra língua. Outros que chamam \textit{Cayuara},\footnote{ \textit{Cayuara}: 
interpretando este nome à letra pode
significar ``comedor de caju'', de \textit{cayu} = ``caju'' + \textit{uara,
uhara ou guara} = ``comedor''. No entanto, ainda hoje se designam os
índios da região de Mato Grosso e de Goiás, \textit{Caayua} ou ``índios
de matas''.} vivem em covas, têm outra língua. Outros que
chamam \textit{Guaranaguaçu},\footnote{ \textit{Guaranaguaçu}: mais
um grupo tribal de difícil identificação, que traduzido com base no
termo utilizado por Cardim, como o povo que consumia \textit{Guaraná
(Paulinia cupana}, Mart.), da família das Sapindáceas, que é uma planta
de grande consumo pelos índios, devido às suas qualidades excitantes,
pelo seu conteúdo de cafeína e teobromina.} vivem em covas,
têm outra língua. Outros muito dentro no sertão que chamam
\textit{Camuçuyara},\footnote{ \textit{Camuçuyara}: o próprio nome
deste povo \textit{cam"-uçu"-yara} significava etimologicamente ``peitos
grandes que têm'' ou ``os que têm longas mamas'', o que segundo alguns
autores atuais podem ser as ``famosas'' Amazonas, o que vem
coincidir com a descrição de Fernão Cardim, ainda que não mencione o
fato de serem mulheres.} estes têm mamas que lhes dão por
baixo da cinta, e perto dos joelhos, e quando correm cingem"-nas na
cinta, não deixam de ser muito guerreiros, comem gente, têm outra língua. 

 Há outra nação que chamam \textit{Igbigra"-apuajara},\footnote{ \textit{Igbigra"-apuajara}: 
esta designação de Cardim para este grupo
tribal parece adequada, já que em abanheenga, este termo designa ``os
jogadores'' ou ``atiradores de paus''.} senhores de
paus agudos, porque pelejam com paus tostados agudos, são valentes,
comem gente, têm outra língua. Há outra que chamam 
\textit{Aruacuig},\footnote{ \textit{Aruacuig}: mais um nome de um grupo tribal de
difícil identificação.} vivem em casas, têm outra língua, mas
entendem"-se com estes acima ditos, que são seus vizinhos. Outros há que
chamam \textit{Guayacatu} e \textit{Guayatun},\footnote{ \textit{Guayacatu} 
e \textit{Guayatun}: mais dois nomes que não
figuram em outros textos.} estes têm língua diferente, vivem
em casas. Outros há que chamam \textit{Curupehé},\footnote{ \textit{Curupehé}: 
mesma situação dos anteriores.} não comem
carne humana, quando matam cortam a cabeça do contrário e levam"-na por
amostra, não têm casa, são como ciganos. Outros que chamam
\textit{Guayó},\footnote{ \textit{Guayó}: nome de difícil
identificação até porque segundo alguns etimologistas pode ser
\textit{guachis, guatós, huachis, goyá, coyá}, entre tantos outros
nomes.} vivem em casa, pelejam com flechas ervadas, comem
carne humana, têm outra língua. Outros que chamam \textit{Cicu},\footnote{ \textit{Cicu}: 
mais um nome de um grupo tribal de difícil 
identificação.} têm a mesma língua e costumes dos acima ditos.
Há outros a que chamam \textit{Pahaju},\footnote{ \textit{Pahaju}: 
também de difícil identificação este povo, que pode vir a ser
identificado, segundo alguns etimologistas, como Baptista Caetano, com
os \textit{Pacajás} das bocas do Amazonas, pois o nome tem uma certa
analogia.} comem gente, têm outra língua. Outros há que chamam
\textit{Jaicuju},\footnote{ \textit{Jaicuju}: tal como os outros
grupos tribais este nome é de difícil identificação.} têm a
mesma língua que estes acima. Outros que chamam 
\textit{Tupijó},\footnote{ \textit{Tupijó}: também não aparece identificado em outros
autores, no entanto, como o termo é formado por \textit{tupi} + o
sufixo \textit{jó (yóc}) reportando"-se ao quíchua, poder"-se"-ia
considerar como ``os valentes''.} vivem em casas, têm roças, e
têm outra língua. Outros \textit{Maracaguaçu},\footnote{ \textit{Maracaguaçu}: 
mais um nome difícil de identificação, já que não
aparece referenciado em outros autores e apenas etimologicamente
poderia ser encarado como \textit{maraca + guaçu} = ``maracá grande'', o
que não parece ser nome para ser atribuído a um grupo tribal.} 
são vizinhos dos acima ditos, têm a mesma língua. Outros chamam"-se
\textit{Jacuruju};\footnote{ \textit{Jacuruju}: outro nome que não
aparece referenciado em outros autores.} têm roças, vivem em
casa, têm outra língua. Outros que se chamam 
\textit{Tapuuys},\footnote{ \textit{Tapuuys}: este nome parece ainda mais difícil de
identificar pois ocorre no texto de Samuel Purchas como
\textit{Tapecuin}, para o qual também não conseguimos encontrar
explicação.} são vizinhos dos sobreditos acima, têm a mesma
língua. Outros há que chamam \textit{Anacuju},\footnote{ \textit{Anacuju}: mais um nome de grupo tribal de difícil
identificação, mas que sendo em quíchua pode ser explicado por
\textit{anacu} = ``manto'', ``capa'', portanto \textit{Anacuju} seria ``os
que têm capa ou manto''.} têm a mesma língua e costumes que
os de cima e todos pelejam com flechas ervadas. Outros que se chamam
\textit{Piracuju},\footnote{ \textit{Piracuju}: mais um grupo tribal
de difícil identificação.} têm a mesma língua que os de cima e
flechas ervadas. Outros há que chamam \textit{Taraguaig},\footnote{ \textit{Taraguaig}: idêntica situação para este grupo, pois tanto pode
ser em abanheenga \textit{teraqua} = ``famoso'' ou \textit{tirakua} = ``flecha'' + o sufixo \textit{ayg} = ``ervado'', conforme o sentido dado no
texto por Cardim, de que era um povo que pelejava com flechas
envenenadas com ervas.} têm outra língua, pelejam com flechas
ervadas. Há outros que chamam \textit{Panacuju},\footnote{ \textit{Panacuju}: outro nome de grupo tribal de difícil
identificação.} sabem a mesma língua dos outros acima ditos. Outros
chamam \textit{Tipe},\footnote{ \textit{Tipe}: outra situação
idêntica até porque existem nomes parecidos com este, como
\textit{tipeb} = ``nariz chato'' ou \textit{timbêb} e que poderia servir
para designar este povo; no entanto trata"-se apenas de
conjecturas.} são do campo, pelejam com flechas ervadas.
Outros há que chamam \textit{Guacarajara},\footnote{ \textit{Guacarajara}: tal como os anteriores este não figura em nenhum
autor, no entanto trata"-se de um nome que tem semelhanças com alguns
como \textit{Guacara} ou \textit{Guacari}, que são índios do tronco
tupi do Amazonas e rio Negro.} têm outra língua, vivem em
casas, têm roças. Outros vizinhos dos sobreditos que chamam
\textit{Camaragôã}.\footnote{ \textit{Camaragôã}: mais um nome de
grupo tribal de difícil explicação até porque pode"-se confundir com
\textit{camaraguar} que significaria etimologicamente ``comedor de
camará'', que é uma planta da família das Verbenáceas e das Solanáceas,
que segundo este autor era semelhante às silvas de Portugal.} 

 Há outros que chamam \textit{Curupyá},\footnote{ \textit{Curupyá}: 
outro grupo tribal de difícil identificação, apenas sabemos que eram
inimigos dos Tupinaquins, segundo a afirmação de Fernão Cardim.}
foram contrários dos \textit{Tupinaquins}, outros que chamam
\textit{Aquirinó},\footnote{ \textit{Aquirinó}: nome que não é
referido em outras obras e que, se interpretarmos o vocábulo, é uma frase
\textit{akir"-i"-nõ} = ``são covardes eles também''.} têm
diferente língua. Outros que chamam \textit{Piraguaygaguig},\footnote{ \textit{Piraguaygaguig}: outro grupo tribal de difícil identificação,
ainda que o seu nome, em abanheenga, apareça com uma significação
estranha já que é formado de \textit{piraqua} = ``valente'' +
\textit{aquy} = ``mole'', ``frouxo'', o que seria então ``o
forte"-fraco''.} vivem de baixo de pedras, são contrários dos de
cima ditos. Outros que chamam \textit{Pinacuju}.\footnote{ \textit{Pinacuju}: grupo tribal de difícil identificação até porque
Cardim poderá estar referindo"-se ao mesmo povo com nomes semelhantes e
que são apresentados neste texto, como \textit{Panacuju, Anacuju, 
Piracuju} ou \textit{Raracuju.}} Outros há que chamam
\textit{Parapotô},\footnote{ \textit{Parapotô}: também de difícil
identificação.} estes sabem a língua dos do mar. Outros
\textit{Caraembâ},\footnote{ \textit{Caraembâ}: este nome torna"-se
muito difícil de explicar já que o termo \textit{Cará} ocorre em
numerosos vocábulos e em nomes de diversos grupos tribais, e tem
significados diferentes consoante a língua. Assim, entre outras
referências, note"-se que na língua dos Aimarás \textit{kara, cara,} 
significa ``pelado'', ou ainda, ``de uma só cor'', ``pintas'', ``manchas''; no
quíchua há ainda \textit{ccara} = ``couro'', entre outros significados.
Põe"-se ainda a questão que \textit{Cará} é o nome de várias plantas da
família das dioscoreáceas, como a mangará e semelhantes aos nabos. }
têm outra língua. Outros que chamam 
\textit{Caracuju},\footnote{ \textit{Caracuju}: mais um nome de difícil explicação,
muito semelhante ao anterior e que poderá encontrar as interpretações
mais estranhas nas diversas línguas indígenas.} têm outra
língua. Outros que chamam \textit{Mainuma},\footnote{ \textit{Mainuma}: 
outro nome de grupo tribal de difícil identificação e que se for na
língua abanheenga só para \textit{ma} existem várias explicações muito
diferentes.} estes se misturam com \textit{Gaimurês},\footnote{ \textit{Guaimurê}: 
vide nota supra sobre este grupo tribal.} contrários dos do mar; entendem"-se 
com os \textit{Gaimurês}, mas têm outra língua. Outros há que chamam 
\textit{Aturay}\footnote{ \textit{Aturay}: torna"-se complicado identificar este povo já que
se encontram referências a nomes idênticos em abanheenga com
\textit{atiriri} = ``pequenino'', ``murcho'', ``encolhido'' e \textit{atur},
em tupi, que significa ``curto'', ``breve'', entre outros significados e
entre outras línguas indígenas.} também entram em comunicação
com os \textit{Guaimurês}. Outros há que chamam 
\textit{Quigtaio},\footnote{ \textit{Quigtaio}: este nome de grupo tribal, tal como
todos os que começam por \textit{Q} não figuram em outros
autores.} também comunicam e entram com os \textit{Guaimurês.}
Há outros que chamam \textit{Guigpé};\footnote{ \textit{Guigpé}: 
outro grupo tribal de difícil identificação até porque pode ser
escrito desta forma ou \textit{Quipgé}, o que pode conduzir a várias
conjecturas} estes foram moradores de Porto Seguro. Outros se
chamam \textit{Quigrajubê},\footnote{ \textit{Quigrajubê}: mais um
grupo tribal cujo nome se inicia por \textit{Q}, o que torna difícil
de identificar, já que não são mencionados em outros autores.} 
são amigos dos sobreditos. Outros que chamam
\textit{Angararî},\footnote{ \textit{Angararî}: também não aparece
referência para este grupo tribal em outros autores. Literalmente
pode"-se traduzir o termo \textit{angarory} por ``alma alegre''.} 
estes vivem não muito longe do mar, entre Porto Seguro e o Espírito
Santo. Outros que chamam \textit{Amixocori}\footnote{ \textit{Amixocori}: 
idêntica situação dos anteriores nomes mencionados
por Cardim ao longo deste texto.} são amigos dos de cima. Há
outros que chamam \textit{Carajâ},\footnote{ \textit{Carajâ}: 
possivelmente trata"-se dos \textit{Carijós} já descritos por Cardim,
que ``[\ldots{}] correm pela costa do mar e sertão até o Paraguai''. É esta a
mesma opinião de Baptista Caetano, que considera que os \textit{Carajás}
e \textit{Carijós} são nomes do mesmo povo, incluindo ainda os Carijós e
Carajás de Goiás e do Araguaia. Cf. Notas da obra de Fernão Cardim,
\textit{op. cit}, ed. de 1980, p. 118.} vivem no sertão da
parte de São Vicente; foram do Norte correndo para lá, têm outra
língua. Há outros que chamam \textit{Agiputá};\footnote{ \textit{Agiputá}: 
mais um nome que não figura nas listas de grupos
tribais de outros autores e que poderá ter várias explicações pelo
abanheenga, concretamente o adjetivo \textit{apitupa} que significa
``os desalentados'', ``os desanimados''.} vivem no sertão para a
banda de \textit{Aquitipi.}\footnote{ Trata"-se possivelmente de uma
região, atendendo ao contexto, e não de mais um grupo tribal, no
entanto não conseguimos localizá"-la.} Outros há que chamam
\textit{Caraguatajara};\footnote{ \textit{Caraguatajara}: outro grupo
tribal de difícil identificação, ainda mais que, se se traduzir
literalmente do tupi, ocorre ``senhor das bromélias'', de
\textit{caraguata} = ``planta espinhosa que produz frutos amarelos em
cachos fortemente ácidos'', possivelmente ``bromélias'' + \textit{jara} ou
\textit{yara} = ``senhor''. O próprio Cardim refere"-se a estas plantas
\textit{caraguatá} no seu texto sobre ``\textit{Do Clima e Terra do
Brasil\ldots{}''}, onde o termo tupi ocorre pela primeira vez num texto
português.} têm língua diferente. Há outros que chamam
\textit{Aguiguira},\footnote{ \textit{Aguiguira}: outro grupo tribal
de difícil identificação, até porque poderá ser \textit{akiguira}, ou
\textit{aquiguira}, ou outras variantes. Se decorrer de
\textit{aquiqui} poderá querer referir"-se a bugios, ou seja,
macacos.} estes estão em comunicação com os acima ditos. Outra
nação há no sertão contrária dos \textit{Muriapigtanga}\footnote{ \textit{Muriapigtanga}: 
outro nome de grupo tribal que não ocorre em
outros textos e que segundo Baptista Caetano trata"-se de um vocábulo
abanheenga, tal como \textit{Apigapigtanga}, para os quais se poderiam
encontrar várias explicações, entre as quais a analogia com
\textit{Tapigapigtanga}, que significaria ``cabilda de selvagens'', ``de
inimigos'', de Tapuias.} e dos 
\textit{Tarapé},\footnote{ \textit{Tarapé}: outro grupo tribal que não figura nos
outros autores e que traduzido literalmente seria \textit{ta"-rapé} = 
``o caminho da povoação'', ou se for \textit{taa} = ``senhor'' +
\textit{rapé} = ``caminho'', ``senhor do caminho'', o que não são nomes
para designar um grupo tribal. Mas atendendo ao texto cardiniano que
vem a seguir, designando estes índios como de pequena estatura, poderá
ser \textit{carapé} que designaria ``os chatos'', ``os baixos'' e
``truculentos''.} é gente pequena, anã, baixos do corpo, mas grossos de
perna e espáduas, a estes chamam os portugueses Pigmeu, e os índios lhe
chamam \textit{Tapig"-y"-mirim},\footnote{ \textit{Tapig"-y"-mirim}: pela
descrição do texto cardiniano trata"-se dos Pigmeus, designados na nota
anterior por \textit{Tarapé.} Traduzindo"-se literalmente podemos
encontrar o termo genérico \textit{tapyi} = ``grupo tribal de raça
diferente, de língua diferente'' + \textit{mirim} = ``pequeno'', ou seja,
``grupo tribal de homens pequenos'', o que coincide com a descrição de
Cardim.} porque são pequenos. Outros há que chamam
\textit{Quiriciguig},\footnote{ \textit{Quiriciguig}: tal como já foi
referido, os nomes começados por \textit{Q} não figuram nos outros
autores.} estes vivem no sertão da Bahia, bem longe. Outros
que chamam \textit{Guirig}\footnote{ \textit{Guirig} ou \textit{Quirig}:  
mais um nome de grupo tribal de difícil identificação, ainda mais
porque Cardim atribui"-lhes o atributo de cavaleiros.} são
grandes cavaleiros e amigos dos ditos acima.

 Outros se chamam \textit{Guajerê};\footnote{ \textit{Guajerê}: outro
nome de difícil identificação e que segundo Baptista Caetano tem
indícios de ser muito alterado, talvez com troca ou até mesmo erro de
sílabas.} vivem no sertão de Porto Seguro muito longe. Há
outra nação que chamam \textit{Aenaguig};\footnote{ \textit{Aenaguig}: 
também não aparece referenciado em outros autores este nome de grupo
tribal, pode"-se, no entanto, interpretar o seu nome como sendo ``o
descendente do outro'', entendendo"-se \textit{ae} como adjetivo = ``o
outro'', ``o diferente'' e \textit{aguig} esteja por \textit{aqui} ou
\textit{oqui} = ``colateral'' ou ``derivado''.} estes foram
moradores da terra dos \textit{Tupinaquins},\footnote{ Vide nota
supra.} e porque os Tupinaquins ficaram senhores das terras se
chamam \textit{Tupinaquins.} Há outros que chamam 
\textit{Guaytacâ};\footnote{ O nome deste grupo tribal \textit{Guaytacâ} tem sido
justificado como sendo os \textit{Goitacá} que era uma família
diferente das do tronco Tupi, que provinham do tronco Macro"-Jê e que
viviam no trecho de costa compreendido entre o rio Cricaré e o cabo de
São Tomé, ocupando também o interior dessa região. Etimologicamente o
termo \textit{Guaytacâ} significa o ``nômada'', ``errante'', ``sem paradeiro
certo'', o que coincide com a descrição de Cardim.} estes vivem
na costa do mar entre o Espírito Santo e Rio de Janeiro; vivem no campo
e não querem viver nos matos e vão comer às roças, vêm dormir às roças,
vêm dormir às casas, não têm outros tesouros, vivem como o gado que
pasce no campo, e não vêm às casas mais que a dormir; correm tanto que
a cosso tomam a caça. Outros que chamam 
\textit{Igbigranupâ},\footnote{ \textit{Igbigranupâ}: 
outro grupo tribal que não ocorre em outros
textos sobre índios, mas que segundo Baptista Caetano é um nome do
abanheenga, mas que pode ter as mais diversas interpretações, entre
elas ``os bate"-pau'', ``os joga"-pau'', que está de acordo com o hábito que
estes povos tinham de guerrear, batendo com os paus nos outros, segundo
o texto cardiniano.} são contrários dos \textit{Tupinaquins} e
comunicam com os \textit{Guaimurês};\footnote{ \textit{Guaimuré}: vide
nota supra sobre este grupo tribal.}  quando justam com os
contrários fazem grandes estrondos, dando com uns paus nos outros.

 Outros que chamam \textit{Quirigmã},\footnote{ \textit{Quirigmã}:  
segundo Baptista Caetano este nome é o adjetivo \textit{Kyreymbá} que
designa ``os valentes'', ``os valorosos'', ``os esforçados''.} estes
foram senhores das terras da Bahia e por isso se chama a Bahia
\textit{Quigrigmurê.}\footnote{ \textit{Quigrigmurê}: como aparece
referido no texto parece designar ``lugar'', mas como nome de grupo
tribal aplicado ao lugar tem analogia com \textit{Quinimuré} ou
\textit{Quinimurá} e assim significaria ``índios navegantes do norte do
Brasil'', segundo Batista Caetano.} Os Tupinabas os botaram de
suas terras e ficaram senhores delas, e os \textit{Tapuyas} foram para
o Sul. Há outros que chamam \textit{Maribuió};\footnote{ \textit{Maribuió}: 
outro grupo tribal de difícil identificação até
porque Cardim não menciona nada a seu respeito.} moram no
sertão em direito do Rio Grande. Outros que chamam 
\textit{Cataguá};\footnote{ \textit{Cataguá}: não aparece identificado em outros
autores, mas etimologicamente o nome poderá designar ``índios dos
confins'', ``do mato'', já que \textit{caá} = ``mato'', ``árvore'' +
\textit{tã} = ``duro'' + \textit{guá} = ``vale''. Aparece identificado no
\textit{Vocabulário Tupi"-Guarani"- Português}, de Francisco da Silveira
Bueno, como sendo um grupo tribal de Minas Gerais.} esses
vivem no direito de Jequericarê\footnote{ \textit{Jequericarê} ou
\textit{Juqueriquerê} é um rio do estado de São Paulo, de
\textit{Ykei"-ker"-ê} = ``rio salgado''.} entre o Espírito Santo e Porto
Seguro. Outros há que chamam 
\textit{Tapuxerig};\footnote{ \textit{Tapuxerig}: atendendo ao texto 
cardiniano pode"-se interpretar
como sendo \textit{Tapyyi"-cury} que significa ``o tapuia que escorrega''
ou ``que se escafede'', ou ainda, ``o adversário que se safa''.} 
são contrários dos outros Tapuyas, comem"-lhes as roças. Outros que
moram pelo sertão que vai para São Vicente, chamam"-se
\textit{Amocaxô},\footnote{ \textit{Amocaxô}: este nome de grupo
tribal também não vem em nenhuma das outras obras sobre os índios do
Brasil. No entanto, o sufixo \textit{amo}, entre outros significados,
tem o de ``longe'', ``lá'', o que condiz com a descrição
cardiniana.} foram contrários dos \textit{Tupinaquins}. Outros
que chamam \textit{Nonhã},\footnote{ \textit{Nonhã}: não é nome de
grupo tribal que figure nos outros autores e também não é fácil
encontrar pela descrição cardiniana alguma semelhança no vocabulário
tupi"-guarani, ou outro ameríndio, a não ser \textit{toba ña} ou
\textit{tobaya} como ``cara aberta'', ``cara larga''.} têm rostos
muito grandes. Há outros, e estes se chamam \textit{Apuy},\footnote{ \textit{Apuy}: 
nome de difícil identificação já que não ocorre em
nenhuma obra e com a descrição de Cardim, de ``cantor'', não há
referência em nenhuma língua.} moram perto do campo do sertão,
são grandes cantores, têm diferente língua. Outros há que chamam
\textit{Panaquiri},\footnote{ \textit{Panaquiri}: também não é
mencionado em outros autores, e como Cardim não indica nenhuma
referência não é fácil identificar este grupo tribal. O próprio termo
\textit{pana} pode"-se explicar de diversas formas pelo abanheenga,
mas como nome de grupo tribal parece estar mais próximo do quíchua,
onde significa ``irmão'', ``irmã''. No Javari, junto da fronteira do Peru,
menciona"-se um grupo tribal com o nome de Pano, do grupo de famílias
menores estabelecidas a sul do Amazonas.} diferente dos acima ditos.
Outros também diferentes que chamam 
\textit{Bigvorgya.}\footnote{ \textit{Bigvorgya}: nome de grupo tribal de difícil
identificação.} Há outra nação que chamam 
\textit{Piriju},\footnote{ \textit{Piriju}: mais um nome na mesma situação até porque
Cardim não menciona nada a seu respeito, e a nível das diversas línguas
pode"-se identificar de várias formas, como em tupi \textit{pira"-jyg} = 
``couro rijo''; em guarani \textit{pirajub} = ``pele amarela'', entre
outras muitas explicações.} e destes há grande número. 

 Todas estas setenta e seis nações de Tapuias, que têm as mais delas
diferentes línguas, são gente brava, silvestre e indômita, são
contrários quase todas do gentio que vive na costa do mar, vizinho dos
portugueses; somente certo gênero de Tapuias que vivem no Rio São
Francisco, e outros que vivem mais perto são amigos dos portugueses, e
lhes fazem grandes agasalhos quando passam por suas terras. Destes há
muitos cristãos que foram trazidos pelos Padres do sertão, e aprendendo
a língua dos do mar que os Padres sabem, os batizaram e vivem muitos
deles casados nas aldeias dos Padres, e lhes servem de intérpretes para
remédio de tanto número de gente que se perde, e somente com estes
Tapuias se pode fazer algum fruto; com os mais Tapuias, não se pode
fazer conversão por serem muito andejos e terem muitas e diferentes
línguas dificultosas. Somente fica um remédio, se Deus Nosso Senhor não
descobrir outro, e é havendo às mãos alguns filhos seus aprenderem a
língua dos do mar, e servindo de intérpretes fará algum fruto ainda que
com grande dificuldade pelas razões acima ditas e outras 
muitas.\footnote{ Uma das primeiras preocupações do padre Manuel da
Nóbrega consistiu em instruir os missionários na língua utilizada pelos
índios, pelo que incumbiu o padre João de Azpilcueta Navarro de a
aprender. Por outro lado, encarregou o irmão Vicente Rodrigues de
ministrar a doutrina cristã aos filhos dos indígenas e de assegurar o
funcionamento de uma ``escola de ler e escrever'', destinada tanto aos
filhos dos colonos como aos dos índios. Mas enquanto não dominavam o
tupi, os inacianos pregavam, doutrinavam e confessavam com recurso a
intérpretes, utilizando designadamente os serviços de Diogo Álvares, o
Caramuru. Cf. Jorge Couto, ``Padre Manuel da Nóbrega'', in
\textit{História de Portugal}, coord. João Medina, vol. \textsc{v}, Lisboa,
Ediclube, 1993, pp. 162--172.} 


